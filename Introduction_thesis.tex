\chapter{\label{introduction}Introduction}
The standard model of particle physics (SM) is a mathematical theory that describes three of the four fundamental interactions among the elementary particles that make up matter. This model, which emerged from experimental discoveries and theoretical advances in the 1960s-70s, has been enormously successful in describing the empirical findings of particle interactions. 

One of the most exciting experimental observations of the 20th century has been the confirmation of an asymmetry between matter and anti-matter, the so called charge-parity (\textit{CP}) violation, in the decay of neutral $K$ mesons in 1964~\cite{CPV_kaon}. It is incorporated into the SM in the form of an irreducible complex phase in the quark-mixing matrix (also called the Cabibbo-Kobayashi-Maskawa (CKM) matrix)~\cite{ckm}. It has been shown that \textit{CP} violation is a necessary ingredient in explaining the dominance of matter over anti-matter in the universe~\cite{CPV_m_am}. However, the amount of \textit{CP} violation predicted by the SM is not enough to account for the observed asymmetry. 

In 1980, it was shown that the CKM mechanism predicted large \textit{CP} violating effects in certain \B meson decays as well~\cite{CPV_B1,CPV_B2}. This led to the development of the dedicated \B-meson factories PEPII~\cite{PEPII} at SLAC and KEKB~\cite{kekb} at KEK, where the Babar~\cite{babar_detector} and Belle~\cite{belle_detector} detectors are located, respectively. These experiments became operational in 1999 and have accumulated enormous quantities of data at the \upsFour resonance. At the time of writing, the BaBar experiment has accumulated 433~\invfb, while the Belle experiments has accumulated a world record of 848~\invfb of data. These data sets correspond to 476 and 932 million \BBbar pairs, respectively. With these samples it is possible to test the SM by probing \B-meson decays with unprecedented precision. Several results have, in fact, pushed the limits of the SM and may require beyond the SM (BSM) explanations~\cite{kpiDCPV_babar, kpiDCPV_belle}. 


\section{Hadronic \bm{\B} Meson Decays}
The study of branching fractions and angular distributions of \B-meson decays to hadronic final states tests our understanding of both weak and strong interactions. Specifically, \B-meson decays to two vector mesons (\Bvv decays) can shed light on the helicity structure of weak interactions of quarks through polarization studies. Recently, \B decays mediated by \bsqq penguin amplitudes have received much attention in the literature. Unlike \bc spectator amplitudes (which are much better measured), penguin amplitudes contain an internal loop and thus are potentially sensitive to new propagators and couplings. Such mediating particles may have an energy scale too high to access directly. Several measured \bsqq decays have yielded unexpected results; e.g., the decays \BToPhiKst and \BToRhoKstz are found to have large transverse polarization~\cite{belle_phiKstz,babar_phiKstz,belle_RhoPKstz,babar_RhoPKstz}, and $B$ decays to the closely related final states \KpPim and \KpPiz exhibit different patterns of direct \textit{CP} violation~\cite{kpiDCPV_babar, kpiDCPV_belle}. These results are difficult to accommodate within the Standard Model and may indicate the presence of new physics~\cite{Kagan QCD peng annih,Ladisa etal FSI,Cheng Chua Soni FSI,Colangelo FSI,charming penguins,magnetic penguin,new sets of form factors,Beneke etal Enhanced Electroweak Penguin Amplitudes,Li and Mishima Polarization in Bvv decays,Cheng Yang BR and Polarization in Bvv decays,Grossman Beyond the SM with B and K physics,RH currents,scalar interactions,NP operators}. Furthermore, \bsqq decays are useful for determining the angles ${\phi_{2}}$ and ${\phi_{3}}$ of the unitarity triangle~\cite{phi3_ref1,phi3_ref2,phi3_ref3}.

In this Thesis we present a study of the \bsdd decay \BzToOmeKstz. This rare \Bvv decay has yet to be observed. Measurements of the branching fraction and angular distributions will provide insight into the unexpected results obtained for other \bsqq decays.

%In this Thesis we present a study of the \bsdd decay \BzToOmeKstz. This is a rare \Bvv decay which has yet to be observed. Measurements of the branching fraction and angular distributions will provide important information necessary to gain insight into the anomalous result obtained for other \bsqq decays.

\subsection{Decay Rate}
To obtain the decay rate of a \B meson to some final state $f$, it is necessary to calculate the transition amplitude \calM for \BtoF. There are many possible contributions to \calM, each of which can be pictorially represented by a Feynman diagram. Purely hadronic \B-meson decays proceed mainly through the weak decay of a $b$ quark to a lighter quark, while the light quark bound in the \B meson remains a spectator. In Fig.\ref{b_decay_feynman_diagrams}, we show the lowest-order Feynman diagrams for non-leptonic $b$ decays. In addition to the acyclic connected current-current ``tree" diagrams, there are two types of one-loop flavor-changing neutral current ``penguin" topologies: gluonic (QCD) and electroweak (EW) penguins originating from strong and electroweak interactions, respectively. Decays in which the spectator quark hadronizes with one of the quarks produced from the $W$ ($Z,g,\gam$) in the tree (penguin) decay are referred to as {\it internal}, whereas hadronization of the spectator quark with the decay product of the $b$ quark are referred to as {\it external}.

A very useful tool to analyze \B decays are low energy effective hamiltonians \Heff evaluated in renormalization group improved perturbation theory~\cite{Fleischer}. To evaluate \Heff, one makes use of the ``Operator Product Expansion" (OPE)~\cite{OPE} which involves field operators \Qk describing both tree and penguin (QCD and EW) transitions. In general, the matrix elements for transitions from initial state $i$ to final state $f$ are of the form
\begin{equation}
\OPE,
\end{equation}
where \HME are the non-perturbative hadronic matrix elements describing the ``long-distance" contributions to the decay amplitude, \Ck are the perturbatively calculable Wilson coefficient functions describing the ``short-distance" contributions, and $\mu$ denotes an appropriate normalization scale. 
%where \Ck are the perturbatively calculable Wilson coefficient functions, and $\mu$ denotes an appropriate normalization scale (in this case the $b$ quark mass $m_{b}$). For transitions of the form \DbEqOne, \DcEqDuEqZero, we have

In the case of \B meson decays, we have transitions of the form \DbEqOne, \DcEqDuEqZero. The effective hamiltonian corresponding to these decays is given by
%For transitions of the form \DbEqOne, \DcEqDuEqZero, we have
\begin{equation}
\HeffFull,
\end{equation}
where \GF denotes the Fermi constant, the index \qqInds corresponds to \bd and \bs transitions, respectively, and $\mu$ is taken to be of $O(m_{b})$. The field operators are divided into three categories based on the aforementioned decay topologies:
%The field operators are divided into their afrorementioned three categories:
\begin{enumerate}
\item{Tree operators:}
\begin{eqnarray}
\Qone \nonumber \\
\Qtwo 
\end{eqnarray}

\item{QCD penguin operators:}
\begin{eqnarray}
\Qthree \nonumber \\
\Qfour \nonumber \\
\Qfive \nonumber \\
\Qsix 
\end{eqnarray}

\item{EW penguin operators:}
\begin{eqnarray}
\Qseven \nonumber \\
\Qeight \nonumber \\
\Qnine \nonumber \\
\Qten
\end{eqnarray}
\end{enumerate}
Here $\alpha$ and $\beta$ denote the SU(3) color indices, \qqdef, \qqp runs over the quark flavors being active at the scale \muOmb, i.e. \qqpInudcs, and \eqqp are the corresponding electrical quark charges~\cite{Fleischer}.

%The theoretical descriptions of hadronic \B-meson decays are complicated by the fact that the initial and final states are linked by various trees of gluon and quark interactions, pair production, and loops. As a result, the {\it factorization hypothesis}~\cite{Fact: Bjorken}, in which amplitudes factorize into products of two current matrix elements, is widely used in heavy-quark physics~\cite{BBk}. In factorization, the currents are separated by inserting the vacuum state and disregarding any QCD interactions between them. Thus, for hadronic \B meson decays to two vector mesons, the matrix elements \BvvBraKet are factorized into the products of two factors \BvvBraKetFact, where the bilinear quark current operators \JOneTwo can be calculated in more well-defined theoretical frameworks [SUCH AS....maybe ref the babar thesis]. This assumption is expected to be valid [SEE WHAT BABAR BOOK SAYS AND MAYBE REF DPI] for external spectator decays where the large energy carried by the $W$ bosons causes the products of the $W$ to be well separated from the decay product of the $b$ quark and the spectator quark~\cite{Stone, Fact: Bjorken, Fact: Dugan}. Factorization has also been used to calculate the rates for color-suppressed and penguin decays, though the predictions and the experimental results are not always in agreement (as described in Sec. \ref{Penguin level Bvv decays}). The experimental sensitivity in \B factories has now become sufficient to allow us to test the correctness of the factorization hypothesis by measuring the decay rates and polarization of an increased number of rare \B meson decays which have branching fractions of $O(10^{-6})$.
%The experimental sensitivity in \B factories has now become sufficient to allow us to begin testing the correctness of these underlying assumptions.

%The theoretical descriptions of hadronic \B-meson decays are complicated by the fact that the initial and final states are linked by various trees of gluon and quark interactions, pair production, and loops. As a result, the {\it factorization hypothesis}~\cite{Fact: Bjorken}, in which amplitudes factorize into products of two current matrix elements, is widely used in heavy-quark physics~\cite{BBk}. In factorization, the currents are separated by inserting the vacuum state. Thus, for hadronic \B meson decays to two vector mesons, the matrix elements \BvvBraKet are factorized into the products of two factors \BvvBraKetFact, where the bilinear quark current operators \JOneTwo can be calculated in more well-defined theoretical frameworks [SUCH AS....maybe ref the babar thesis]. Under ``naive" factorization (FA) any QCD interactions between are ignored. This assumption simplifies the calculations, but has the limitiation  This assumption is expected to be valid [SEE WHAT BABAR BOOK SAYS AND MAYBE REF DPI] for external spectator decays where the large energy carried by the $W$ bosons causes the products of the $W$ to be well separated from the decay product of the $b$ quark and the spectator quark~\cite{Stone, Fact: Bjorken, Fact: Dugan}. FA has also been used to calculate the rates for color-suppressed and penguin decays, and in many cases, it provides the correct order of magnitude for the branching fractions. However, due to the neglect of QCD interactions, FA cannot predict direct \textit{CP} asymmetries. Additionally, it has been found that predictions for \fL under FA are in strong disagreement with the experimental results for several channels (see discussion in Sec. \ref{Penguin level Bvv decays}). 
%Thus FA is no longer adequate for a detailed phenomenological analysis of \B-factory data [QUOTE NEW NEUBERT].

%%%%%%%%%%%%%%%%%%%%%%%%%%%%%%%%%%%%%%%%%%%%
%%%%%%%%%%%%%%   b-DECAY FEYNMAN DIAGRAM   %%%%%%%%%%%%
%%%%%%%%%%%%%%%%%%%%%%%%%%%%%%%%%%%%%%%%%%%%
\begin{figure}[t]
\centerline{
\hbox{
\includegraphics[width=0.90\columnwidth,height=!]{b_decay_fd.pdf}\hspace{0.2cm} }}
\caption{Lowest order contributions to non-leptonic $b$-quark decays ($q\in\{u,c,t\}$). \label{b_decay_feynman_diagrams}}
\vskip -0.2cm
\end{figure}

Theoretical descriptions of hadronic \B-meson decays are complicated by the fact that the initial and final states are linked by various trees of gluon and quark interactions, pair production, and loops. As a result, the {\it factorization hypothesis}~\cite{Fact: Dugan}, in which amplitudes factorize into products of two current matrix elements, is widely used in heavy-quark physics~\cite{BBk}. In the ``naive" factorization approach (FA)~\cite{FA1,FA2}, the currents are separated by inserting the vacuum state and disregarding any QCD interactions between them. Thus, for hadronic \B meson decays to two vector mesons, the matrix elements \BvvBraKet are factorized into the products of two factors \BvvBraKetFact, where the bilinear quark current operators \JOneTwo can be calculated in more well-defined theoretical frameworks. This assumption is valid for external spectator decays such as the \bTocubard mediated \BToDPi decay, where the large energy carried by the $W$ bosons causes the products of the $W$ to be well separated from the decay product of the $b$ quark and the spectator quark~\cite{Stone, Fact: Dugan}. FA has also been used to calculate the rates for color-suppressed and penguin decays, and in many cases, it provides the correct order of magnitude for the branching fractions. However, due to the neglect of QCD interactions, FA cannot predict direct \textit{CP} asymmetries. Additionally, it has been found that predictions for \fL under FA are in strong disagreement with the experimental results for several channels, notably, decays which proceed via a \DsEqOne transition (see discussion in Sec. \ref{Penguin level Bvv decays}). Presently, the only \bsqq dominated \Bvv channels which have been measured are \BToPhiKst and \BToRhoKstz decays~\cite{belle_phiKstz,babar_phiKstz,belle_RhoPKstz,babar_RhoPKstz}. The remaining channels are heavily suppressed and thus very large statistics are required to accurately measure the decay rates, polarizations and \textit{CP} asymmetries. The large samples of accumulated \BBbar events at the \B-factories are now providing sensitivity to an increased number of these rare \B-meson decays which have branching fractions of $O(10^{-6})$ and beyond.

%IDENTICAL TO ABOVE BUT INCLUDES BJORKEN REFERENCE (BUT I CAN'T FIND THE PAPER)
%Theoretical descriptions of hadronic \B-meson decays are complicated by the fact that the initial and final states are linked by various trees of gluon and quark interactions, pair production, and loops. As a result, the {\it factorization hypothesis}~\cite{Fact: Bjorken,Fact: Dugan}, in which amplitudes factorize into products of two current matrix elements, is widely used in heavy-quark physics~\cite{BBk}. In the ``naive" factorization approach (FA)~\cite{FA1,FA2}, the currents are separated by inserting the vacuum state and disregarding any QCD interactions between them. Thus, for hadronic \B meson decays to two vector mesons, the matrix elements \BvvBraKet are factorized into the products of two factors \BvvBraKetFact, where the bilinear quark current operators \JOneTwo can be calculated in more well-defined theoretical frameworks. This assumption is valid for external spectator decays such as the \bTocubard mediated \BToDPi decay, where the large energy carried by the $W$ bosons causes the products of the $W$ to be well separated from the decay product of the $b$ quark and the spectator quark~\cite{Stone, Fact: Bjorken, Fact: Dugan}. FA has also been used to calculate the rates for color-suppressed and penguin decays, and in many cases, it provides the correct order of magnitude for the branching fractions. However, due to the neglect of QCD interactions, FA cannot predict direct \textit{CP} asymmetries. Additionally, it has been found that predictions for \fL under FA are in strong disagreement with the experimental results for several channels, notably, decays which proceed via a \DsEqOne transition (see discussion in Sec. \ref{Penguin level Bvv decays}). Presently, the only \bsqq dominated \Bvv channels which have been measured are \BToPhiKst and \BToRhoKstz decays~\cite{belle_phiKstz,babar_phiKstz,belle_RhoPKstz,babar_RhoPKstz}. The remaining channels are heavily suppressed and thus very large statistics required to accurately measure the decay rates, polarizations and \textit{CP} asymmetries. The large samples of accumulated \BBbar events at the \B-factories are now providing sensitivity to an increased number of these rare \B-meson decays which have branching fractions of $O(10^{-6})$ and beyond.

%....well-defined theoretical frameworks [SUCH AS....MAYBE REF BABAR BOOK]


The rare decay \BzToOmeKstz undergoes a \DsEqOne transition and has Cabibbo supressed internal tree-level \bsuu, and dominant internal gluonic penguin-level \bsdd contributions. The corresponding Feynman diagrams are depicted in Fig.\ref{wKst_feynman_diagrams}. The tree contribution is heavily suppressed and thus we generally refer to \BzToOmeKstz as a \bsdd penguin decay. Due to the many uncertainties in the theoretical predictions for \Bvv decays (e.g., CKM matrix elements, form factors, quark masses, etc.), the branching fractions for these channels are usually reported as ranges. For \BzToOmeKstz decays, the theoretical calculations for the branching fraction cover the range \TheoryBrRange~\cite{ThPred: Ali Kramer, ThPred: Chen, ThPred: Cheng Yang, ThPred: Zou Xiao, ThPred: Beneke Rohrer Yang, Cheng Yang BR and Polarization in Bvv decays}. 

%Recently, the authors of~\cite{ThPred: Beneke Rohrer Yang} have predicted ---- and the authors of~\cite{Cheng Yang BR and Polarization in Bvv decays} have predicted... [MAYBE ALSO MENTION~\cite{ThPred: Zou Xiao} AND SAY IT'S A NP CALCULATION USING SUSY]


Experimentally, this channel has yet to be observed; only 90\% confidence level (C.L.) upper limits (U.L.) have been established. The first measurement was made by the CLEO collaboration. Using a data sample of \DataSetCLEO, \YieldCLEO events were obtained and an upper limit \BrCLEO~\cite{CLEO_wKstz} was set. With a data set of \DataSetBabarOld the BaBar collaboration obtained \YieldBabarOld events and measured \BrBabarOld with \SigBabarOld significance~\cite{babar_wKstz_1}. Recently, with an increased data sample of \DataSetBABAR, the BaBar collaboration obtained \YieldBabar events and measured \BrBabar with \SigBabar significance~\cite{babar_wKstz_2}. No further reports have been made. The analysis reported here uses \dataset of data containing \NBBbarVal \BBbar pairs; this sample is almost three times larger than that used in BaBar's most recent measurement~\cite{babar_wKstz_2}.

%say that while factorization is a good approx for tree processes (give example of B-Dpi), it's not entirely clear if it should hold for penguin dominated decays. Close this section by saying that this can be tested through polarization studies. 

%%%%%%%%%%%%%%%%%%%%%%%%%%%%%%%%%%%%%%%%%%%%
%%%%%%%%%%%%%%%   WKSTZ FEYNMAN DIAGRAM   %%%%%%%%%%%%%
%%%%%%%%%%%%%%%%%%%%%%%%%%%%%%%%%%%%%%%%%%%%
\begin{figure}[h]
\centerline{
\hbox{
\includegraphics[width=0.60\columnwidth,height=!]{wKst_feynman_diagrams.pdf}\hspace{0.2cm} }}
\caption{Tree (a) and penguin (b) Feynman diagrams for \BzToOmeKstz decays. \label{wKst_feynman_diagrams}}
\vskip -0.2cm
\end{figure}

\section{Angular Dependence \label{pol th}}
Decays of \B mesons into two vector mesons can shed light on the helicity structure of weak interactions of quarks through polarization studies. There are three different angular momentum projections that can be used: the helicity basis, the transversity basis and the partial wave decomposition. The helicity formalism allows for a direct determination of the longitudinal rate, while the transversity formalism is most useful for \textit{CP} analysis since the states are parity eigenstates. 

In this section we review the central concepts and derive the critical expressions in the helicity basis, which will be used in the angular analysis of \BzToOmeKstz decays. For completeness, we also briefly discuss the transversity and partial wave decomposition formalisms, though neither set of amplitudes is explicitly used in this analysis.

%In this section we review the central concepts and derive the critical expressions in the helicity basis, which will be used in the angular analysis of \BzToOmeKstz decays. For completeness, we briefly discuss the transversity and partial wave decomposition formalisms, but refer the reader to e.g., BABAR book, for more details.


% They are equivalent, but each with different physical interpretations. 
%A detailed description of the .... can be found [-----].


\subsection{Helicity Basis\label{Helicity basis}}
Let us consider the \BzToOmeKstz decay in the rest frame of the \Bz, where the axis of spin quantization is taken to be the decay axis of the \Ome and \Kstz mesons ($z$ axis) in the \Bz rest frame. The \B meson is spinless, and thus by conservation of spin, the \Ome and \Kstz mesons must have zero or opposite spin projections along the decay axis. This constraint gives rise to three possible polarization states. Their amplitudes are labeled as \Alambda ($\lambda = 0,\pm1$) in the helicity basis. The amplitude \Az corresponds to both \Ome and \Kstz having \szZ along the decay axis; this amplitude is referred to as ``longitudinal" polarization. The \Apm amplitudes correspond to both \Ome and \Kstz having \szPM along their direction of motion; these amplitudes are referred to as ``transverse" polarization. A simple diagram depicting these three states is shown in Fig. \ref{helicity_amplitudes}. 
%%%%%%%%%%%%%%%%%%%%%%%%%%%%%%%%%%%%%%%%%%%%
%%%%%%%%%%%%%%%%%%   SPIN DIAGRAM   %%%%%%%%%%%%%%%%
%%%%%%%%%%%%%%%%%%%%%%%%%%%%%%%%%%%%%%%%%%%%
\begin{figure}[t]
\centerline{
\hbox{
\includegraphics[width=0.60\columnwidth,height=!]{thesis_diagram_1_B0-OmeKstz.pdf}\hspace{0.2cm} }}
\caption{Possible spin configurations for \BzToOmeKstz decays, shown in the rest frame of the \Bz. The dotted line corresponds to the decay axis (taken to be in the $z$ direction), with arrows indicating the direction of motion of the vector mesons. The thick solid arrow denotes the spin projection. From top to bottom, the \Az, \Ap and \Am states. \label{helicity_amplitudes}}
\vskip -0.2cm
\end{figure}

The helicity frame for \BzToOmeKstz decays is show in Fig. \ref{helicity_diagram}. The unit vector along the direction of the \Ome is \bfv. The decay plane of the \Ome is the plane of the three-pions in the \Ome rest frame (the ellipse), with normal \pOne. The angle between \pOne and the flight direction of the \Ome measured in the \Ome rest frame is \ThetaOme. The flight direction of the \kp measured in the \Kstz rest frame is denoted by \pTwo, and the angle between \pTwo and the flight direction of the \Kstz is \ThetaKstz. The vector along the projection of \pOne (\pTwo) orthogonal to the direction of \Ome (\Kstz) is denoted by \bfc (\bfd). The azimuthal angle $\phi$ is defined as the angle between \bfc and \bfd. Thus, we can define the three angles by
\begin{eqnarray}
\CosOmeDef \nonumber \\
\CosKstzDef \nonumber \\
\CosPhiDef \nonumber \\
\SinPhiDef
\end{eqnarray}
%%%%%%%%%%%%%%%%%%%%%%%%%%%%%%%%%%%%%%%%%%%%
%%%%%%%%%%%%%%%%%   HELICITY DIAGRAM   %%%%%%%%%%%%%%%
%%%%%%%%%%%%%%%%%%%%%%%%%%%%%%%%%%%%%%%%%%%%
\begin{figure}[t]
\centerline{
\hbox{
\includegraphics[width=0.90\columnwidth,height=!]{helicity_diagram.pdf}\hspace{0.2cm} }}
\caption{Helicity frame for \BzToOmeKstz. The variables are defined in the text. \label{helicity_diagram} }
%\caption{Helicity frame for \BzToOmeKstz. The decay plane of the \Ome is the plane of the 3 pions in the \Ome rest frame (the ellipse), with normal \pOne. The angle between \pOne and the flight direction of the \Ome measured in the \Ome rest frame is \ThOne. The flight direction of the \kp measured in the \Kstz rest frame is denoted by \pTwo. The vector along the projection of \pOne (\pTwo) orthogonal to the direction of \Ome (\Kstz) is denoted by \bfc (\bfd). The azimuthal angle $\phi$ is defined as the angle between \bfc and \bfd. \label{helicity_diagram} }
\vskip -0.2cm
\end{figure}
%%%%%%%%%%%%%%%%%%%%%%%%%%%%%%%%%%%%%%%%%%%%
%%%%%%%%%%%%%%%%%%%%%%%%%%%%%%%%%%%%%%%%%%%%
%%%%%%%%%%%%%%%%%%%%%%%%%%%%%%%%%%%%%%%%%%%%
In general, the angular dependence for \BvOvT decays, where the vector mesons decay to spinless particles, can be expressed in terms of the spherical harmonics~\cite{BBk} as
\begin{equation}\label{GenHelDecay}
\GenHelDecay.
\end{equation}
Making the substitutions \ThOneToThOme and \ThTwoToThKstz, and expanding the right hand side of Eq. \ref{GenHelDecay}, we obtain the full angular distribution of \BzToOmeKstz decays in the helicity basis: 
\begin{equation}   \label{FullHelDef}
\HelDefLineOneOmeKstz   %\nonumber 
\end{equation}
\begin{equation} 
\HelDefLineTwoOmeKstz  \nonumber \\
\end{equation}
\begin{equation} 
\HelDefLineThreeOmeKstz  \nonumber  \\
\end{equation}
\begin{equation} 
\HelDefLineFourOmeKstz.  \nonumber
\end{equation}
\begin{comment}
\begin{eqnarray}\label{FullHelDef}
\lefteqn{\HelDefLineOneOmeKstz}   \\
& \HelDefLineTwoOmeKstz  \nonumber \\
& \HelDefLineThreeOmeKstz  \nonumber  \\
& \HelDefLineFourOmeKstz  \nonumber
\end{eqnarray}
\end{comment}
To perform a full angular analylsis in \Bvv decays, very high statistics are required. Based on the experimental U.L.~\cite{babar_wKstz_2} and our MC efficiency for \BzToOmeKstz decays (Sec.~\ref{sigmc decay}), the \NBBbarVal \BBbar pair data set currently available will not yield sufficient events to perform a full angular analysis (Sec.~\ref{gsim mc test}). Nevertheless, it will be possible to measure the fraction of longitudinal polarization \fL, which has it's own importance. To do so, we integrate Eq. \ref{FullHelDef} over the angle $\phi$ (assuming the azimuthal acceptance is uniform) and obtain 
%To perform a full angular analylsis in \Bvv decays, very high statistics are required. As previously mentioned, theoretical calculations for the branching fraction of \BzToOmeKstz decays are only of the order \OrderBr. The experimental U.L. Thus, the \NBBbarVal \BBbar pair data set currently available will not yield sufficient events to perform a full angular analysis. Nevertheless, it will be possible to measure the fraction of longitudinal polarization \fL, which has it's own importance. To do so, we integrate Eq. \ref{FullHelDef} over the angle $\phi$ (assuming the azimuthal acceptance is uniform) and obtain 
\begin{equation}\label{HelDefFinal}
\HelDefFinalOmeKstz,
\end{equation}
where we have defined \fL as%the fraction of longitudinal polarization \fL as 
\begin{equation}
\fLDef.
\end{equation}
From Eq. \ref{HelDefFinal}, it is clear that \fL can be extracted from a one-parameter fit to the $\CosOme - \CosKstz$ distribution. 

\subsection{Transversity Basis\label{Transversity Basis}}
In the transversity basis, \Az remains unchanged, while the transverse amplitudes are defined as spin projections for one vector meson parallel (\Apar) and perpendicular (\Aperp) to the decay plane of the other. In terms of the helicity amplitues \Ap and \Am, they are defined as~\cite{transverse_amplitudes}:
\begin{equation}
\AparDef~~~~~~~~~~~~~\AperpDef,
\end{equation}
where each amplitude contributes to only one \textit{CP} eigenstate (\Apar: \textit{CP}-even, \Aperp: \textit{CP}-odd). %Thus, it is convenient to use the transversity basis for time-dependent analysis. 

\subsection{Partial Wave Decomposition\label{Partial Wave Decomposition}}
The partial wave decomposition separates out the possible $S$ ($L=0$), $P$ ($L=1$) and $D$ ($L=2$) orbital angular momenta between the vector mesons. The $S$, $P$ and $D$ amplitudes can be expressed in terms of the the transversity amplitudes as~\cite{partial_wave_amplitudes}
\begin{equation}
\Sdef,~~~~~~~\Pdef,~~~~~~~\Ddef.
\end{equation}
We refer the reader to, e.g.~\cite{hel_MC_dist, BBk, transverse_amplitudes, partial_wave_amplitudes}, for detailed derivations and description of the transversity basis and partial wave decomposition formalisms. 



\section{Expected Hierarchy of Decay Amplitudes}
%The SM with factorization [REF FA] leads to the expectation that for \Bvv decays (i.e. \bqbar quark decay), there is a hierarchy of decay amplitudes
The FA assumption leads to the expectation that for \Bvv decays (i.e. \bqbar quark decay), there is a hierarchy of decay amplitudes
\begin{equation}\label{HelAmpHierarchy}
\HelAmpHierarchy,
\end{equation}
where \mV and \mB are the vector and \B meson masses, respectively (for \Bbar decay, the order of \Ap and \Am is reversed). In terms of the polarization fractions, we write
\begin{equation}\label{HelAmpHierarchy2}
\NaiveFactfL~~~~~~~~~~~~~~~~\NaiveFactfT, 
\end{equation}
where \fPar and \fPerp are the parallel and perpendicular polarization fractions, respectively. This follows from the \VmA structure of the weak interactions in the SM, which dictates that the $W$ bosons only couple to left-handed quarks (\qL) and right-handed anti-quarks (\qbarR). Thus the leading operator in the \BzToOmeKstz channel generates decays of the form \bbarTosbarRdLdbarR. This configuration is clearly manifest in the \Az state: the \sqbar and $d$ (\dqbar and $d$) quark constituents of the \Kstz (\Ome) meson are right- and left-handed, respectively. To achieve the \Ap configuration, the collinear $d$ quark (produced from the gluon) must undergo a `helicity-flip' to make the \Kstz positively polarized.  The spectator $d$ quark does not undergo a weak interaction, and thus it can be left- or right-handed; for \Ap it is right-handed. In the \Am state, both the \sqbar quark from the \Kstz and the \dqbar from the \Ome must undergo helicity flips for the mesons to be negatively polarized; here the $d$ quark in the \Ome meson is left-handed. For finite \mB, each helicity flip reduces the amplitude by a factor of $1/\mB$; this results in the hierarchy (\ref{HelAmpHierarchy}). This is summarized in the picture depicted in Fig. \ref{helicity_flip_diagram}.

%%%%%%%%%%%%%%%%%%%%%%%%%%%%%%%%%%%%%%%%%%%%
%%%%%%%%%%%%%%%%   HELICITY FLIP DIAGRAM   %%%%%%%%%%%%%%
%%%%%%%%%%%%%%%%%%%%%%%%%%%%%%%%%%%%%%%%%%%%
\begin{figure}[h]
\centerline{
\hbox{
\includegraphics[width=0.60\columnwidth,height=!]{helicity_flip_diagram.pdf}\hspace{0.2cm} }}
\caption{Possible spin configurations for the quarks of the \Ome and \Kstz mesons in \BzToOmeKstz decays~\cite{gritsans_talk}. The thin arrows denote the direction of motion, while the thick arrows denote the spin projection of the quarks. The thick solid arrows indicate that a `helicity-flip' has occurred. The ``$R$" and ``$L$" subscripts correspond to the left- and right-handedness of the quarks.\label{helicity_flip_diagram}}
\vskip -0.2cm
\end{figure}



%%%%%%%%%%%%%%%%%%%%%%%%%%%%%%%%%%%%%%%%%%%%
%%%%%%%%%%%%%%%%  TREE LEVEL BVV DECAYS  %%%%%%%%%%%%%
%%%%%%%%%%%%%%%%%%%%%%%%%%%%%%%%%%%%%%%%%%%%
\section{Tree Level \bm{\Bvv} Decays \label{Tree level Bvv decays}}
The measured \calB and \fL for tree level \Bvv decays, where $VV$ = \RhoRho, \OmeRho, \Ome\Ome and \PhiPhi, is summarized in Table \ref{tree B and fL summary table}. The \RhopRhom, \RhopRhoz and \OmeRhop channels have been observed with large \calB and \fL, which is consistent with the hierarchy (\ref{HelAmpHierarchy}). In \RhozRhoz decays, the tree contribution is SM suppressed, thereby making it more sensitive to the penguin amplitude. As a result, the \calB for \RhozRhoz decays is much smaller than that of the \RhopRhom and \RhopRhoz channels. With a data set of \BzToRhozRhozBabarDataSet \BBbar pairs, BaBar has evidence for \BzToRhozRhoz decays with 3.1\sig significance, while Belle, using \NBBbarVal \BBbar pairs, sets a 90\% C.L. upper limit on the branching fraction, where the signal yield has a significance of 1.0\sig. The analysis methods between BaBar and Belle differ significantly, e.g., in the invariant mass range used to fit the $\rho$ mesons for the signal extraction; BaBar performs a fit to a tight mass window of \RhozRhozWindowBabar, while Belle fits the $\rho$ mesons with a wide window of \RhozRhozWindowBelle. As a result, the fit used in the Belle analysis is sensitive to non-resonant decays, which are measured to be \BzToRhozPipPimBelleBr and \BzToFourPiBelleBr, where the signal yields have significances of \BzToRhozPipPimBelleSig and \BzToFourPiBelleSig, respectively. For Belle's \calB measurement, \fL is assumed to be 1 to obtain the most conservative U.L. as the efficiency for $\fL=1$ is smaller than that for $\fL=0$. We will return to a discussion of the invariant mass windows used in \Bvv decays by Belle and BaBar in Chap.~\ref{summary}.

For the \OmeRhoz, \OmeOme and \PhiPhi channels, no significant signals are observed and thus we only quote the 90\% C.L. upper limits. With increased data, it will be possible to measure \calB and \fL for these channels as well.

%Babar's most recent measurement of \BzToRhozRhozBabarfLEqn is significantly lower than what is observed in the \RhopRhom and \RhopRhoz channels. However, we note that in their previous publication~\cite{babar_rho0rho0_new}, which used the slightly smaller data set of \BzToRhozRhozBabarDataSetOld \BBbar pairs, the value measured for \fL was \BzToRhozRhozBabarfLOld, where the \calB had a significance of \BzToRhozRhozBabarSigOld. 

%The two values are within their errors, so they are not inconsistent. 

%More data is needed to obtain a more accurate value for \fL for \BzToRhozRhoz.
%The hierarchy (\ref{HelAmpHierarchy}) is realized in tree level \Bvv decays. 
%The longitudinal polarization in --- satisfies \NaiveFact in ... , as in naive factorization

%\BpToOmeRhop compliments \BzToRhopRhom and \BpToRhopRhoz, confirms large \calB and \fL [grit 2004, pg12]. Large decay rate with ``tree" (compared to pipi) [grit 2004, pg10].

%Both \BzToRhozRhoz and \BzToOmeRhoz set tight constraints on ``penguin pollution" [grit 2004, pg 11].

\begin{table}[h]
\small
\begin{center}
\caption{Branching fractions \calB and longitudinal polarization fractions \fL for tree-level \Bvv decay modes. For measurements with less than $3.0\sig$ significance, only the 90\% confidence level upper limits are shown. For \calB and \fL, the first (second) error is statistical (systematic). These results are reported in Refs.~\cite{babar_rho0rho0,babar_rhoprho0,babar_rhoprhom,belle_rho0rho0,belle_rhoprho0,belle_rhoprhom,babar_KstKst,babar_wKstz_2}. \label{tree B and fL summary table}}
\vspace{0.5cm}
\begin{tabular*}{1.0\textwidth}{@{\extracolsep{\fill}}lcccc}
\hline
Mode & \multicolumn{2}{ c }{\calB ($10^{-6})$}  & \multicolumn{2}{ c }{\fL } \\
                          & Belle & \BABAR & Belle & \BABAR  \\ \hline \hline
\RhopRhom  &   \BzToRhopRhomBelleBr    &   \BzToRhopRhomBabarBr   &   \BzToRhopRhomBellefL   &   \BzToRhopRhomBabarfL      \\ \hline  
\RhopRhoz   &    \BpToRhopRhozBelleBr    &   \BpToRhopRhozBabarBr   &   \BpToRhopRhozBellefL   &    \BpToRhopRhozBabarfL     \\ \hline  
\RhozRhoz   &    \BzToRhozRhozBelleBr    &   \BzToRhozRhozBabarBr   &   $-$   &    \BzToRhozRhozBabarfL     \\ \hline  
\OmeRhop     &   $-$    &   \BpToOmeRhopBr   &   $-$   &  \BpToOmeRhopfL       \\ \hline  
\OmeRhoz     &   $-$    &   \BzToOmeRhozBr   &  $-$    &  $-$       \\ \hline  
\OmeOme      &   $-$    &   \BzToOmeOmeBr   &   $-$   &  $-$       \\ \hline  
\PhiPhi           &   $-$    &  \BzToPhiPhiBabarBr    &   $-$   &  $-$       \\ \hline  
\end{tabular*}
\end{center}
\end{table}


%%%%%%%%%%%%%%%%%%%%%%%%%%%%%%%%%%%%%%%%%%%%
%%%%%%%%%%%%%%  PENGUIN LEVEL BVV DECAYS  %%%%%%%%%%%%%
%%%%%%%%%%%%%%%%%%%%%%%%%%%%%%%%%%%%%%%%%%%%
\section{Penguin Level \bm{\Bvv} Decays\label{Penguin level Bvv decays}}
The measured \calB and \fL for penguin level \Bvv decays, where $VV$ = \PhiKst, \RhoKst, \OmeKst, \KstzKstzbar, \KstzKstz and {\OmePhi, is summarized in Table \ref{penguin B and fL summary table}. For the \PhiKst and \RhoKstz channels, the measured $\fL\sim0.5$. This is in strong contrast to the expected value based on FA (Eqns.~\ref{HelAmpHierarchy} and~\ref{HelAmpHierarchy2}) and the results of the other $VV$ modes, and has been considered as a puzzle. 

The low \fL observed in the \PhiKst and \RhoKstz channels implies that non-factorizable contributions to the decay amplitude play a significant role. Thus we must go beyond the predictions of FA. QCD factorization (QCDF)~\cite{QCDF1,QCDF2,QCDF3}, which supersedes FA, provides the means to compute two-body decay amplitudes from fist principles. However, the accuracy of QCDF is limited by power corrections to the heavy-quark limit and the uncertainties of theoretical inputs, e.g., quark masses, form factors, and light-cone distribution amplitudes~\cite{QCDF2}. %Recently, using QCDF and input from other the experimental results of the \bsqq penguins rhokst, phikst, the authors of --- and --- have predicted that....

Additional attempts to explain this enhancement of the transverse amplitudes within the SM include: QCD penguin annihilation (penguin-induced annihilation)~\cite{Kagan QCD peng annih}; the transverse gluon emitted by the \bsg transition (the magnetic penguin)~\cite{magnetic penguin}; the charming penguin in soft-collinear effective theory (SCET)~\cite{charming penguins}; the rescattering effect through final state interactions (FSI)~\cite{Ladisa etal FSI, Cheng Chua Soni FSI, Colangelo FSI}; new sets of form factors~\cite{new sets of form factors}; and the effect of an enhanced electroweak penguin amplitude~\cite{Beneke etal Enhanced Electroweak Penguin Amplitudes}. 

There have also been several mechanisms proposed which involve new physics, e.g., the presence of new right-handed currents~\cite{Grossman Beyond the SM with B and K physics, RH currents}, a new type of scalar interaction with a simple Higgs model associated with a tree level flavor changing neutral current (FCNC)~\cite{scalar interactions}, and new-physics operators of the form \bsqq ($q = u,d$)~\cite{NP operators}. However, it has been argued~\cite{Li and Mishima Polarization in Bvv decays} that several of these explanations involve many free parameters, and cannot account for the polarization measurements of all \Bvv modes simultaneously. Thus, measurements of the rates and polarizations of the remaining penguin-dominated \Bvv are necessary to resolve this ambiguity. %Measuring \fL in \BzToOmeKstz will provide an important piece of the puzzle.

%Thus, measurements of the rates and polarizations of the remaining \bsqq penguin-dominated \Bvv decays are necessary to resolve the ambiguity observed in the \PhiKst and \RhoKstz channels. 

%[DISCUSS WKSTZ IN DETAIL AND LIST THE PREDICTIONS FOR FL. PERHAPS TALK ABOUT THE INDIVIDUAL MODELS AND THE VALUES THEY PREDICT, IF NOT TOO COMPLICATED. Say that xx model predicts low fL and this one predicts high. Emphasize that the experimental value is crucial since the varying theories disagree on their predictions.]


%To resolve the ambiguity... it is crucial to measure the rates and polarization fractions for the remaining \bsqq penguin-dominated \Bvv decays.


%Several attempts to explain the enhancement of the transverse amplitudes have been proposed both within [REF], and beyond [REF] the SM. They include: 
%For the high statistics \PhiKst channels, a full angular analysis has been performed by both Belle and BaBar. 
%\BpToRhopKstz is ``pure" penguin like \BToPhiKst [grit 2004, pg14].

%The charmless decay \BzToKstzKstzbar proceeds through both electroweak and gluonic \bd penguin loops. The decay \BzToKstzKstz is SM suppressed and could appear via an intermediate heavy boson~\cite{babar_KstKst}.

%Further information about the effects of penguin-dominated decays can come from branching fraction and polarization measurements in decays such as \BToOmeKst and \BzToOmePhi which are conjugate to \BToPhiKst via an SU(3) rotation~\cite{babar_wKstz_2,SU3}

\begin{table}[t]
\small
\begin{center}
\caption{\label{penguin B and fL summary table}Branching fractions \calB and longitudinal polarization fractions \fL for penguin-level \Bvv decay modes. For measurements with less than $3.0\sig$ significance, only the 90\% confidence level upper limits are shown. For \calB and \fL, the first (second) error is statistical (systematic). These results are reported in Refs.~\cite{belle_phiKstz,babar_phiKstz,belle_RhoPKstz,babar_RhoPKstz,babar_KstKst,babar_wKstz_2}.}
\vspace{0.5cm}
\begin{tabular*}{1.0\textwidth}{@{\extracolsep{\fill}}lcccc}
\hline
Mode & \multicolumn{2}{ c }{\calB ($10^{-6})$}  & \multicolumn{2}{ c }{\fL } \\
                          & Belle & \BABAR & Belle & \BABAR  \\ \hline \hline
\PhiKstz         &   \BzToPhiKstzBelleBr    &    \BzToPhiKstzBabarBr  &   \BzToPhiKstzBellefL   &   \BzToPhiKstzBabarfL      \\ \hline  
\PhiKstp         &   \BpToPhiKstpBelleBr    &   \BzToPhiKstpBabarBr   &   \BpToPhiKstpBellefL   &    \BzToPhiKstpBabarfL     \\ \hline  
\RhopKstz     &   \BpToRhopKstzBelleBr    &   \BpToRhopKstzBabarBr   &   \BpToRhopKstzBellefL   &     \BpToRhopKstzBabarfL    \\ \hline  
\RhozKstz      &    $-$    &   \BzToRhozKstzBabarBr   &   $-$    &    \BzToRhozKstzBabarfL     \\ \hline  
\RhozKstp     &   $-$    &  \BpToRhozKstpBabarBr    &   $-$   &   $-$      \\ \hline  
\RhomKstp    &   $-$    &   \BzToRhomKstpBabarBr   &  $-$    &     $-$    \\ \hline  
\OmeKstz       &    $-$     &   \BzToOmeKstzBabarBr    &    $-$    &    $-$     \\ \hline  
\OmeKstp      &   $-$    &   \BpToOmeKstpBr               &   $-$   &   $-$      \\ \hline  
\KstzKstzbar         &   $-$    &    \BzToKstzKstzbarBabarBr     &  $-$   &    \BzToKstzKstzbarBabarfL     \\ \hline  
\KstzKstz        &   $-$    &    \BzToKstzKstzBabarBr     &  $-$   &   $-$      \\ \hline  
\OmePhi         &   $-$    &   \BzToOmePhiBr      &  $-$   &  $-$       \\ \hline  
\end{tabular*}
\end{center}
\end{table}




%where \LambdaVal is the strong interaction scale and ... (for \Bbar decay, the order of \Ap and \Am are reversed). This hierarchy follows from the \VmA structure of the weak interactions in the SM, which dictates that the $W$ bosons only couple to left-handed quarks and right-handed anti-quarks. This is clearly manifest in the \Az configuration: the \sqbar and $d$ quark constituents of the \Kstz meson are right- and left-handed, respectively; the \dqbar of the \Ome meson is right-handed, while the spectator $d$ quark (also from the \Ome) can be left- or right-handed since it does not undergo a weak interaction. To achieve the \Ap configuration, the collinear $d$ quark (produced from the gluon) must undergo a `helicity-flip' in the positively polarized \Kstz. The spectator $d$ quark does not undergo a weak interaction, and thus it can be left- or right-handed.


\begin{comment}
\subsection{Predictions for \bm{\BzToOmeKstz}\label{Predictions for wKstz}}
%The \BzToOmeKstz decay is dominated by a \bsdd penguin transition
The \bsdd penguin dominated \BzToOmeKstz decay proceeds via a similar mechanism to the \BToPhiKst and \BToRhoKstz decays. Branching fraction and polarization measurements will further our understanding of the decay dynamics of \bsqq penguin transitions and help resolve the competing explanations for the anomalous {\fL $\sim$ \fT measured in the  \BToPhiKst and \BToRhoKstz  decays.

Theoretical predictions for the \calB and \fL of \BzToOmeKstz decays vary depending on the mechanism employed. 

% Using a theoretical framework based on the next-to-leading-order QCD-improved 

 [REF]. ---- Recently, H.-Y. Cheng and K.-C. Yang [REF] predicted ---- and --- .
\end{comment}


\begin{comment}

\pagebreak

\subsection{Explaining \bm{\fL $\sim$ \fT} in \bm{\BToPhiKst} and \bm{\BToRhoKstz} \label{Explaining low fL}}

Learn the difference between: naive fact., QCD fact., and pQCD.

The SM with factorization predicts that 

Find ref[6] in PRL96, 141801 and quote it when i say naive factorization for the first time.


POSSIBLE EXPLANATIONS: 
\begin{itemize}

\item{}
SM:  pg162 in kagan (QCD penguin annihilation graphs)~\cite{Kagan QCD peng annih}

\item{}
SM:  FSI~\cite{Ladisa etal FSI, Cheng Chua Soni FSI, Colangelo FSI}

\item{}
SM:  charming penguins~\cite{charming penguins}

\item{}
SM:  magentic penguin~\cite{magnetic penguin}.

\item{}
SM:  Enhanced electroweak penguin amplitudes~\cite{Beneke etal Enhanced Electroweak Penguin Amplitudes}

\item{} 
SM:  New sets of form factors~\cite{new sets of form factors}

\item{}
NP:  New types of scalar interactions~\cite{scalar interactions}.

\item{}
NP:  NP effects arising from right-handed currents~\cite{RH currents}.

\item{}
NP: NP operators which contribute to the polarization states of rhokst~\cite{NP operators}. There's a nice summary of annih(kagan), FSI, and the magnetic penguin explanations at the end of pg2.

\item{}
New physics. Perhaps I should quote the end of Kagan's paper too when he talks about new-right hand currents. pg7 of grossmans~\cite{Kagan QCD peng annih, Grossman Beyond the SM with B and K physics}. Look through babar's list of NP papers in their wkst paper. 

\item{}
Not sure about these yet~\cite{Beneke etal Enhanced Electroweak Penguin Amplitudes, Li and Mishima Polarization in Bvv decays, Cheng Yang BR and Polarization in Bvv decays}.

see pg2 lhs in~\cite{Li and Mishima Polarization in Bvv decays} PRD71, 054025. Says that PQCD predicts that \fL only goes down to 0.75. Find a good ref on PQCD and cite it (or just cite this one).

\end{itemize}

\end{comment}

\begin{comment}
For two-body decays \pxx, ($X$ = $P$ (psuedoscalar), $V$ (vector), etc.) where both \xOne and \xTwo decay into spinless particles, the angular distribution can be expressed in terms of the spherical harmonics [----] as
\begin{equation}
\GenHelDecay,
\end{equation}
where \AsubM is the decay amplitude, \JsubI is the angular momentum quantum number for \XsubI, \ThsubI is the helicity angle for \XsubI (defined by the direction of the decay products of the \XsubI in the \XsubI rest frame), and \PhiDef is the azimuthal angle between the two decay planes.
%where \AsubM is the decay amplitude, \JsubI and \ThsubI are the angular momentum quantum number and helicity angle, respectively, for \XsubI, and \PhiDef is the azimuthal angle between the two decay planes, as defined below.



%Let us consider the \BzToOmeKstz decay in the rest frame of the \Bz, where the axis of spin quantization is taken as the $z$ direction. We denote the spin state of the \Bz as \JM, and the spin states of the \Ome and \Kstz mesons as \JM and \JM, respectively. The \B meson, composed of two spin-1/2 quarks with total spin J=0, occupies the spin singlet state:

%Let us consider the \BzToOmeKstz decay in the rest frame of the \Bz, where the axis of spin quantization is taken as the $z$ direction. The \B meson is a psuedoscalar (spinless) particle composed of two spin-1/2 quarks. Thus, it occupies the spin singlet state:
%\begin{equation}
%\Singlet
%\end{equation}
%The \Ome and \Kstz mesons have spin-1, thus they can occupy 

Charmless two-body \B meson decays into final states involving two vector mesons ($VV$)

\B meson decays into final states involving two vector mesons ($VV$) 

This section is organized as follow: a brief introduction to the helicity formalism

\B meson decays to two particles with non-zero spin 

Let us denote the spin state of the \B meson as |JM>

For \B meson decay to two vector particles, the 

\B mesons occupy the spin state (JM) 

Let us consider the two-body decay \Bvv in the rest frame of the parent particle, where ---- and --- denote the spin state of the parent particle  and the quantization axis is taken as the z direction

Let us consider the \BzToOmeKstz decay in the rest frame of the \Bz, where the axis of spin quantization is taken as the \z direction.

where \JM, \JM and \JM denote the spin states of the \Bz, \Ome and \Kstz mesons respectively

We consider two-body decays 

In this note, we consider two-body decays 

Let JM denote the spin state of the 

The \B meson is 


For \B meson decays to two vector mesons, \JOne = \JTwo = 1. Since the \B-meson is spinless, there are three possible spin configurations for the vector mesons: 

conservation of angular momentum dictates that the vector particles ..... define lambda in terms of m here. (maybe rewrite m as ms) (maybe show a diagram like in gritsan's talks with the arrows for spin and particle direction. see if i can do it in keynote.)
Thus, in terms of helicities lambda, we can re-write eqn (1) as 

----------show the eqn from gritsan's slides only------------
\end{comment}











\begin{comment}
%%%%%%%%%%%%%%%%%%%%%%%%%%%%%%%%%%%%%%%%%%%%
%%%%%%%%%%%%%%%%%   FEYNMAN DIAGRAM   %%%%%%%%%%%%%%%
%%%%%%%%%%%%%%%%%%%%%%%%%%%%%%%%%%%%%%%%%%%%
\begin{figure}[b]
\small
\centering
\vspace{7mm}
\subfigure{ \input{diagram.tex}} \\
\vspace{1mm}
\caption{\label{feynman_diagram} Penguin diagram for \BzToOmeKstz decays.}
\end{figure}
 \end{comment}