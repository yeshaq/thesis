\clearpage
\section{Background estimation for processes with genuine \texorpdfstring{\met}{MET}\label{sec:backgrounds}}

Once all the signal selection requirements have been imposed, the
contribution from QCD multijet events is expected to be negligible, as
demonstrated in Section~\ref{sec:qcd}.

The remaining backgrounds in the hadronic signal region stem from SM
processes with genuine \met in the final state.  In the case of events
where no b-quark jets are identified, the largest backgrounds with
genuine \met arise from the production of W and Z bosons in
association with jets. The weak decay \znunu\ is the only significant
contribution from Z + jets events. For W + jets events, the two
dominant sources are leptonic W decays in which the lepton is not
reconstructed or fails the isolation or acceptance requirements, and
the weak decay $\wtaunu$ where the $\tau$ decays hadronically and is
identified as a jet. Contributions from SM processes such as
single-top, Drell-Yan, and diboson production are also expected. For
events with one or more reconstructed b-quark jets, \ttbar production
followed by semi-leptonic weak decays becomes the most important
single background source. For events with only one reconstructed
b-quark jet, the contribution of both W + jets and Z + jets
backgrounds are of a similar size to the \ttbar background.  For
events with two reconstructed b-quark jets, \ttbar production
dominates, while events with three or more reconstructed b-quark jets
originate almost exclusively from \ttbar events, in which at least one
jet is misidentified as originating from a bottom quark.

In order to estimate the contributions from each of these backgrounds,
three data control samples are used, which are binned identically to
the signal region. Two independent estimates of the irreducible
background of \znunu\ + jets events are determined from the data
control samples comprising $Z\rightarrow\mu\mu$ + jets and $\gamma$ +
jets events, which have similar kinematic properties but different
acceptances. The $Z\rightarrow\mu\mu$ + jets events have similar
kinematic properties when the two muons are ignored, but a smaller
branching fraction, while the $\gamma$ + jets events have similar
kinematic properties when the photon is
ignored~\cite{PAS-SUS-08-002,Bern:2011pa}, but a larger production
cross section. A $\mu$ + jets data sample provides an estimate for all
other SM backgrounds, which is dominated by \ttbar and W production
leading to W + jets final states.

As described previously, the event selection criteria for the control
samples are defined to ensure that any potential contamination from
multijet events is negligible. Further, the control sample selection
criteria also suppress contributions from a wide variety of SUSY
models, including those considered in this analysis. Any potential
signal contamination in the data control samples is accounted for in
the fitting procedure described in Section~\ref{sec:results}.

\subsection{Overview of the method\label{sec:background-method}}

The method used to estimate the aforementioned SM background
contributions in the hadronic signal region relies on the use of a
{\it transfer factor} (TF) determined from MC samples to transform the
observed yield in a given \scalht, jet (\njet) and b-tag (\nb)
multiplicity bin of a control sample, $\nobs^{\rm
  control}(\scalht,\njet,\nb)$, into a predicted yield for the
corresponding bin of the hadronic signal region, $\npre^{\rm
  signal}(\scalht,\njet,\nb)$. The choice of \njet and \nb event
categorisation and \scalht binning in the control samples is identical
to that for the signal region, as defined in Table~\ref{tab:ht-bins}
in Section~\ref{sec:selection}. 

Each transfer factor is simply a ratio of the yields obtained from MC
simulation for the same bin of the signal region and a given control
sample:

\begin{equation}
  \label{equ:tf-ratio}
  {\rm TF} = \frac{N_{\rm MC}^{\rm signal}(\scalht,\njet,\nb)}{N_{\rm
      MC}^{\rm control}(\scalht,\njet,\nb)} 
\end{equation}

In this way, ``na\"ive'' predictions for the total SM background can
be made by considering separately the sum of the predictions from the
\mj and \gj samples or the \mj and \mmj samples:

\begin{equation}
  \label{equ:pred-method}
  \npre^{\rm signal}(\scalht,\njet,\nb) = \frac{N_{\rm MC}^{\rm
      signal}(\scalht,\njet,\nb)}{N_{\rm MC}^{\rm
      control}(\scalht,\njet,\nb)} \times \nobs^{\rm
    control}(\scalht,\njet,\nb)   
\end{equation}

When constructing the transfer factors, the MC expectations for the
following SM processes are considered: W + jets ($N_{\rm W}$), \ttbar
+ jets ($N_{\ttbar}$), \znunu\ + jets ($N_{\znunu}$), DY + jets
($N_{\mathrm DY}$), \gj ($N_\gamma$), single top + jets
production via the s, t, and tW-channels ($N_{\rm top}$), and WW +
jets, WZ + jets, and ZZ + jets ($N_{\rm di-boson}$). Details on the MC
samples used are given in Sec.~\ref{sec:mc-samples}. All MC samples
are normalised to the intergrated luminosity of the appopriate data
sample.

While ``na\"ive'' predictions can be made using
Equ.~\ref{equ:pred-method}, the fitting prcedure that provides the
final result is defined formally by the likelihood model described in
Sec.~\ref{sec:statistics}. In summary, the observation in each bin
(defined in terms of the variables \njet, \nb, and \scalht) of the
signal sample is modelled as Poisson-distributed about the sum of a SM
expectation (and a potential signal contribution). The components of
this SM expectation are related to the expected yields in the control
samples via transfer factors derived from simulation, as described
in Sec.~\ref{sec:background-method}. The observations in each bin
(again defined by \njet, \nb, and \scalht) of the control samples are
similarly modelled as Poisson-distributed about the expectated yields
for each control sample. In this way, for a given bin, the observed
yields in the signal and control samples are all connected via the
transfer factors derived from simulation.

The sum of expected yields from all MC samples, obtained for the
relevant control sample selection, enter the denominator of each
transfer factor:

\begin{equation}
  \label{equ:ratio-denom}
  N_{\rm MC}^{\rm control}(\scalht,\njet,\nb) = N_{\rm W} + N_{\ttbar} + N_{\znunu} +
N_{\rm DY} + N_{\gamma} + N_{\rm top} + N_{\rm di-boson}
\end{equation}

For the b jet multiplicity bins satisfying $n_b \leq 1$, the \mj
control sample is used to predict primarily the W + jets and \ttbar +
jets backgrounds, but all remaining residual SM backgrounds except
\znunu\ + jets events are also considered. (The \znunu\ + jets
background is instead accounted for through the \mmj and \gj samples.)
The sum of expected yields from all MC samples except \znunu\ + jets,
obtained for the hadronic signal region selection, enter the numerator
of each transfer factor:

\begin{equation}
  \label{equ:ratio-numer-mj}
  N_{\rm MC}^{\rm signal}(\scalht,\njet,\nb \leq 1) = N_{\rm W} +
  N_{\ttbar} + N_{\rm DY} + N_{\rm top} + N_{\rm di-boson}
\end{equation}

For the same b jet multiplicity bins ($0 \leq \nb \leq 1$), the \mmj
and \gj control samples are used to predict the \znunu\ + jets process
only, and the expected yields in the bins of the signal region as
obtained from the \znunu\ sample enter the numerator of each transfer
factor:

\begin{equation}
  \label{equ:ratio-numer-mmj}
  N_{\rm MC}^{\rm signal}(\scalht,\njet,0 \leq \nb \leq 1) = N_{\znunu}
\end{equation}

For the b jet multiplicity bins satisfying $n_b \geq 2$, the \mj
control sample is again used to predict primarily the W + jets and
\ttbar + jets backgrounds, but also to predict all other remaining
residual SM backgrounds {\it including} \znunu\ + jets events. The sum
of expected yields from all MC samples {\it including} \znunu\ + jets,
obtained for the hadronic signal region selection, enter the numerator
of each transfer factor:

\begin{equation}
  \label{equ:ratio-numer-mj}
  N_{\rm MC}^{\rm signal}(\scalht,\njet,\nb \geq 2) = N_{\rm W} +
  N_{\ttbar} + N_{\rm DY} + N_{\rm top} + N_{\rm di-boson} + N_{\znunu}
\end{equation}

In this case, the \mmj and \gj control samples are not used, as the
yields in these two data control samples are expected to be negligible
due to the requirement of at least two b jets per event. The method of
using a W + jets sample to predict the \znunu\ + jets background has
been used previously~\cite{RA1Paper, RA1Paper2011}, and this approach
is addressed by a set of closure tests described in
Sec.~\ref{sec:bkgd-syst}, in which a \mj sample (rich in W + jets and
\ttbar events) is used to make predictions of yields in a \mmj sample
(rich in Z$\rightarrow\mu\mu$ + jets events). The data control samples
used to predict the SM backgrounds for each event category are
summarised in Table~\ref{tab:fit-plots}.

\begin{table}[ht!]
  \caption{Summary of control samples used to predict the SM
    background for each event category. }
  \label{tab:fit-plots}
  \centering
  \begin{tabular}{ lll }
    \hline
    \hline
    \njet   & \nb     & Control samples \\ [1.0ex]
    \hline
    2--3    & 0       & \mj, \mmj, \gj  \\
    2--3    & 1       & \mj, \mmj, \gj  \\
    2--3    & 2       & \mj             \\
    $\geq$4 & 0       & \mj, \mmj, \gj  \\
    $\geq$4 & 1       & \mj, \mmj, \gj  \\
    $\geq$4 & 2       & \mj             \\
    $\geq$4 & 3       & \mj             \\
    $\geq$4 & $\geq4$ & \mj             \\
    \hline
    \hline
  \end{tabular}
\end{table}

The selection criteria for the three control samples closely resemble
those for the signal region, differing mainly through the use of a
muon, di-muon, or photon {\it tag} (that is ignored in the calculation
of jet-based kinematic variables such as \scalht, \mht, \alphat, \etc)
and some minimal additional kinematic requirements (\eg invariant or
transerve mass windows) to obtain W, Z, and \ttbar-enriched event
samples. The same selection criteria are designed to suppress signal
contamination in the control samples so that unbiased data-driven
estimates for the SM backgrounds in the signal region can be
made. Hence, we refer to these samples as {\it control} samples
although in the final simultaneous fit, any potential signal
contamination is properly taken into account.

The control sample definitions and binning scheme are chosen so that
the reliance on simulation to extrapolate correctly from a control
region to the signal region is minimised. Many systematic effects are
expected to cancel largely in the transfer factor. However, a
systematic uncertainty is assigned to each transfer factor to account
for theoretical uncertainties and effects such as the mismodelling of
kinematics (\eg acceptances) and instrumental effects (\eg
reconstruction efficiencies), as described in
Sec.~\ref{sec:bkgd-syst}.

%Kinematic cuts are applied to enrich as much as possible the \wj,
%\ttbar, and \znunu components in the muon and di-muon control
%samples. The definition of the samples are geared towards efficiency
%rather than purity (even so, the purities are at the level $>$90\%)
%and any contamination from "backgrounds" (\eg \ttbar in the case of
%the \mmj sample) are simply incorporated into the transfer
%factors. Alternatively, a zero b-jet requirement can be applied to the
%\mmj sample to obtain a higher purity of Z$\rightarrow\mu\mu$ + jets
%events (\ie, with reduced contamination from \ttbar), which can then
%be used to give an expectation for the \znunu + jets background for
%all b-jet categories in the signal region.

\subsection{Definition of the control samples\label{sec:def-control-samples}}

\subsubsection{The \texorpdfstring{\mj}{muon plus jets} control sample}

Events from the \wj and \ttbar processes are found in the hadronic
signal sample due to unidentified leptons (either out of acceptance or
not reconstructed) and hadronic tau decays originating from
high-p$_{T}$ W bosons. An estimate of these background processes is
obtained through the use of a \mj sample. The selection criteria for
this sample are chosen to identify W bosons decaying to a muon and a
neutrino in the phase-space of the signal. The muon is not considered
in the calculation of event-level variables such as \scalht, \mht and
\alphat. All cuts on such jet-based quantities are consistent with
those applied in the hadronic search region and the same \njet, \nb,
and \scalht binning is used. The only exception is that no \alphat
requirement is made, as motivated by the discussion in
Sec.~\ref{sec:larger}. In order to select events containing W bosons,
exactly one tight isolated muon within an acceptance of \PT $>$ 30
\gev and $|\eta| <$ 2.1 is required (due to the trigger), and the
transverse mass of the W candidate must satisfy $30 < \mt(\mu,\pfmet)
< 125\gev$ (to suppress QCD multijet and potential signal
events). Events are vetoed if $\Delta R(\mu,\textrm{jet}_i) < 0.5$
running over all jets $i$. The single isolated track veto, described
in Sections~\ref{sec:reconstruction} and~\ref{sec:vetoes}, is also
applied, which considers all single isolated tracks in the event
except that associated with the identified, isolated muon. Finally,
the cleaning cut $\mht/\pfmet$ is also applied, as done in the signal
region, where the \pfmet is adjusted to account for the transverse
momentum of the identified, isolated muon.

% Events are vetoed if either of the following conditions are met:
% $\Delta R(\mu,\textrm{jet}_i) < 0.5$, running over all jets $i$; or
% a second muon candidate exists that is either loose, non-isolated or
% outside acceptance and the two muons have an invariant mass that
% satisfies $m_{Z} - 25 < M_{\mu_1\mu_2} < m_{Z} + 25$ (to suppress
% $Z\rightarrow\mu\mu$).

\subsubsection{The \texorpdfstring{\mmj}{di-muon plus jets} control sample}

The \znunu\ + jets process forms an irreducible background and can be
estimated using the \zmumu + jets process, which has similar kinematic
properties but a different acceptance and a smaller branching ratio. A
background estimate is obtained through the use of a \mmj sample. Most
of the selection criteria are identical to those for the \mj sample,
but the few that differ are tuned to identify Z bosons decaying to two
muons in the kinematic phase space of the signal region. The muons are
not considered in the calculation of event-level variables such as
\scalht, \mht and \alphat. All cuts on such jet-based quantities are
consistent with those applied in the hadronic search region and the
same \njet, \nb, and \scalht binning is used. The only exception is
that no \alphat requirement is made, as motivated by the discussion in
Sec.~\ref{sec:larger}. In order to select an event sample containing Z
bosons, exactly two tight isolated muons within an acceptance of $\Pt
> 30\gev$ and $|\eta| < 2.1$ are required (due to the trigger). The
invariant mass of the two muons must satisfy $m_{Z} - 25 <
M_{\mu_1\mu_2} < m_{Z} + 25$. Events are vetoed if $\Delta
R(\mu_{i},\textrm{jet}_j) < 0.5$ is satisfied, running over all muons
$i$ and all jets $j$. The single isolated track veto, described in
Sections~\ref{sec:reconstruction} and~\ref{sec:vetoes}, is also
applied, considering all single isolated tracks in the event except
those associated with the two identified, isolated muons. Finally, the
cleaning cut $\mht/\pfmet$ is also applied, as done in the signal
region, where the \pfmet is adjusted to account for the transverse
momenta of the two identified, isolated muons. The \mmj sample can be
used to make predictions in all the \scalht bins, providing coverage
at low \scalht where the \gj sample cannot.

\subsubsection{The \texorpdfstring{\gj}{photon plus jets} control sample}

The \znunu\ + jets process can also be estimated using the \gj
process, which has a larger cross section and kinematic properties
similar to those of \znunu\ events when the photon is
ignored~\cite{PAS-SUS-08-002,Bern:2011pa}. The \gj sample is defined
by requiring exactly one photon satisfying tight isolation criteria
and within an acceptance of $\pt > 165\gev$ and $|\eta| <
1.45$. Furthermore, events are vetoed if $\Delta
R(\gamma,\textrm{jet}_j) < 1.0$ is satisfied, running over all jets
$j$. As for the muon-based samples, the photon is not considered in
the calculation of event-level variables such as \scalht, \mht and
\alphat. All cuts on jet-based quantities are consistent with those
applied in the hadronic search region, and the same \HT binning is
used. Given that the photon is ignored, the \gj sample can only be
used for the region $\scalht > 375\gev$ due to the photon acceptance
of $\pt > 165\gev$ (enforced by the trigger) and the requirement
$\alphat > 0.55$.

\subsection{Increasing the acceptance of the muon control samples\label{sec:larger}}

As described in Sec.~\ref{sec:def-control-samples} above, the
selection criteria of the three control samples are defined such that
the background composition and event kinematics of the three control
samples mirror as closely as possible those for the signal
region. This is done in order to minimise the reliance on the
simulation to model correctly the backgrounds and event kinematics in
the control and signal samples.

However, in the case of the \mj and \mmj samples, no requirement is
made on \alphat in the selection criteria of the samples. This is made
possible by the remaining kinematic selection criteria, which are
sufficiently selective to ensure that the muon samples remain rich in
events from the \wj, \ttbar and \zmumu processes with negligible
contamination from QCD multijet events. These selection criteria
include, for example, requiring exactly one or two tight isolated
muon(s), and imposing acceptance windows on the invariant mass of the
di-muon sytem or the transverse mass of the muon-\pfmet system, as
described above. 
%The absence of QCD multijet events is demonstrated by the control
%distributions shown in Sec.~\ref{sec:est-control-samples} below. 
Thus, the acceptance of the two muon control samples can be
significantly increased, which simultaneously improves their
predictive power and further reduces the effect of any potential
signal contamination.  In the case of the \gj sample (used only for
the region $\scalht > 375\gev$), the requirement $\alphat > 0.55$ is
still necessary to suppress contamination from QCD multijet events,
even after the substantial photon \pt cut in the offline selection.

The extrapolation in the variable \alphat is tested through a
dedicated set of closure tests, described in Sec.~\ref{sec:bkgd-syst},
which demonstrate that the different \alphat requirements for the \mj
and \mmj control samples and signal region have no significant
systematic bias on the prediction. That the \alphat variable
introduces no acceptance bias for processes with genuine \met is due
to the accurate modelling by the CMS simulation of such processes,
namely W + jets, \ttbar, and \znunu\ + jets. Background estimates for
these processes are provided by the \mj and \mmj samples, as
identified at the beginning of this Section. Importantly, the same
cannot be said for QCD multijet events, as in this case the only
events that survive the \alphat cut are pathological cases in which a
jet is severely mismeasured or even lost due to detector
inefficiencies. Such effects certainly do change the event kinematics
and, in these cases, an \alphat cut will selectively choose events
with particular kinematic features and topologies. For these
pathological cases, one cannot rely on MC to model correctly the
behaviour and therefore the \alphat acceptance. This is a crucial
distinction to be made between QCD multijet events and processes with
significant genuine \met. The assumption is that processes with
genuine \met are selected by the \alphat variable based on the
escaping invisible particle(s) rather than any pathological effects.

%\subsection{Distributions from the data control samples\label{sec:est-control-samples}}
%
%Distributions of key variables for the \mj, \mmj, and \gj samples are
%shown in below in Figs.~\ref{fig:mu-distr}, \ref{fig:mumu-distr},
%and~\ref{fig:phot-distr}, respectively. The first two figures show
%the (leading) muon \pt and isolation distributions, along with the
%leading jet \pt, \scalht, \mht and \alphat distributions. For the \gj
%sample, the photon \pt distribution is%and isolation distributions are
%shown in place of the corresponding muon distribution.%s.
%No requirement is made on the number of b-jets per event. In general,
%the agreement between data and simulation is good, giving confidence
%that the samples are well understood. The MC distributions highlight
%the composition of each sample. The contribution from QCD multijet
%events is expected to be negligible.
%
%Figure~\ref{fig:jet-bjet} shows the jet and b-jet multiplicity
%distributions for the \mj, \mmj, and \gj control samples, as defined
%in Sec.~\ref{sec:def-control-samples} and following the requirement
%$\scalht > 375\gev$. An accurate modelling of the jet multiplicity in
%data is achieved for all samples. The b-jet distributions demonstrate
%the changing background composition as a function of the number of
%b-jets. For the \mj and \mmj samples, the requirement of zero b-jets
%results in sub-samples that are rich in W and Z bosons, respectively,
%with little contamination from \ttbar. The \ttbar background becomes
%dominant in the \mj sample when exactly one b-jet is required. The
%requirement of up to two b-tags per event significantly suppresses all
%processes except for \ttbar production. Requiring at least three
%b-tags also suppresses \ttbar production. In the case of the \mmj
%sample, some contamination from \ttbar is observed for $\nb =
%1$. Requiring more than one b-jet significantly suppresses the yields
%in both the \mmj and \gj samples.
%
%\clearpage
%\begin{figure}[!h]
%  \centering
%  \subfigure[Muon \pt.]{
%    \includegraphics[width=0.4\textwidth]{figures/data-mc/v1/muon/Stack_muPt_Muon_all_OneMuon_-1To-2b_log}
%  } 
%  \subfigure[Muon isolation.]{
%    \includegraphics[width=0.4\textwidth]{figures/data-mc/v1/muon/Stack_muonIso_Muon_all_OneMuon_-1To-2b_log}
%  } \\
%  \subfigure[Transverse mass.]{
%    \includegraphics[width=0.4\textwidth]{figures/data-mc/v1/muon/Stack_PFMTmu_Muon_all_OneMuon_-1To-2b_log}
%  } 
%  \subfigure[\scalht.]{
%    \includegraphics[width=0.4\textwidth]{figures/data-mc/v1/muon/Stack_HT_Muon_all_OneMuon_-1To-2b_log}
%  } \\
%  \subfigure[\mht.]{
%    \includegraphics[width=0.4\textwidth]{figures/data-mc/v1/muon/Stack_MHT_Muon_all_OneMuon_-1To-2b_log}
%  } 
%  \subfigure[\alphat.]{
%    \includegraphics[width=0.4\textwidth]{figures/data-mc/v1/muon/Stack_AlphaT_Muon_all_OneMuon_-1To-2b_log}
%  } 
%  \caption{Data--MC comparisons of key variables for the \mj control
%    sample, for the region $\scalht > 275\GeV$. Bands represent the
%    uncertainties due to the limited size of MC samples. No
%    requirement on the number of b-jets per event is made.}
%  \label{fig:mu-distr}
%\end{figure}
%
%\begin{figure}[!h]
%  \centering
%  \subfigure[Leading muon \pt.]{
%    \includegraphics[width=0.4\textwidth]{figures/data-mc/v1/mumu/Stack_muPt_Muon_all_DiMuon_-1To-2b_log}
%  } 
%  \subfigure[Muon isolation.]{
%    \includegraphics[width=0.4\textwidth]{figures/data-mc/v1/mumu/Stack_muonIso_Muon_all_DiMuon_-1To-2b_log}
%  } \\
%  \subfigure[Di-muon invariant mass.]{
%    \includegraphics[width=0.4\textwidth]{figures/data-mc/v1/mumu/Stack_Zmass_Muon_all_DiMuon_-1To-2b_log}
%  } 
%  \subfigure[\scalht.]{
%    \includegraphics[width=0.4\textwidth]{figures/data-mc/v1/mumu/Stack_HT_Muon_all_DiMuon_-1To-2b_log}
%  } \\
%  \subfigure[\mht.]{
%    \includegraphics[width=0.4\textwidth]{figures/data-mc/v1/mumu/Stack_MHT_Muon_all_DiMuon_-1To-2b_log}
%  } 
%  \subfigure[\alphat.]{
%    \includegraphics[width=0.4\textwidth]{figures/data-mc/v1/mumu/Stack_AlphaT_Muon_all_DiMuon_-1To-2b_log}
%  } 
%  \caption{Data--MC comparisons of key variables for the \mmj control
%    sample, for the region $\scalht > 275\GeV$. Bands represent the
%    uncertainties due to the limited size of MC samples. No
%    requirement on the number of b-jets per event is made.}
%  \label{fig:mumu-distr}
%\end{figure}
%
%\begin{figure}[!h]
%  \centering
%  \subfigure[Photon \pt.]{
%    \includegraphics[width=0.4\textwidth]{figures/data-mc/v1/photon/Stacked_PhotonPt_all_Photon_375_upwards}
%  } 
%  \subfigure[\scalht.]{
%    \includegraphics[width=0.4\textwidth]{figures/data-mc/v1/photon/Stacked_HT_after_alphaT_55_all_Photon_375_upwards}
%  } \\
%  \subfigure[\mht.]{
%    \includegraphics[width=0.4\textwidth]{figures/data-mc/v1/photon/Stacked_MHT_after_alphaT_55_all_Photon_375_upwards}
%  } 
%  \subfigure[\alphat.]{
%    \includegraphics[width=0.4\textwidth]{figures/data-mc/v1/photon/Stacked_AlphaT_all_Photon_375_upwards}
%  } 
%  \caption{Data--MC comparisons of key variables for the \gj control
%    sample, for the region $\scalht > 375\GeV$. Bands represent the
%    uncertainties due to the limited size of MC samples. No
%    requirement on the number of b-jets per event is made.}
%  \label{fig:phot-distr}
%\end{figure}
%
%\begin{figure}[!h]
%  \centering
%  \subfigure[\njet for the \mj sample.]{
%    \includegraphics[width=0.4\textwidth]{figures/data-mc/v1/muon/Stack_ncommjet_Muon_all_OneMuon_-1To-2b_log}
%  } 
%  \subfigure[\njet for the \mmj sample.]{
%    \includegraphics[width=0.4\textwidth]{figures/data-mc/v1/mumu/Stack_ncommjet_Muon_all_DiMuon_-1To-2b_log}
%  } \\
%  \subfigure[\njet for the \gj sample.]{
%    \includegraphics[width=0.4\textwidth]{figures/data-mc/v1/photon/Stacked_JetMultiplicityAfterAlphaT_55_all_Photon_375_upwards}
%  } 
%  \subfigure[\nb for the \mj sample.]{
%    \includegraphics[width=0.4\textwidth]{figures/data-mc/v1/muon/Stack_nbjet_Muon_all_OneMuon_-1To-2b_log}
%  } \\
%  \subfigure[\nb for the \mmj sample.]{
%    \includegraphics[width=0.4\textwidth]{figures/data-mc/v1/mumu/Stack_nbjet_Muon_all_DiMuon_-1To-2b_log}
%  } 
%  \subfigure[\nb for the \gj sample.]{
%    \includegraphics[width=0.4\textwidth]{figures/data-mc/v1/photon/Stacked_Btag_Post_AlphaT_5_55_all_Photon_375_upwards}
%  } 
%  \caption{Data--MC comparison of the number of reconstructed jets
%    (top) and b-jets (bottom) per event in the (left) \mj sample,
%    (middle) \mmj sample, and (right) \gj control sample. Bands
%    represent the uncertainties due to the limited size of MC
%    samples.}\label{fig:jet-bjet}
%\end{figure}

%\subsection{Transfer factors and ``na\"ive'' predictions\label{sec:est-control-samples}}
%
%Appendix~\ref{app:tf} contains tables summarising the observed and
%expected yields from data and simulation, respectively, in the bins of
%the three control samples. Also listed are the expectations from
%simulation for the various background contributions in the signal
%region, along with the corresponding transfer factors. The yields
%are binned in \scalht, jet multiplicity and number of b-jets per
%event. The errors associated with the transfer factors reflect the
%uncertainty due to the finite size of the MC samples used to determine
%the factors. Any trigger inefficiency is also factored into the
%transfer factors (\ie, the trigger is effectively emulated and
%yields from the MC samples are corrected to account for any
%inefficiency). Also, all MC expectations are corrected to account for
%any discrepancies between data and MC for the efficiency and mistag
%rate of the b-tagging algorithm used, as described further in
%Sec.~\ref{sec:btag-eff-correction}. However, no systematic
%uncertainties on the transfer factors are quoted in the tables.
%
%The same tables also list ``na\"ive'' predicted yields obtained from
%each control sample for individual SM backgrounds in the signal region
%(\eg, W + jets and \ttbar from the \mj sample, and \znunu + jets from
%the \mmj and \gj samples). These predictions are given for
%illustrative purposes only. For the analysis result, the predictions
%for the total SM background are determined by a fit to the yields in
%the signal region and all three control samples, as described in
%Sec.~\ref{sec:statistics}. In addition to observed yields, the fit
%takes as input the transfer factors with their associated
%statistical and systematic uncertainties. 
%
%Illustrative predictions for the total SM background can be made for
%each bin in the signal region, by combining the individual
%predictions. One such combination can be made by using the individual
%predictions from the \mj and \mmj samples (or, alternatively, the
%predictions from the \mj and \gj samples), the result of which can be
%compared with the observed yields in the bins of the signal
%region. Predictions in this way are made for bins of the three
%exclusive b-tag categories requiring exactly zero and one b-tags per
%event. When requiring at least two b-tagged jets per event, only the
%\mj sample has sufficiently large yields to predict accurately the
%total SM background. The errors on the total SM predictions reflect
%statistical uncertainties only. It is again noted that these
%``na\"ive'' predictions are for illustration only, with the final SM
%expectations for all signal region bins given by the simultaneous fit
%to yields in all data samples. Table~\ref{tab:tf-summary} summarises
%the contents of the various sections in the Appendix that list tables
%containing observed yields, MC expectations, transfer factors, and
%``na\"ive'' predictions for individual and total SM backgrounds.
%
%\begin{table}[h!]
%  \caption{Each section in Appendix~\ref{app:tf} contains tables that
%    list: observed yields, MC expectations, transfer factors, and
%    ``na\"ive'' predictions for individual SM backgrounds, for each of
%    the \mj, \mmj, and \gj control samples; and total SM predictions
%    when combining the individual predictions from the \mj and \mmj
%    samples and the \mj and \gj samples, separately.}
%  \label{tab:tf-summary}
%  \centering
%  \footnotesize
%  \begin{tabular}{ llll }
%    \hline
%    Section            & \njet bin & \nb bin & Control samples used \\ [0.5ex]
%    \hline
%    \ref{app:23j0b}    & 2--3      & 0       & \mj, \mmj, \gj       \\
%    \ref{app:23j1b}    & 2--3      & 1       & \mj, \mmj, \gj       \\
%    \ref{app:23j2b1mu} & 2--3      & 2       & \mj                  \\
%    \ref{app:4j0b}     & $\geq$4   & 0       & \mj, \mmj, \gj       \\
%    \ref{app:4j1b}     & $\geq$4   & 1       & \mj, \mmj, \gj       \\
%    \ref{app:4j2b1mu}  & $\geq$4   & 2       & \mj                  \\
%    \ref{app:4j3b1mu}  & $\geq$4   & 3       & \mj                  \\
%    \ref{app:4j4b1mu}  & $\geq$4   & $\geq$4 & \mj                  \\
%    \hline
%  \end{tabular}
%\end{table}


%\begin{table}[h!]
%  \caption{}
%  \label{tab:}
%  \centering
%  \footnotesize
%  \begin{tabular}{ lll }
%    \hline
%    \hline
%    \njet bin & \nb bin & Control samples used \\ [0.5ex]
%    \hline
%    2--3      & 0       & \mj, \mmj, \gj       \\
%    2--3      & 1       & \mj, \mmj, \gj       \\
%    2--3      & 2       & \mj                  \\
%    $\geq$4   & 0       & \mj, \mmj, \gj       \\
%    $\geq$4   & 1       & \mj, \mmj, \gj       \\
%    $\geq$4   & 2       & \mj                  \\
%    $\geq$4   & 3       & \mj                  \\
%    $\geq$4   & $\geq$4 & \mj                  \\
%    \hline
%    \hline
%  \end{tabular}
%\end{table}

