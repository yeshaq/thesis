%\includeonly{Introduction_thesis, The_Belle_Experiment, event_selection, cut_summary_table, continuum_suppression, branching_fraction_measurement, determination_of_signal_yield, gsim_and_toy_test, calibration, four_d_data_fit, bkgd_sub_mkpi, 4d_2d_toy, gsim_bias_correction, syst_wkst_wkpi, br_ul_sig, polarization, conclusion, bibliography}

%\includeonly{syst_wkst_wkpi, bibliography}


%%%%%%%%%%%%%%%%%%%%%%%%%%%%%%%%%%%%%%%%%%%%%%%%
%
%  PhD Thesis: master LATEX file 
%  -----------------------------
%
%%%%%%%%%%%%%%%%%%%%%%%%%%%%%%%%%%%%%%%%%%%%%%%%
%\documentclass[11pt,letter,twoside,psfig,epsfig,titlepage]{article}
\documentclass[11pt,letter,twoside,psfig,epsfig,]{article}
%\documentclass[11pt,a4paper,twoside,psfig,epsfig,titlepage]{report}
%\documentclass[11pt,a4paper,,psfig,epsfig, titlepage]{report}
%\newcommand{\squeezeup}{\vspace{-2.5mm}}
\usepackage{chngcntr}
\counterwithin{figure}{section}
\usepackage{a4wide}
\usepackage{graphicx}
\usepackage{amsmath}
\usepackage[square,numbers,sort&compress]{natbib}
\usepackage{url}
\usepackage{braket}
\usepackage[normal,small]{caption2}
\usepackage{afterpage}
\usepackage{float}
\setlength{\textfloatsep}{5pt}

\usepackage{subfigure}
\usepackage{fancyheadings}
%\usepackage{doublespace}
\usepackage{xspace}
%\usepackage[mediumspace]{SIunits}
\usepackage{enumerate}
\usepackage{setspace}

\input epsf
\usepackage{tabularx}
\usepackage{epsfig}
\usepackage{verbatim}
\usepackage{subfigure}
\usepackage{geometry}                % See geometry.pdf to learn the layout options. There are lots.
\usepackage{graphicx}
\usepackage{amssymb}
%\usepackage{header}
\usepackage{epstopdf}
\usepackage{feynmf}
\usepackage{color}
\usepackage[section]{placeins}
%\input{ptdr-definitions}
\input{symbols}
\input{mydefs}
\usepackage{natbib}
\usepackage[hidelinks,bookmarksnumbered,bookmarksopen,bookmarksopenlevel=1,colorlinks=false,pdfborder=0,plainpages=false,pdfpagelabels]{hyperref}
%\usepackage[T1]{fontenc}

%\unitlength=1mm
%\DeclareGraphicsRule{.tif}{png}{.png}{`convert #1 `dirname #1`/`basename #1 .tif`.png}
\usepackage{titlesec}

\titleformat*{\section}{\Huge\bfseries}
\titleformat*{\subsection}{\Large\bfseries}
\titleformat*{\subsubsection}{\large\bfseries}
\titleformat*{\paragraph}{\large\bfseries}
\titleformat*{\subparagraph}{\large\bfseries}


\setlength{\captionmargin}{1.2cm}

% Use the Adobe Postscript fonts...
\renewcommand{\rmdefault}{ptm}%{ptm}
%%\renewcommand{\ttdefault}{pcr}%{pcr}
\renewcommand{\sfdefault}{phv}%


%%%%%%%%%%%%%%%%%%%%%%%%%%%%%%%%%%%%%%%%%%%%%%%%%
%
% This part does a nice little line on the top of every page with
% chapter number and page number..
%
%%%%%%%%%%%%%%%%%%%%%%%%%%%%%%%%%%%%%%%%%%%%%%%%%
%\include{headings}
\pagestyle{fancyplain}

\addtolength{\headwidth}{\marginparsep}
\addtolength{\headwidth}{\marginparwidth}

\addtolength{\extrarowheight}{1pt}

% remember chapter title
%\renewcommand{\chaptermark}[1]%
%{\markboth{#1}{}}
% section number and title
\renewcommand{\sectionmark}[1]%
{\markright{\thesection\ #1}}
\lhead[\fancyplain{}{\bfseries\thepage}]%
{\fancyplain{}{\bfseries\rightmark}}
\rhead[\fancyplain{}{\bfseries\leftmark}]%
{\fancyplain{}{\bfseries\thepage}}
\cfoot{}

\setlength{\unitlength}{2mm}
\newcommand{\clearemptydoublepage}{\newpage{\pagestyle{empty}\cleardoublepage}}
\newcommand{\pageone}{}

\doublespacing

\begin{document}

%%%%%%%%%%%%%%%%%%%%%%%%%%%%%%%
%%%%%%%%%%%%%%%%%%%%%%%%%%%%%%%
%\title{{\boldmath Search for New Physics in All-hadronic Events with AlphaT in 8 TeV Data at CERN}}
%
%\author{
%  by\\
%  Yossof Eshaq\vspace{4mm}\\
%  %\scalebox{0.3}{\includegraphics{belle.eps}} \vspace{5.0mm} \\
%  {Submitted in Partial Fulfillment of the }\\
%  {Requirements for the Degree}\\
%  {Doctor of Philosophy}\\[1cm]
%  Supervised by Professor Aran Garcia-Bellido\\[4mm]
%  Department of Physics \& Astronomy\\
%  Arts, Sciences and Engineering \\
%  School of Arts and Sciences\\[1cm]
%  University of Rochester\\
%  Rochester, New York\\
%  2014
%}
%\date{}
%\maketitle
\clearpage
%%%%%%%%%%%%%%%%%%%%%%%%%%%%%%
%%%%%%%%%%%%%%%%%%%%%%%%%%%%%%
\clearemptydoublepage

\pagenumbering{roman}

%\addcontentsline{toc}{section}{Abstract} 
%
%\abstract{ An inclusive search for supersymmetric processes that
%produce final states with jets and missing transverse energy is
%performed in pp collisions at a centre-of-mass energy of $\sqrt{s} =
%8\TeV$. The data sample corresponds to an integrated luminosity of
%18.5\fbinv collected by the CMS experiment at the LHC. In this search,
%a dimensionless kinematic variable, \alphat, is used to discriminate
%between events with genuine and misreconstructed missing transverse
%energy. The search is based on an examination of the number of
%reconstructed jets per event, the scalar sum of transverse energies of
%these jets, and the number of these jets identified as originating
%from bottom quarks. The results are interpreted with various
%simplified models, with a special emphasis on models with a compressed
%mass spectrum.}
%

%%%% ACKNOWLEDGEMENTS -----------------------------------------------------------
%\clearemptydoublepage
%\clearemptydoublepage
%\clearpage
% \vspace*{4cm}
\section*{Acknowledgements}
%\vspace*{.5cm}

\indent I would like to thank my advisor, Aran Garcia-Bellido, without whom this work would not have happened. 
I thank him for always being available to help, but also for giving me the space to explore my own ideas. 
Without his trust and generosity I would have not spent the wonderful years I did at CERN. I am forever 
grateful to him. I'd like to thank Pablo Goldenzvweig for all the advice he gave me. It has been a great 
pleasure to have been associated with the faculty and staff of the University of Rochester. Connie, Laura, 
thank you for your dedication and vigilance that kept me on track! \\
\indent I am grateful to have worked with a great analysis team; thank you to (in no particular order): Ted Laird, 
Rob Bainbridge, Lucien Lo, Mark Baber, Henning Flaecher, Paris Sphicas, Jad Marrouche, Oliver Buchmueller, 
Bryn Mathias, Sam Rogerson, Darren Burton, Antonis Agapitos, Chris Lucas and Zhaoxia Meng. In some form or 
another they all contributed to this thesis and without their work it would not be what it is today. 
I would like to thank the supy development team, Burt Betchart, Davide Gerbaudo and Ted Laird for their contribution 
in creating such a wonderful analysis framework. My thanks go to the HCAL operations team, especially German 
Martinez who made me feel welcomed from day one and Sudan Paramesvaran for bringing a joyful spirit to HCAL.
I would like to thank Jared Sturdy for his patience when I interrupted his work to have him answer my questions! 
I'm forever grateful to Ted Laird. His help, advice, contributions to this thesis is immeasurable. This work is
heavily based on the framework he's built and the quality of his work has been an inspiration. I'd like to thank 
Mashalo, Richarou and the Laudato family for always being there for me. I would like to thank all my friends for 
the stories and laughs throughout the CERN years. \\
\indent I would like to thank Jennifer and Galen for welcoming me to their respective homes without
an ounce of hesitation, you are wonderful people. 
The love, encouragement and support from my parents, brother, and sister have gotten me to where I am today.
This thesis is dedicated to them. 

%\addcontentsline{toc}{section}{Acknowledgments}
\clearpage


% CONTENTS--------------------------------------------------------------
\clearemptydoublepage
\addcontentsline{toc}{section}{Contents}
\tableofcontents
%
%\clearemptydoublepage
%\addcontentsline{toc}{section}{List of Figures}
%\listoffigures
%%
%\clearemptydoublepage
%\addcontentsline{toc}{section}{List of Tables}
%\listoftables
%%
%%\clearemptydoublepage


%%% CHAPTERS-------------------------------------------------------------
\pagenumbering{arabic}

%\input introduction
\input theory
\input lhcAndCms
\input samples
%%\input reconstruction
\input eventSelection
%%\input trigger
%%\input alphat
\input backgrounds
\input systematics

\input statistics
\input results
\input signal
\input interpretation
\input conclusions

\appendix

\section{Background estimation from control samples\label{app:bk-yields}}
\begin{landscape}
\input{figures/tables/v22/background/tables_0b_le3j.tex}
\input{figures/tables/v22/background/tables_0b_ge4j.tex}
\input{figures/tables/v22/background/tables_1b_le3j.tex}
\input{figures/tables/v22/background/tables_1b_ge4j.tex}
\input{figures/tables/v22/background/tables_2b_le3j.tex}
\input{figures/tables/v22/background/tables_2b_ge4j.tex}
\end{landscape}

\clearpage
\section{Maximum likelihood parameter values\label{app:ml-params}}

\begin{table}\centering
\caption{SM-only maximum-likelihood parameter values (\njetlow, $\nb = 0$).}
\label{tab:mlParameterValues0b_le3j}
\begin{tabular}{lcc} name & value & error \\ \hline
$\mathrm{EWK}^{0}$ & {\tt  1.32e+04} & {\tt  1.1e+02}\\
$\mathrm{EWK}^{1}$ & {\tt  5.42e+03} & {\tt  7.0e+01}\\
$\mathrm{EWK}^{2}$ & {\tt  3.56e+03} & {\tt  5.6e+01}\\
$\mathrm{EWK}^{3}$ & {\tt  2.48e+03} & {\tt  4.2e+01}\\
$\mathrm{EWK}^{4}$ & {\tt  6.89e+02} & {\tt  1.5e+01}\\
$\mathrm{EWK}^{5}$ & {\tt  2.31e+02} & {\tt  1.1e+01}\\
$\mathrm{EWK}^{6}$ & {\tt  7.22e+01} & {\tt  4.5e+00}\\
$\mathrm{EWK}^{7}$ & {\tt  2.81e+01} & {\tt  2.9e+00}\\
$\mathrm{EWK}^{8}$ & {\tt  1.13e+01} & {\tt  1.6e+00}\\
$\mathrm{EWK}^{9}$ & {\tt  6.00e+00} & {\tt  1.0e+00}\\
$\mathrm{EWK}^{10}$ & {\tt  3.71e+00} & {\tt  7.2e-01}\\
$f_\mathrm{Zinv}^{0}$ & {\tt 0.55} & {\tt 0.01}\\
$f_\mathrm{Zinv}^{1}$ & {\tt 0.59} & {\tt 0.02}\\
$f_\mathrm{Zinv}^{2}$ & {\tt 0.59} & {\tt 0.02}\\
$f_\mathrm{Zinv}^{3}$ & {\tt 0.62} & {\tt 0.02}\\
$f_\mathrm{Zinv}^{4}$ & {\tt 0.65} & {\tt 0.02}\\
$f_\mathrm{Zinv}^{5}$ & {\tt 0.67} & {\tt 0.03}\\
$f_\mathrm{Zinv}^{6}$ & {\tt 0.71} & {\tt 0.04}\\
$f_\mathrm{Zinv}^{7}$ & {\tt 0.71} & {\tt 0.05}\\
$f_\mathrm{Zinv}^{8}$ & {\tt 0.75} & {\tt 0.05}\\
$f_\mathrm{Zinv}^{9}$ & {\tt 0.70} & {\tt 0.07}\\
$f_\mathrm{Zinv}^{10}$ & {\tt 0.77} & {\tt 0.06}\\
$\rho_{\mu\mu Z}^{0}$ & {\tt 0.94} & {\tt 0.03}\\
$\rho_{\mu\mu Z}^{1}$ & {\tt 1.03} & {\tt 0.04}\\
$\rho_{\mu\mu Z}^{2}$ & {\tt 0.97} & {\tt 0.04}\\
$\rho_{\mu\mu Z}^{3}$ & {\tt 1.00} & {\tt 0.04}\\
$\rho_{\mu\mu Z}^{4}$ & {\tt 1.00} & {\tt 0.08}\\
$\rho_{\mu\mu Z}^{5}$ & {\tt 1.02} & {\tt 0.13}\\
$\rho_{\mu\mu Z}^{6}$ & {\tt 0.98} & {\tt 0.16}\\
$\rho_{\mu W}^{0}$ & {\tt 0.95} & {\tt 0.03}\\
$\rho_{\mu W}^{1}$ & {\tt 1.02} & {\tt 0.05}\\
$\rho_{\mu W}^{2}$ & {\tt 0.98} & {\tt 0.05}\\
$\rho_{\mu W}^{3}$ & {\tt 1.00} & {\tt 0.06}\\
$\rho_{\mu W}^{4}$ & {\tt 0.99} & {\tt 0.11}\\
$\rho_{\mu W}^{5}$ & {\tt 0.99} & {\tt 0.16}\\
$\rho_{\mu W}^{6}$ & {\tt 0.98} & {\tt 0.18}\\
$\rho_{\gamma Z}^{3}$ & {\tt 0.99} & {\tt 0.04}\\
$\rho_{\gamma Z}^{4}$ & {\tt 0.98} & {\tt 0.07}\\
$\rho_{\gamma Z}^{5}$ & {\tt 0.96} & {\tt 0.12}\\
$\rho_{\gamma Z}^{6}$ & {\tt 0.97} & {\tt 0.16}\\
\hline
\end{tabular}
\end{table}

\begin{table}\centering
\caption{SM-only maximum-likelihood parameter values (\njetlow, $\nb = 1$).}
\label{tab:mlParameterValues1b_le3j}
\begin{tabular}{lcc}name & value & error \\ \hline
$\mathrm{EWK}^{0}$ & {\tt  1.74e+03} & {\tt  3.4e+01}\\
$\mathrm{EWK}^{1}$ & {\tt  8.43e+02} & {\tt  2.4e+01}\\
$\mathrm{EWK}^{2}$ & {\tt  5.80e+02} & {\tt  2.0e+01}\\
$\mathrm{EWK}^{3}$ & {\tt  4.13e+02} & {\tt  1.5e+01}\\
$\mathrm{EWK}^{4}$ & {\tt  1.02e+02} & {\tt  5.0e+00}\\
$\mathrm{EWK}^{5}$ & {\tt  2.70e+01} & {\tt  2.6e+00}\\
$\mathrm{EWK}^{6}$ & {\tt  9.14e+00} & {\tt  1.3e+00}\\
$\mathrm{EWK}^{7}$ & {\tt  3.25e+00} & {\tt  7.4e-01}\\
$\mathrm{EWK}^{8}$ & {\tt  2.33e+00} & {\tt  6.7e-01}\\
$\mathrm{EWK}^{9}$ & {\tt  3.42e-01} & {\tt  1.5e-01}\\
$\mathrm{EWK}^{10}$ & {\tt  2.27e-01} & {\tt  1.2e-01}\\
$f_\mathrm{Zinv}^{0}$ & {\tt 0.37} & {\tt 0.02}\\
$f_\mathrm{Zinv}^{1}$ & {\tt 0.42} & {\tt 0.03}\\
$f_\mathrm{Zinv}^{2}$ & {\tt 0.40} & {\tt 0.03}\\
$f_\mathrm{Zinv}^{3}$ & {\tt 0.44} & {\tt 0.03}\\
$f_\mathrm{Zinv}^{4}$ & {\tt 0.53} & {\tt 0.03}\\
$f_\mathrm{Zinv}^{5}$ & {\tt 0.57} & {\tt 0.05}\\
$f_\mathrm{Zinv}^{6}$ & {\tt 0.64} & {\tt 0.06}\\
$f_\mathrm{Zinv}^{7}$ & {\tt 0.61} & {\tt 0.10}\\
$f_\mathrm{Zinv}^{8}$ & {\tt 0.87} & {\tt 0.05}\\
$f_\mathrm{Zinv}^{9}$ & {\tt 0.39} & {\tt 0.24}\\
$f_\mathrm{Zinv}^{10}$ & {\tt 0.49} & {\tt 0.25}\\
$\rho_{\mu\mu Z}^{0}$ & {\tt 0.97} & {\tt 0.04}\\
$\rho_{\mu\mu Z}^{1}$ & {\tt 1.01} & {\tt 0.05}\\
$\rho_{\mu\mu Z}^{2}$ & {\tt 0.98} & {\tt 0.05}\\
$\rho_{\mu\mu Z}^{3}$ & {\tt 1.00} & {\tt 0.06}\\
$\rho_{\mu\mu Z}^{4}$ & {\tt 1.01} & {\tt 0.11}\\
$\rho_{\mu\mu Z}^{5}$ & {\tt 1.05} & {\tt 0.16}\\
$\rho_{\mu\mu Z}^{6}$ & {\tt 1.04} & {\tt 0.19}\\
$\rho_{\mu W}^{0}$ & {\tt 0.94} & {\tt 0.03}\\
$\rho_{\mu W}^{1}$ & {\tt 1.02} & {\tt 0.05}\\
$\rho_{\mu W}^{2}$ & {\tt 0.97} & {\tt 0.05}\\
$\rho_{\mu W}^{3}$ & {\tt 1.00} & {\tt 0.06}\\
$\rho_{\mu W}^{4}$ & {\tt 1.00} & {\tt 0.12}\\
$\rho_{\mu W}^{5}$ & {\tt 1.00} & {\tt 0.16}\\
$\rho_{\mu W}^{6}$ & {\tt 1.01} & {\tt 0.18}\\
$\rho_{\gamma Z}^{3}$ & {\tt 1.00} & {\tt 0.06}\\
$\rho_{\gamma Z}^{4}$ & {\tt 1.00} & {\tt 0.11}\\
$\rho_{\gamma Z}^{5}$ & {\tt 0.97} & {\tt 0.15}\\
$\rho_{\gamma Z}^{6}$ & {\tt 0.97} & {\tt 0.18}\\
\hline
\end{tabular}
\end{table}

\begin{table}\centering
\caption{SM-only maximum-likelihood parameter values (\njetlow, $\nb = 2$).}
\label{tab:mlParameterValues2b_le3j}
\begin{tabular}{lcc}name & value & error \\ \hline
$\mathrm{EWK}^{0}$ & {\tt  1.76e+02} & {\tt  6.6e+00}\\
$\mathrm{EWK}^{1}$ & {\tt  1.15e+02} & {\tt  6.1e+00}\\
$\mathrm{EWK}^{2}$ & {\tt  1.04e+02} & {\tt  5.8e+00}\\
$\mathrm{EWK}^{3}$ & {\tt  6.81e+01} & {\tt  4.7e+00}\\
$\mathrm{EWK}^{4}$ & {\tt  1.52e+01} & {\tt  1.4e+00}\\
$\mathrm{EWK}^{5}$ & {\tt  3.31e+00} & {\tt  5.6e-01}\\
$\mathrm{EWK}^{6}$ & {\tt  1.11e+00} & {\tt  2.7e-01}\\
$\mathrm{EWK}^{7}$ & {\tt  2.23e-01} & {\tt  7.4e-02}\\
$\mathrm{EWK}^{8}$ & {\tt  1.03e-01} & {\tt  4.2e-02}\\
$\rho_{\mu W}^{0}$ & {\tt 1.00} & {\tt 0.04}\\
$\rho_{\mu W}^{1}$ & {\tt 0.99} & {\tt 0.05}\\
$\rho_{\mu W}^{2}$ & {\tt 0.96} & {\tt 0.05}\\
$\rho_{\mu W}^{3}$ & {\tt 0.97} & {\tt 0.06}\\
$\rho_{\mu W}^{4}$ & {\tt 0.95} & {\tt 0.11}\\
$\rho_{\mu W}^{5}$ & {\tt 1.01} & {\tt 0.17}\\
\hline
\end{tabular}
\end{table}

%\begin{table}\centering
%\caption{SM-only maximum-likelihood parameter values (3b le3j).}
%\label{tab:mlParameterValues3b_le3j}
%\begin{tabular}{lcc}name & value & error \\ \hline
%$\mathrm{EWK}^{0}$ & {\tt  3.55e+00} & {\tt  3.4e-01}\\
%$\mathrm{EWK}^{1}$ & {\tt  4.16e+00} & {\tt  5.5e-01}\\
%$\mathrm{EWK}^{2}$ & {\tt  5.79e+00} & {\tt  8.7e-01}\\
%$\mathrm{EWK}^{3}$ & {\tt  2.47e+00} & {\tt  5.1e-01}\\
%$\mathrm{EWK}^{4}$ & {\tt  4.17e-01} & {\tt  1.7e-01}\\
%$\mathrm{EWK}^{5}$ & {\tt  2.17e-01} & {\tt  1.1e-01}\\
%$\mathrm{EWK}^{6}$ & {\tt  1.65e-01} & {\tt  1.2e-01}\\
%$\mathrm{EWK}^{7}$ & {\tt  0.00e+00} & {\tt  3.2e-03}\\
%$\mathrm{EWK}^{8}$ & {\tt  0.00e+00} & {\tt  1.3e-03}\\
%$\rho_{\mu W}^{0}$ & {\tt 1.00} & {\tt 0.04}\\
%$\rho_{\mu W}^{1}$ & {\tt 1.00} & {\tt 0.06}\\
%$\rho_{\mu W}^{2}$ & {\tt 1.01} & {\tt 0.06}\\
%$\rho_{\mu W}^{3}$ & {\tt 1.01} & {\tt 0.08}\\
%$\rho_{\mu W}^{4}$ & {\tt 1.01} & {\tt 0.12}\\
%$\rho_{\mu W}^{5}$ & {\tt 1.00} & {\tt 0.17}\\
%\hline
%\end{tabular}
%\end{table}

\begin{table}\centering
\caption{SM-only maximum-likelihood parameter values (\njethigh, $\nb = 0$).}
\label{tab:mlParameterValues0b_ge4j}
\begin{tabular}{lcc}name & value & error \\ \hline
$\mathrm{EWK}^{0}$ & {\tt  1.04e+02} & {\tt  7.1e+00}\\
$\mathrm{EWK}^{1}$ & {\tt  5.67e+02} & {\tt  2.0e+01}\\
$\mathrm{EWK}^{2}$ & {\tt  4.54e+02} & {\tt  1.9e+01}\\
$\mathrm{EWK}^{3}$ & {\tt  3.97e+02} & {\tt  1.5e+01}\\
$\mathrm{EWK}^{4}$ & {\tt  2.49e+02} & {\tt  1.0e+01}\\
$\mathrm{EWK}^{5}$ & {\tt  1.34e+02} & {\tt  8.8e+00}\\
$\mathrm{EWK}^{6}$ & {\tt  5.55e+01} & {\tt  4.5e+00}\\
$\mathrm{EWK}^{7}$ & {\tt  1.92e+01} & {\tt  2.4e+00}\\
$\mathrm{EWK}^{8}$ & {\tt  9.79e+00} & {\tt  1.6e+00}\\
$\mathrm{EWK}^{9}$ & {\tt  4.64e+00} & {\tt  1.1e+00}\\
$\mathrm{EWK}^{10}$ & {\tt  4.19e+00} & {\tt  8.8e-01}\\
$f_\mathrm{Zinv}^{0}$ & {\tt 0.48} & {\tt 0.04}\\
$f_\mathrm{Zinv}^{1}$ & {\tt 0.49} & {\tt 0.03}\\
$f_\mathrm{Zinv}^{2}$ & {\tt 0.43} & {\tt 0.05}\\
$f_\mathrm{Zinv}^{3}$ & {\tt 0.49} & {\tt 0.04}\\
$f_\mathrm{Zinv}^{4}$ & {\tt 0.56} & {\tt 0.04}\\
$f_\mathrm{Zinv}^{5}$ & {\tt 0.55} & {\tt 0.06}\\
$f_\mathrm{Zinv}^{6}$ & {\tt 0.61} & {\tt 0.06}\\
$f_\mathrm{Zinv}^{7}$ & {\tt 0.54} & {\tt 0.07}\\
$f_\mathrm{Zinv}^{8}$ & {\tt 0.67} & {\tt 0.07}\\
$f_\mathrm{Zinv}^{9}$ & {\tt 0.65} & {\tt 0.11}\\
$f_\mathrm{Zinv}^{10}$ & {\tt 0.48} & {\tt 0.13}\\
$\rho_{\mu\mu Z}^{0}$ & {\tt 1.00} & {\tt 0.06}\\
$\rho_{\mu\mu Z}^{1}$ & {\tt 0.96} & {\tt 0.05}\\
$\rho_{\mu\mu Z}^{2}$ & {\tt 0.99} & {\tt 0.09}\\
$\rho_{\mu\mu Z}^{3}$ & {\tt 0.98} & {\tt 0.08}\\
$\rho_{\mu\mu Z}^{4}$ & {\tt 0.86} & {\tt 0.11}\\
$\rho_{\mu\mu Z}^{5}$ & {\tt 0.96} & {\tt 0.15}\\
$\rho_{\mu\mu Z}^{6}$ & {\tt 0.90} & {\tt 0.19}\\
$\rho_{\mu W}^{0}$ & {\tt 1.00} & {\tt 0.06}\\
$\rho_{\mu W}^{1}$ & {\tt 0.96} & {\tt 0.05}\\
$\rho_{\mu W}^{2}$ & {\tt 0.99} & {\tt 0.09}\\
$\rho_{\mu W}^{3}$ & {\tt 0.92} & {\tt 0.08}\\
$\rho_{\mu W}^{4}$ & {\tt 0.81} & {\tt 0.13}\\
$\rho_{\mu W}^{5}$ & {\tt 0.99} & {\tt 0.16}\\
$\rho_{\mu W}^{6}$ & {\tt 0.87} & {\tt 0.20}\\
$\rho_{\gamma Z}^{3}$ & {\tt 0.91} & {\tt 0.07}\\
$\rho_{\gamma Z}^{4}$ & {\tt 0.90} & {\tt 0.11}\\
$\rho_{\gamma Z}^{5}$ & {\tt 1.02} & {\tt 0.15}\\
$\rho_{\gamma Z}^{6}$ & {\tt 0.94} & {\tt 0.19}\\
\hline
\end{tabular}
\end{table}

\begin{table}\centering
\caption{SM-only maximum-likelihood parameter values (\njethigh, $\nb = 1$).}
\label{tab:mlParameterValues1b_ge4j}
\begin{tabular}{lcc}name & value & error \\ \hline
$\mathrm{EWK}^{0}$ & {\tt  3.94e+01} & {\tt  3.0e+00}\\
$\mathrm{EWK}^{1}$ & {\tt  2.16e+02} & {\tt  9.8e+00}\\
$\mathrm{EWK}^{2}$ & {\tt  2.38e+02} & {\tt  1.3e+01}\\
$\mathrm{EWK}^{3}$ & {\tt  1.79e+02} & {\tt  9.4e+00}\\
$\mathrm{EWK}^{4}$ & {\tt  1.04e+02} & {\tt  5.9e+00}\\
$\mathrm{EWK}^{5}$ & {\tt  3.64e+01} & {\tt  3.7e+00}\\
$\mathrm{EWK}^{6}$ & {\tt  1.42e+01} & {\tt  1.8e+00}\\
$\mathrm{EWK}^{7}$ & {\tt  8.58e+00} & {\tt  1.4e+00}\\
$\mathrm{EWK}^{8}$ & {\tt  3.89e+00} & {\tt  8.7e-01}\\
$\mathrm{EWK}^{9}$ & {\tt  1.11e+00} & {\tt  3.7e-01}\\
$\mathrm{EWK}^{10}$ & {\tt  1.21e+00} & {\tt  3.9e-01}\\
$f_\mathrm{Zinv}^{0}$ & {\tt 0.17} & {\tt 0.05}\\
$f_\mathrm{Zinv}^{1}$ & {\tt 0.20} & {\tt 0.03}\\
$f_\mathrm{Zinv}^{2}$ & {\tt 0.19} & {\tt 0.05}\\
$f_\mathrm{Zinv}^{3}$ & {\tt 0.22} & {\tt 0.03}\\
$f_\mathrm{Zinv}^{4}$ & {\tt 0.23} & {\tt 0.03}\\
$f_\mathrm{Zinv}^{5}$ & {\tt 0.39} & {\tt 0.07}\\
$f_\mathrm{Zinv}^{6}$ & {\tt 0.36} & {\tt 0.08}\\
$f_\mathrm{Zinv}^{7}$ & {\tt 0.56} & {\tt 0.10}\\
$f_\mathrm{Zinv}^{8}$ & {\tt 0.52} & {\tt 0.12}\\
$f_\mathrm{Zinv}^{9}$ & {\tt 0.43} & {\tt 0.18}\\
$f_\mathrm{Zinv}^{10}$ & {\tt 0.62} & {\tt 0.14}\\
$\rho_{\mu\mu Z}^{0}$ & {\tt 1.00} & {\tt 0.06}\\
$\rho_{\mu\mu Z}^{1}$ & {\tt 1.01} & {\tt 0.06}\\
$\rho_{\mu\mu Z}^{2}$ & {\tt 0.98} & {\tt 0.10}\\
$\rho_{\mu\mu Z}^{3}$ & {\tt 1.00} & {\tt 0.09}\\
$\rho_{\mu\mu Z}^{4}$ & {\tt 1.02} & {\tt 0.16}\\
$\rho_{\mu\mu Z}^{5}$ & {\tt 0.96} & {\tt 0.16}\\
$\rho_{\mu\mu Z}^{6}$ & {\tt 0.93} & {\tt 0.20}\\
$\rho_{\mu W}^{0}$ & {\tt 1.00} & {\tt 0.06}\\
$\rho_{\mu W}^{1}$ & {\tt 1.02} & {\tt 0.05}\\
$\rho_{\mu W}^{2}$ & {\tt 0.90} & {\tt 0.07}\\
$\rho_{\mu W}^{3}$ & {\tt 0.95} & {\tt 0.06}\\
$\rho_{\mu W}^{4}$ & {\tt 1.00} & {\tt 0.14}\\
$\rho_{\mu W}^{5}$ & {\tt 0.90} & {\tt 0.16}\\
$\rho_{\mu W}^{6}$ & {\tt 1.01} & {\tt 0.22}\\
$\rho_{\gamma Z}^{3}$ & {\tt 0.98} & {\tt 0.09}\\
$\rho_{\gamma Z}^{4}$ & {\tt 0.97} & {\tt 0.15}\\
$\rho_{\gamma Z}^{5}$ & {\tt 0.92} & {\tt 0.15}\\
$\rho_{\gamma Z}^{6}$ & {\tt 1.09} & {\tt 0.23}\\
\hline
\end{tabular}
\end{table}

\begin{table}\centering
\caption{SM-only maximum-likelihood parameter values (\njethigh, $\nb = 2$).}
\label{tab:mlParameterValues2b_ge4j}
\begin{tabular}{lcc}name & value & error \\ \hline
$\mathrm{EWK}^{0}$ & {\tt  1.33e+01} & {\tt  1.0e+00}\\
$\mathrm{EWK}^{1}$ & {\tt  7.71e+01} & {\tt  4.3e+00}\\
$\mathrm{EWK}^{2}$ & {\tt  9.55e+01} & {\tt  7.2e+00}\\
$\mathrm{EWK}^{3}$ & {\tt  7.73e+01} & {\tt  5.6e+00}\\
$\mathrm{EWK}^{4}$ & {\tt  4.81e+01} & {\tt  3.7e+00}\\
$\mathrm{EWK}^{5}$ & {\tt  1.84e+01} & {\tt  2.6e+00}\\
$\mathrm{EWK}^{6}$ & {\tt  6.17e+00} & {\tt  9.6e-01}\\
$\mathrm{EWK}^{7}$ & {\tt  1.69e+00} & {\tt  3.7e-01}\\
$\mathrm{EWK}^{8}$ & {\tt  1.80e+00} & {\tt  3.9e-01}\\
$\rho_{\mu W}^{0}$ & {\tt 0.99} & {\tt 0.06}\\
$\rho_{\mu W}^{1}$ & {\tt 0.98} & {\tt 0.05}\\
$\rho_{\mu W}^{2}$ & {\tt 1.01} & {\tt 0.08}\\
$\rho_{\mu W}^{3}$ & {\tt 0.91} & {\tt 0.06}\\
$\rho_{\mu W}^{4}$ & {\tt 0.88} & {\tt 0.12}\\
$\rho_{\mu W}^{5}$ & {\tt 1.02} & {\tt 0.17}\\
\hline
\end{tabular}
\end{table}

\begin{table}\centering
\caption{SM-only maximum-likelihood parameter values (\njethigh, $\nb = 3$).}
\label{tab:mlParameterValues3b_ge4j}
\begin{tabular}{lcc}name & value & error \\ \hline
$\mathrm{EWK}^{0}$ & {\tt  9.83e-01} & {\tt  1.7e-01}\\
$\mathrm{EWK}^{1}$ & {\tt  8.17e+00} & {\tt  7.9e-01}\\
$\mathrm{EWK}^{2}$ & {\tt  1.09e+01} & {\tt  1.5e+00}\\
$\mathrm{EWK}^{3}$ & {\tt  8.21e+00} & {\tt  1.2e+00}\\
$\mathrm{EWK}^{4}$ & {\tt  5.75e+00} & {\tt  8.8e-01}\\
$\mathrm{EWK}^{5}$ & {\tt  2.23e+00} & {\tt  5.3e-01}\\
$\mathrm{EWK}^{6}$ & {\tt  8.81e-01} & {\tt  2.5e-01}\\
$\mathrm{EWK}^{7}$ & {\tt  1.99e-01} & {\tt  9.5e-02}\\
$\mathrm{EWK}^{8}$ & {\tt  2.04e-01} & {\tt  1.1e-01}\\
$\rho_{\mu W}^{0}$ & {\tt 1.00} & {\tt 0.06}\\
$\rho_{\mu W}^{1}$ & {\tt 1.00} & {\tt 0.06}\\
$\rho_{\mu W}^{2}$ & {\tt 1.03} & {\tt 0.10}\\
$\rho_{\mu W}^{3}$ & {\tt 0.97} & {\tt 0.09}\\
$\rho_{\mu W}^{4}$ & {\tt 0.97} & {\tt 0.16}\\
$\rho_{\mu W}^{5}$ & {\tt 1.01} & {\tt 0.17}\\
\hline
\end{tabular}
\end{table}

\begin{table}\centering
\caption{SM-only maximum-likelihood parameter values (\njethigh, $\nb \geq 4$).}
\label{tab:mlParameterValuesge4b_ge4j}
\begin{tabular}{lcc}name & value & error \\ \hline
$\mathrm{EWK}^{0}$ & {\tt  0.00e+00} & {\tt  8.5e-03}\\
$\mathrm{EWK}^{1}$ & {\tt  1.44e-01} & {\tt  8.5e-02}\\
$\mathrm{EWK}^{2}$ & {\tt  4.77e-01} & {\tt  2.8e-01}\\
$\mathrm{EWK}^{3}$ & {\tt  9.27e-01} & {\tt  3.0e-01}\\
$\rho_{\mu W}^{0}$ & {\tt 0.97} & {\tt 0.13}\\
\hline
\end{tabular}
\end{table}


\clearpage
\section{SM-only yield tables\label{app:ml-yields}}

The following tables compare the observations in the hadronic and
control samples with the maximum-likelihood expectations obtained by
the SM-only fit.

\begin{table}[ht!]
\caption{0b le3j}
\label{tab:ensemble-0b le3j}
\centering
\begin{tabular}{ lllllllll }

\hline
\scalht Bin (GeV)       & 375--475                       & 475--575                       & 575--675                       & 675--775                       & 775--875                       & 875--975                       & 975--1075                      & 1075--$\infty$                 \\ [1.000000ex]
\hline
SM hadronic\T           & $2637^{+50}_{-48}$             & $759^{+24}_{-23}$              & $252^{+14}_{-13}$              & $76.5^{+6.6}_{-4.7}$           & $33.7^{+3.5}_{-3.7}$           & $11.8^{+2.1}_{-2.3}$           & $6.3^{+1.4}_{-1.2}$            & $3.2^{+0.9}_{-0.9}$            \\ 
Data hadronic\B         & $2627$                         & $762$                          & $253$                          & $77$                           & $32$                           & $9$                            & $9$                            & $4$                            \\ 
\hline
SM $\mu$+jets\T         & $9074^{+82}_{-117}$            & $3546^{+53}_{-60}$             & $1538^{+36}_{-40}$             & $686^{+21}_{-27}$              & $325^{+17}_{-17}$              & $158^{+12}_{-12}$              & $78.6^{+6.4}_{-8.5}$           & $54.2^{+6.9}_{-7.6}$           \\ 
Data $\mu$+jets\B       & $9078$                         & $3545$                         & $1538$                         & $686$                          & $326$                          & $159$                          & $78$                           & $54$                           \\ 
\hline
SM $\gamma$+jets\T      & $3993^{+59}_{-60}$             & $1208^{+35}_{-35}$             & $408^{+18}_{-20}$              & $127^{+10}_{-9}$               & $48.8^{+6.2}_{-6.9}$           & $19.9^{+3.5}_{-4.2}$           & $12.0^{+3.0}_{-2.6}$           & $7.7^{+2.5}_{-2.6}$            \\ 
Data $\gamma$+jets\B    & $4000$                         & $1206$                         & $408$                          & $127$                          & $50$                           & $22$                           & $10$                           & $7$                            \\ 
\hline

\end{tabular}
\end{table}

\begin{table}[ht!]
\caption{0b ge4j}
\label{tab:ensemble-0b ge4j}
\centering
\begin{tabular}{ lllllllll }

\hline
\scalht Bin (GeV)       & 375--475                       & 475--575                       & 575--675                       & 675--775                       & 775--875                       & 875--975                       & 975--1075                      & 1075--$\infty$                 \\ [1.000000ex]
\hline
SM hadronic\T           & $460^{+16}_{-15}$              & $298^{+12}_{-11}$              & $146^{+7}_{-7}$                & $66.0^{+4.4}_{-5.5}$           & $27.1^{+3.2}_{-3.5}$           & $14.0^{+2.0}_{-1.9}$           & $6.5^{+1.4}_{-1.5}$            & $3.2^{+1.0}_{-1.0}$            \\ 
Data hadronic\B         & $470$                          & $302$                          & $158$                          & $66$                           & $28$                           & $15$                           & $6$                            & $2$                            \\ 
\hline
SM $\mu$+jets\T         & $1254^{+35}_{-36}$             & $889^{+31}_{-29}$              & $567^{+22}_{-23}$              & $308^{+17}_{-15}$              & $162^{+10}_{-13}$              & $81.3^{+8.9}_{-8.5}$           & $46.9^{+6.8}_{-7.0}$           & $28.6^{+6.4}_{-4.9}$           \\ 
Data $\mu$+jets\B       & $1249$                         & $888$                          & $562$                          & $308$                          & $162$                          & $81$                           & $47$                           & $29$                           \\ 
\hline
SM $\gamma$+jets\T      & $432^{+20}_{-20}$              & $346^{+17}_{-16}$              & $162^{+11}_{-12}$              & $83.0^{+7.1}_{-8.1}$           & $32.6^{+5.5}_{-4.9}$           & $21.8^{+3.7}_{-4.0}$           & $7.7^{+2.2}_{-2.2}$            & $4.2^{+1.9}_{-1.8}$            \\ 
Data $\gamma$+jets\B    & $427$                          & $344$                          & $155$                          & $83$                           & $32$                           & $21$                           & $8$                            & $5$                            \\ 
\hline

\end{tabular}
\end{table}

\begin{table}[ht!]
\caption{1b le3j}
\label{tab:ensemble-1b le3j}
\centering
\begin{tabular}{ lllllllll }

\hline
\scalht Bin (GeV)       & 375--475                       & 475--575                       & 575--675                       & 675--775                       & 775--875                       & 875--975                       & 975--1075                      & 1075--$\infty$                 \\ [1.000000ex]
\hline
SM hadronic\T           & $413^{+15}_{-14}$              & $111^{+6}_{-4}$                & $35.8^{+3.4}_{-2.7}$           & $10.1^{+1.6}_{-1.3}$           & $3.7^{+0.8}_{-0.8}$            & $1.6^{+0.6}_{-0.7}$            & $0.5^{+0.3}_{-0.4}$            & $0.1^{+0.1}_{-0.0}$            \\ 
Data hadronic\B         & $440$                          & $116$                          & $37$                           & $15$                           & $3$                            & $2$                            & $1$                            & $0$                            \\ 
\hline
SM $\mu$+jets\T         & $2286^{+45}_{-45}$             & $789^{+31}_{-26}$              & $326^{+20}_{-16}$              & $139^{+11}_{-11}$              & $62.7^{+8.0}_{-7.9}$           & $25.1^{+4.7}_{-4.9}$           & $16.1^{+4.0}_{-4.2}$           & $7.9^{+3.0}_{-2.9}$            \\ 
Data $\mu$+jets\B       & $2272$                         & $787$                          & $325$                          & $137$                          & $63$                           & $25$                           & $16$                           & $8$                            \\ 
\hline
SM $\gamma$+jets\T      & $457^{+23}_{-22}$              & $147^{+11}_{-9}$               & $49.7^{+6.5}_{-5.5}$           & $18.1^{+3.9}_{-3.6}$           & $5.6^{+2.1}_{-2.4}$            & $4.3^{+1.8}_{-2.2}$            & $1.4^{+0.9}_{-1.4}$            & $0.0^{+0.0}_{--0.0}$           \\ 
Data $\gamma$+jets\B    & $444$                          & $144$                          & $49$                           & $15$                           & $6$                            & $4$                            & $1$                            & $0$                            \\ 
\hline

\end{tabular}
\end{table}

\begin{table}[ht!]
\caption{1b ge4j}
\label{tab:ensemble-1b ge4j}
\centering
\begin{tabular}{ lllllllll }

\hline
\scalht Bin (GeV)       & 375--475                       & 475--575                       & 575--675                       & 675--775                       & 775--875                       & 875--975                       & 975--1075                      & 1075--$\infty$                 \\ [1.000000ex]
\hline
SM hadronic\T           & $192^{+9}_{-8}$                & $122^{+6}_{-6}$                & $44.8^{+3.2}_{-3.7}$           & $17.1^{+2.4}_{-2.0}$           & $6.8^{+1.2}_{-1.3}$            & $5.4^{+1.5}_{-1.5}$            & $2.4^{+1.0}_{-0.8}$            & $1.2^{+0.8}_{-0.8}$            \\ 
Data hadronic\B         & $196$                          & $132$                          & $44$                           & $14$                           & $8$                            & $6$                            & $2$                            & $0$                            \\ 
\hline
SM $\mu$+jets\T         & $1241^{+36}_{-33}$             & $888^{+29}_{-26}$              & $384^{+17}_{-20}$              & $200^{+14}_{-11}$              & $86.6^{+9.0}_{-10.7}$          & $55.3^{+7.8}_{-7.4}$           & $24.9^{+4.6}_{-5.6}$           & $10.7^{+3.1}_{-3.1}$           \\ 
Data $\mu$+jets\B       & $1238$                         & $881$                          & $385$                          & $202$                          & $86$                           & $55$                           & $25$                           & $11$                           \\ 
\hline
SM $\gamma$+jets\T      & $99.0^{+9.7}_{-8.7}$           & $79.8^{+8.0}_{-8.7}$           & $32.7^{+5.9}_{-5.3}$           & $11.9^{+3.6}_{-3.1}$           & $7.6^{+2.3}_{-2.8}$            & $3.3^{+1.2}_{-1.6}$            & $3.7^{+2.0}_{-1.6}$            & $1.1^{+0.8}_{-1.1}$            \\ 
Data $\gamma$+jets\B    & $98$                           & $77$                           & $33$                           & $13$                           & $7$                            & $3$                            & $4$                            & $2$                            \\ 
\hline

\end{tabular}
\end{table}

\begin{table}[ht!]
\caption{2b le3j}
\label{tab:ensemble-2b le3j}
\centering
\begin{tabular}{ lllllllll }

\hline
\scalht Bin (GeV)       & 375--475                       & 475--575                       & 575--675                       & 675--775                       & 775--875                       & 875--975                       & 975--1075                      & 1075--$\infty$                 \\ [1.000000ex]
\hline
SM hadronic\T           & $63.0^{+3.9}_{-4.1}$           & $18.0^{+1.3}_{-1.4}$           & $4.2^{+0.6}_{-0.5}$            & $1.1^{+0.2}_{-0.2}$            & $0.2^{+0.1}_{-0.1}$            & $0.0^{+0.0}_{-0.0}$            & $0.0^{+0.0}_{-0.0}$            & $0.0^{+0.0}_{-0.0}$            \\ 
Data hadronic\B         & $80$                           & $19$                           & $8$                            & $3$                            & $0$                            & $0$                            & $0$                            & $0$                            \\ 
\hline
SM $\mu$+jets\T         & $685^{+22}_{-29}$              & $218^{+13}_{-14}$              & $78.8^{+9.4}_{-9.0}$           & $19.9^{+4.0}_{-4.0}$           & $14.8^{+4.1}_{-3.9}$           & $5.0^{+2.0}_{-2.1}$            & $2.0^{+1.0}_{-1.0}$            & $1.0^{+1.0}_{-1.0}$            \\ 
Data $\mu$+jets\B       & $668$                          & $217$                          & $75$                           & $18$                           & $15$                           & $5$                            & $2$                            & $1$                            \\ 
\hline

\end{tabular}
\end{table}

%
%\documentclass[8pt]{article}
%\usepackage[landscape]{geometry}
%\usepackage{xspace}
%\newcommand{\alt}{\ensuremath{\alpha_{\rm{T}}}\xspace}
%\newcommand{\RaT}{\ensuremath{R_{\alt}}\xspace}
%\def\scalht{\mbox{$H_{\rm{T}}$}\xspace}
%\newcommand{\ra}{\ensuremath{\rightarrow}}
%\newcommand{\znunu}{\ensuremath{{\rm Z} \ra \nu\bar{\nu}}}
%\newcommand{\ttNew}{\ensuremath{\rm{t}\bar{\rm{t}}}\xspace}
%\newcommand\T{\rule{0pt}{2.6ex}}
%\newcommand\B{\rule[-1.2ex]{0pt}{0pt}}
%\begin{document}
\begin{table}[ht!]
\caption{2b ge4j}
\label{tab:ensemble-2b ge4j}
\centering
\begin{tabular}{ lllllllll }

\hline
\scalht Bin (GeV)       & 375--475                       & 475--575                       & 575--675                       & 675--775                       & 775--875                       & 875--975                       & 975--1075                      & 1075--$\infty$                 \\ [1.000000ex]
\hline
SM hadronic\T           & $74.6^{+3.0}_{-3.5}$           & $46.6^{+2.5}_{-2.3}$           & $19.6^{+1.6}_{-1.4}$           & $7.6^{+1.2}_{-1.1}$            & $1.9^{+0.4}_{-0.3}$            & $0.9^{+0.2}_{-0.2}$            & $0.4^{+0.1}_{-0.1}$            & $0.4^{+0.2}_{-0.1}$            \\ 
Data hadronic\B         & $72$                           & $52$                           & $31$                           & $12$                           & $1$                            & $2$                            & $0$                            & $1$                            \\ 
\hline
SM $\mu$+jets\T         & $757^{+23}_{-28}$              & $520^{+23}_{-21}$              & $285^{+18}_{-15}$              & $128^{+10}_{-10}$              & $54.1^{+8.8}_{-5.8}$           & $20.1^{+4.1}_{-4.7}$           & $10.6^{+3.0}_{-3.0}$           & $9.6^{+2.9}_{-2.9}$            \\ 
Data $\mu$+jets\B       & $760$                          & $515$                          & $274$                          & $124$                          & $55$                           & $19$                           & $11$                           & $9$                            \\ 
%\hline
%Data $\gamma$+jets\B    & $19$                           & $17$                           & $10$                           & $4$                            & $0$                            & $2$                            & $1$                            & $0$                            \\ 
\hline

\end{tabular}
\end{table}
%\end{document}


\clearpage
\section{Systematic uncertainties on simplified models\label{app:signal}}

\subsection{T2cc\label{app:t2cc}}

%\begin{figure}[h!]
%  \begin{center}
%    \subfigure[$m_{\sTop} = 250\gev, m_{\rm LSP} = 170\gev$]{
%      \includegraphics[width=0.6\textwidth, trim=0 0 0 30, clip=true]{figures/sms/t2cc/v25/T2cc_sig_inj_250_170}
%    } \\
%%    \subfigure[$m_{\sTop} = 250\gev, m_{\rm LSP} = 230\gev$]{
%%      \includegraphics[width=0.6\textwidth, trim=0 0 0 30, clip=true]{figures/sms/t2cc/v25/T2cc_sig_inj_250_230}
%%    } \\
%    \subfigure[$m_{\sTop} = 250\gev, m_{\rm LSP} = 240\gev$]{
%      \includegraphics[width=0.6\textwidth, trim=0 0 0 30, clip=true]{figures/sms/t2cc/v25/T2cc_sig_inj_250_240}
%    } \\
%    \caption{The expected signal significance (in terms of the number
%      of standard deviations) per signal region bin for the
%      \texttt{T2cc} model with $m_{\sTop} = 250\gev$ and $m_{\rm LSP}
%      = 170\gev$ (Top) and $m_{\rm LSP} = 170\gev$ (Bottom).}
%    % the best fit point $m_{\rm LSP} = 170\gev$ (Middle) 
%    \label{fig:sms-t2cc-sig}
%  \end{center}
%\end{figure}

\begin{figure}[h!]
  \begin{center}
    \subfigure[\label{fig:sms-pdf-t2cc-0b_le3j}\njetlow, $\nb = 0$]{
      \includegraphics[width=0.43\textwidth,page=2]{figures/sms/t2cc/v1/t2cc_unc}
    }
%    \subfigure[\label{fig:sms-pdf-t2cc-1b_le3j}\njetlow, $\nb = 1$]{
%      \includegraphics[width=0.43\textwidth,page=9]{figures/sms/t2cc/v1/t2cc_unc}
%    }\\
    \subfigure[\label{fig:sms-pdf-t2cc-0b_ge4j}\njethigh, $\nb = 0$]{
      \includegraphics[width=0.43\textwidth,page=23]{figures/sms/t2cc/v1/t2cc_unc}
    }
    \subfigure[\label{fig:sms-pdf-t2cc-1b_ge4j}\njethigh, $\nb = 1$]{
      \includegraphics[width=0.43\textwidth,page=30]{figures/sms/t2cc/v1/t2cc_unc}
    }\\
    \caption{\label{fig:sms-pdf-t2cc}Ratio of efficiency times
      acceptance for the central value of the envelope calculation relative 
      to the nominal PDF set used to produce the \texttt{T2cc} sample. The categories
      used to interpret \texttt{T2cc} are shown.}
  \end{center}
\end{figure}

\begin{figure}[h!]
  \begin{center}
    \subfigure[\njetlow, $\nb = 0$.]{
      \includegraphics[width=0.35\textwidth, page=7]{figures/sms/t2cc/v1/t2cc_unc}
    }
    \subfigure[\njetlow, $\nb = 0$.]{
      \includegraphics[width=0.35\textwidth, page=6]{figures/sms/t2cc/v1/t2cc_unc}
    }\\
%   \subfigure[\njetlow, $\nb = 1$.]{
%      \includegraphics[width=0.35\textwidth, page=14]{figures/sms/t2cc/v1/t2cc_unc}
%    }
%    \subfigure[\njetlow, $\nb = 1$.]{
%      \includegraphics[width=0.35\textwidth, page=13]{figures/sms/t2cc/v1/t2cc_unc}
%    }\\
    \subfigure[\njethigh, $\nb = 0$.]{
      \includegraphics[width=0.35\textwidth, page=28]{figures/sms/t2cc/v1/t2cc_unc}
    }
    \subfigure[\njethigh, $\nb = 0$.]{
      \includegraphics[width=0.35\textwidth, page=27]{figures/sms/t2cc/v1/t2cc_unc}
    }\\
    \subfigure[\njethigh, $\nb = 1$.]{
      \includegraphics[width=0.35\textwidth, page=35]{figures/sms/t2cc/v1/t2cc_unc}
    }  
    \subfigure[\njethigh, $\nb = 1$.]{
      \includegraphics[width=0.35\textwidth, page=34]{figures/sms/t2cc/v1/t2cc_unc}
    }\\
    \caption{\label{fig:sms-jes-t2cc}The fractional change in
      signal efficiency due to systematically (Left) decreasing and
      (Right) increasing all jet energies by their JES uncertainties. For
      each mass point, the largest value between the two variations is assigned
      as the JES systematic uncertainty. The categories used to interpret \texttt{T2cc} are shown.}
  \end{center}
\end{figure}

\begin{figure}[h!]
  \begin{center}
    \subfigure[\njetlow, $\nb = 0$.]{
      \includegraphics[width=0.35\textwidth, page=1]{figures/sms/t2cc/v1/t2cc_unc}
    }
    \subfigure[\njetlow, $\nb = 0$.]{
      \includegraphics[width=0.35\textwidth, page=4]{figures/sms/t2cc/v1/t2cc_unc}
    }\\
%    \subfigure[\njetlow, $\nb = 1$.]{
%      \includegraphics[width=0.35\textwidth, page=8]{figures/sms/t2cc/v1/t2cc_unc}
%    }
%    \subfigure[\njetlow, $\nb = 1$.]{
%      \includegraphics[width=0.35\textwidth, page=11]{figures/sms/t2cc/v1/t2cc_unc}
%    }\\
    \subfigure[\njethigh, $\nb = 0$.]{
      \includegraphics[width=0.35\textwidth, page=22]{figures/sms/t2cc/v1/t2cc_unc}
    }
    \subfigure[\njethigh, $\nb = 0$.]{
      \includegraphics[width=0.35\textwidth, page=25]{figures/sms/t2cc/v1/t2cc_unc}
    }\\
    \subfigure[\njethigh, $\nb = 1$.]{
      \includegraphics[width=0.35\textwidth, page=29]{figures/sms/t2cc/v1/t2cc_unc}
    }  
    \subfigure[\njethigh, $\nb = 1$.]{
      \includegraphics[width=0.35\textwidth, page=32]{figures/sms/t2cc/v1/t2cc_unc}
    }\\
    \caption{\label{fig:sms-isr-t2cc}The fractional change in signal
      efficiency due to systematically (Left) decreasing and (Right)
      increasing event weights according to ISR uncertainties. For
      each mass point, the largest value between the two variations is assigned
      as the ISR systematic uncertainty. The categories used to interpret \texttt{T2cc} are shown.}
  \end{center}
\end{figure}

\begin{figure}[h!]
  \begin{center}
    \subfigure[\njetlow, $\nb = 0$.]{
      \includegraphics[width=0.35\textwidth, page=3]{figures/sms/t2cc/v1/t2cc_unc}
    }
    \subfigure[\njetlow, $\nb = 0$.]{
      \includegraphics[width=0.35\textwidth, page=5]{figures/sms/t2cc/v1/t2cc_unc}
    }\\
%    \subfigure[\njetlow, $\nb = 1$.]{
%      \includegraphics[width=0.35\textwidth, page=10]{figures/sms/t2cc/v1/t2cc_unc}
%    }
%    \subfigure[\njetlow, $\nb = 1$.]{
%      \includegraphics[width=0.35\textwidth, page=12]{figures/sms/t2cc/v1/t2cc_unc}
%    }\\
    \subfigure[\njethigh, $\nb = 0$.]{
      \includegraphics[width=0.35\textwidth, page=24]{figures/sms/t2cc/v1/t2cc_unc}
    }
    \subfigure[\njethigh, $\nb = 0$.]{
      \includegraphics[width=0.35\textwidth, page=26]{figures/sms/t2cc/v1/t2cc_unc}
    }\\
    \subfigure[\njethigh, $\nb = 1$.]{
      \includegraphics[width=0.35\textwidth, page=31]{figures/sms/t2cc/v1/t2cc_unc}
    }  
    \subfigure[\njethigh, $\nb = 1$.]{
      \includegraphics[width=0.35\textwidth, page=33]{figures/sms/t2cc/v1/t2cc_unc}
    }\\
    \caption{\label{fig:sms-btag-t2cc}The fractional change in signal
      efficiency due to systematically (Left) decreasing and (Right)
      increasing all b-tag efficiencies according to the scale factor
      uncertainties. For each mass point, the largest value between the 
      two variations is assigned as the b-tag systematic uncertainty. 
      The categories used to interpret \texttt{T2cc} are shown.}
  \end{center}
\end{figure}

%\begin{figure}[h!]
%  \begin{center}
%    \subfigure[\njetlow, $\nb = 0$.]{
%     \includegraphics[width=0.48\textwidth,page=1]{figures/sms/t2cc/v1/t2cc_pfJet_totalUnc.pdf}
%    }                                                                  
%    \subfigure[\njetlow, $\nb = 1$.]{                                  
%     \includegraphics[width=0.48\textwidth,page=2]{figures/sms/t2cc/v1/t2cc_pfJet_totalUnc.pdf}
%    }\\                                                                
%    \subfigure[\njethigh, $\nb = 0$.]{                                 
%      \includegraphics[width=0.48\textwidth,page=5]{figures/sms/t2cc/v1/t2cc_pfJet_totalUnc.pdf}
%    }                                                                  
%    \subfigure[\njethigh, $\nb = 1$.]{                                 
%      \includegraphics[width=0.48\textwidth,page=6]{figures/sms/t2cc/v1/t2cc_pfJet_totalUnc.pdf}
%    }\\
%    \caption{\label{fig:sms-total-t2cc}The total systematic
%      uncertainty in the signal efficiency times acceptance for all
%      relevant event categories for the \texttt{T2cc} intepretation.}
%  \end{center}
%\end{figure}

%\FloatBarrier
%%%%%%%%%%%%%%%%%%%%%%%%%%%%%%%%%%%%%%%%%%%%%%%%%%%%%%%%%%%%%%%%%%%%%%%%%%%%%%%%
%%%%%%%%%%%%%%%%%%%%%%%%%%%%%%%%%%%%%%%%%%%%%%%%%%%%%%%%%%%%%%%%%%%%%%%%%%%%%%%%
%%%%%%%%%%%%%%%%%%%%%%%%%%%%%%%%%%%%%%%%%%%%%%%%%%%%%%%%%%%%%%%%%%%%%%%%%%%%%%%%

\clearpage
\subsection{T2tt\label{app:t2tt}}

%
\begin{figure}[h!]
  \begin{center}
    \subfigure[\label{fig:sms-pdf-t2tt-1b_ge4j}\njethigh, $\nb = 1$]{
      \includegraphics[width=0.43\textwidth,page=2]{figures/sms/t2tt/v1/t2tt_unc}
    }
    \subfigure[\label{fig:sms-pdf-t2tt-2b_ge4j}\njethigh, $\nb = 2$]{
      \includegraphics[width=0.43\textwidth,page=2]{figures/sms/t2tt/v1/t2tt_unc}
    }\\
    \caption{\label{fig:sms-pdf-t2tt}Ratio of efficiency times
      acceptance for the central value of the envelope calculation relative 
      to the nominal PDF set used to produce the \texttt{T2tt} sample. The categories
      used to interpret \texttt{T2tt} are shown.}
  \end{center}
\end{figure}

\begin{figure}[h!]
  \begin{center}
    \subfigure[\njethigh, $\nb = 1$.]{
      \includegraphics[width=0.35\textwidth, page=14]{figures/sms/t2tt/v1/t2tt_unc}
    }
    \subfigure[\njethigh, $\nb = 1$.]{
      \includegraphics[width=0.35\textwidth, page=13]{figures/sms/t2tt/v1/t2tt_unc}
    }\\
    \subfigure[\njethigh, $\nb = 2$.]{
      \includegraphics[width=0.35\textwidth, page=21]{figures/sms/t2tt/v1/t2tt_unc}
    }
    \subfigure[\njethigh, $\nb = 2$.]{
      \includegraphics[width=0.35\textwidth, page=20]{figures/sms/t2tt/v1/t2tt_unc}
      }\\     
      \caption{\label{fig:sms-jes-t2tt}The fractional change in
      signal efficiency due to systematically (Left) decreasing and
      (Right) increasing all jet energies by their JES uncertainties. For
      each mass point, the largest value between the two variations is assigned
      as the JES systematic uncertainty. The categories used to interpret \texttt{T2tt} are shown.}
  \end{center}
\end{figure}

\begin{figure}[h!]
  \begin{center}
    \subfigure[\njethigh, $\nb = 1$.]{
      \includegraphics[width=0.35\textwidth, page=8]{figures/sms/t2tt/v1/t2tt_unc}
    }
    \subfigure[\njethigh, $\nb = 1$.]{
      \includegraphics[width=0.35\textwidth, page=11]{figures/sms/t2tt/v1/t2tt_unc}
    }\\
    \subfigure[\njethigh, $\nb = 2$.]{
      \includegraphics[width=0.35\textwidth, page=15]{figures/sms/t2tt/v1/t2tt_unc}
    }
    \subfigure[\njethigh, $\nb = 2$.]{
      \includegraphics[width=0.35\textwidth, page=18]{figures/sms/t2tt/v1/t2tt_unc}
    }\\
    \caption{\label{fig:sms-isr-t2tt}The fractional change in signal
      efficiency due to systematically (Left) decreasing and (Right)
      increasing event weights according to ISR uncertainties. For
      each mass point, the largest value between the two variations is assigned
      as the ISR systematic uncertainty. The categories used to interpret \texttt{T2tt} are shown.}
  \end{center}
\end{figure}

\begin{figure}[h!]
  \begin{center}
    \subfigure[\njethigh, $\nb = 1$.]{
      \includegraphics[width=0.35\textwidth, page=10]{figures/sms/t2tt/v1/t2tt_unc}
    }
    \subfigure[\njethigh, $\nb = 1$.]{
      \includegraphics[width=0.35\textwidth, page=12]{figures/sms/t2tt/v1/t2tt_unc}
    }\\
    \subfigure[\njethigh, $\nb = 2$.]{
      \includegraphics[width=0.35\textwidth, page=17]{figures/sms/t2tt/v1/t2tt_unc}
    }
    \subfigure[\njethigh, $\nb = 2$.]{
      \includegraphics[width=0.35\textwidth, page=19]{figures/sms/t2tt/v1/t2tt_unc}
    }\\
    \caption{\label{fig:sms-btag-t2tt}The fractional change in signal
      efficiency due to systematically (Left) decreasing and (Right)
      increasing all b-tag efficiencies according to the scale factor
      uncertainties. For each mass point, the largest value between the 
      two variations is assigned as the b-tag systematic uncertainty. 
      The categories used to interpret \texttt{T2tt} are shown.}
  \end{center}
\end{figure}


\input{example}

\clearpage
\bibliography{alphaT}
\bibliographystyle{unsrt}


\end{document}   
