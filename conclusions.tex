\clearpage
\section{Summary\label{sec:conclusions}}

An inclusive search for supersymmetry with the CMS experiment is
reported, based on a data sample of pp collisions collected at
$\sqrt{s} = 8\TeV$, corresponding to an integrated luminosity of $18.5
\pm 0.4 \fbinv$.  Final states with two or more energetic jets and
significant \met, as expected from the production and decay of massive
squarks and gluinos, have been analysed. The analysis strategy
reported here has increased the coverage of SUSY parameter space with
respect to previous searches.

The analysis strategy is to maximise the sensitivity of the search to
a wide variety of SUSY event topologies arising from squark-squark,
squark-gluino, and gluino-gluino production and decay, particularly
those with compressed mass spectra, while still maintaining the
inclusive nature of the search.

The signal region is binned according to the number of reconstructed
jets, the scalar sum of the transverse energy of jets, and the number
of jets identified to originate from bottom quarks. The sum of
standard model backgrounds per bin has been estimated from a
simultaneous binned likelihood fit to event yields in the signal
region and $\mu$ + jets, $\mu\mu$ + jets, and $\gamma$ + jets control
samples. The observed yields in the signal region are found to be in
agreement with the expected contributions from standard model
processes.  

The observed yields in the signal region are found to be in agreement
with the expected contributions from standard model processes. Limits
are set in the SUSY particle mass parameter space of a simplified
model that assumes the direct pair production of top squarks and the
loop-induced two-body decay to a charm quark and neutralino. Top
squark masses below $\sim$200\gev are disfavoured while a mass limit
of $275 \pm 25 \gev$ is expected. A maximum likelihood value for the
signal strength parameter $\mu$ of $1.00 \pm 0.33$ is observed for
$m_{\sTop} = 250\gev$ and $m_{\chiz} = 230\gev$ with a global
significance of 2.6$\sigma$. 
%This same mass point yields the largest significance of 3.2$\sigma$
%under the constraint $\mu = 1$, which is reduced to 2.5$\sigma$ when
%accounting for a factor of seven due to the look-elsewhere effect.

%Limits are set in the SUSY particle mass parameter space of simplified
%models, with an emphasis on compressed-spectrum scenarios. The results
%can also be used to perform interpretations in other relevant models,
%such as the CMSSM.
