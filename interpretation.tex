\clearpage
\section{Limits on SMS production cross sections\label{sec:interpretation}}

Upper limits on the production cross section of the two simplified 
SUSY models described in chapter~\ref{sec:signal} are discussed in this
chapter. 


\subsection{Limit setting procedure\label{sec:cls}}

The following has been formulated in~\cite{LairdThesis} and is summarized 
here for completeness. 
Consider the likelihood ratio~\cite{Cowan:2010js}:

\begin{equation}
\label{eq:PLR}
\lambda(\mu) = \frac{ L(\mu,
\hat{\hat{\vec{\theta}}}) } {L(\hat{\mu}, \hat{\vec{\theta}}) } \;.
\end{equation}

Where $\hat{\hat{\vec{\theta}}}$ in the numerator denotes the
value of $\vec{\theta}$, the set of nuisance parameters in the likelihood,
that maximizes $L$ for the specified $\mu$.
The denominator is the maximized likelihood function, i.e., $\hat{\mu}$
and $\hat{\vec{\theta}}$ are their ML estimators. In the context of this
analysis $\mu \equiv f$ where $f$ is the signal strength as defined 
in section~\ref{sec:signalContrib}.  One can define the profile
likelihood ratio test statistic as~\cite{Cowan:2010js}:

\begin{equation}
\label{eq:qmu}
q_{\mu} =
\left\{ \! \! \begin{array}{ll}
               - 2 \ln \lambda(\mu)  & \hat{\mu} \le \mu  \;, \\*[0.2 cm]
               0 & \hat{\mu} > \mu \;,
              \end{array}
       \right.
\end{equation}

To test a specific signal model, one approach would be to perform many 
pseudo-experiments under the two separate hypotheses: background-only 
(i.e. $\mu=0$) and background with signal (i.e. $\mu=1$). $CL_{b}$ and 
$CL_{s+b}$~\cite{read,Junk} are defined as 1 - the quantile of $q_{\mu(obs)}$ in each of 
these distributions, where $q_{\mu(obs)}$ is the test statistic evaluated 
with the observed data. Figure~\ref{fig:hybrid_plot} illustrates two example distributions of
the test statistic under each hypothesis.  $CL_{s}$ is then defined as $CL_{s} \equiv CL_{s+b}/CL_{b}$, 
and a value $CL_{s} < 0.05$ excludes the signal hypothesis at 95\% confidence level. 

Another approach is to instead calculate the upper limit on the signal 
cross section by determining $f$ at $CL_{s} = 0.05$.  This approach samples 
CLs at varying signal strengths and the upper limit is 
determined through interpolation. If the upper limit on the cross section 
is lower than the theoritical cross section, the signal model is excluded.

\begin{figure}[h!t]
  \begin{center}
      \includegraphics[width=0.45\textwidth,]{figures/hybrid_plot}
      \caption{\label{fig:hybrid_plot} The distributions of the test statistic $q_{\mu}$
        in the background-only (red, on the right) and signal+background (blue, on the left) hypotheses. 
        The black line represents the value of the $q_{\mu}$ on the tested data. The shaded areas represent 
        $1-CL_{b}$ (red) and $CL_{s+b}$ (blue).  From~\cite{Moneta:1289965}}.
    \label{fig:hybrid_plot}
  \end{center}
\end{figure}


\subsection{Upper limits on SMS models}

For each mass-pair in a given model, the upper limit is calculated as described in 
section~\ref{sec:cls} using the likelihood constructed in section~\ref{sec:likelihood}
by considering the relevant categories listed in table~\ref{tab:simplified-models}.

\subsubsection{Upper limits on T2cc}
The color scale in figure~\ref{fig:upperLimits-t2cc} shows the observed upper 
limit at 95\% confidence level for each sparticle mass-pair bin in the SMS 
model where a stop particle decays to a charm quark and a neutralino (\texttt{T2cc}). 
The mass-pair bins left of the thick black curve have a calculated cross 
section upper limit below the theoritical production cross section, 
i.e. $\frac{\sigma_{\texttt{obs. upper limit}}}{\sigma_{\texttt{theory}}} < 1$, and
are therfore excluded. The thinner black lines represent exclusion regions obtained by
comparing the calculated upper limit with the theoritical cross section shifted up and 
down by its theoritical uncertainty i.e. 
$\frac{\sigma_{\texttt{obs. upper limit}}}{\sigma_{\texttt{theory} \pm \texttt{uncertainty}}} < 1$.
The dashed purple lines indicate the median (thick line) $\pm 1 \sigma$ 
(thin lines) expected exclusion regions. The $\pm 1 \sigma$ expected exclusion 
regions are not obtained by varying the theoritical cross section but rather 
by comparing the calculated expected upper limit $\pm 1 \sigma$ uncertainy
with the nominal cross section. The observed upper limit curve lies within
$+1 \sigma$ expected curve.  

\subsubsection{Upper limits on T2tt}

The color scale in figure~\ref{fig:upperLimits-t2tt} shows the observed upper 
limit at 95\% confidence level for each sparticle mass-pair bin in the SMS 
model where a stop particle decays to a top quark and a neutralino (\texttt{T2tt}). 
The definitions of the excluded regions remain the same as \texttt{t2cc}.
The observed upper limit disagrees with the expected by more than $2 \sigma$.
This discrepency is further studied in the following sections. 

\subsubsection{Discrepency\label{crap}}
The descripency is investigated by a closer study of two bins 
in $(m_{\st},m_{\text{LSP}})$.  
Figures~\ref{fig:t2tt-best-fit-400} and~\ref{fig:t2tt-best-fit-575} 
show the \scalht-binned observed data yields and expectations for the 
hadronic sample, as determined by a simultaneous fit to all data 
samples under a signal+background hypothesis. The observed event 
yields in data (black dots), the SM expectations (dark blue) and the sum of 
the SM backgrounds and signal expectation (pink) are shown. 
Two event categories are considered by the fit: (Left) \njethigh and 
$\nb = 1$, (Right) \njethigh and $\nb = 2$. The fit is performed 
simultaneously across both event categories. The signal expectations are 
for the best fit mass-pair bins $m_{\sq} = 400\GeV$ and $m_{\text{LSP}} =
0\GeV$ (Top) and $m_{\sq} = 575\GeV$ and $m_{\text{LSP}} = 0$.


%\begin{table}[h!]
%  \caption{The first two columns specify the model and its
%    production and decay. The next column specifies the event
%    categories (in terms of \njet and
%    \nb) considered for each interpretation. The last 
%    two columns indicate the search sensitivity for each model,
%    where $m_{\sq(\sGlu)}^{\textrm{best}}$ and
%    $m_{\textrm{LSP}}^{\textrm{best}}$ represent the largest mass 
%    beyond which no limit can be set for squarks/gluinos and the LSP,
%    respectively. The exclusion range for $m_{\sq(\sGlu)}$ is bounded
%    from below by the kinematic region considered for each model, as
%    defined in the text. The quoted estimates are determined 
%    conservatively from the observed exclusion based on the
%    theoretical production cross section minus $1\sigma$
%    uncertainty. 
%    %For model \texttt{T2tt}, the search is at the threshold of
%    %sensitivity for the considered ($m_{\sQua},m_{\rm LSP}$) parameter
%    %space, as discussed in the text. 
%  }  
%  \label{tab:sms}
%  \centering
%  \footnotesize
%  \begin{tabular}{ llcccc }
%    \hline
%    Model
%    & Production/decay
%    & Event categories
%    & Limit plot
%%    & $m_{\sq(\sGlu)}^{\textrm{best}}$~(GeV) 
%%    & $m_{\textrm{LSP}}^{\textrm{best}}$~(GeV) 
%    \\ [0.5ex]
%    \hline
%    \texttt{T2cc}
%    &
%    $\textrm{pp}\,\rightarrow\,\sTop\sTop\,\rightarrow\,\textrm{c}\chiz\bar{\textrm{c}}\chiz$
%    & ($\le3$,0), ($\ge4$,0), ($\ge4$,1)
%    & \ref{fig:upperLimits-t2cc}
%%    & 250
%%    & 250
%     \\
%    \texttt{T2tt} 
%    & 
%    $\textrm{pp}\,\rightarrow\,\sTop\sTop\,\rightarrow\,\textrm{t}\chiz\bar{\textrm{t}}\chiz$
%    & ($\geq4$,1),($\geq4$,2)
%    & \ref{fig:upperLimits-t2tt}
%%    & 400 
%%    & 25
%    \\ 
%    \hline
%  \end{tabular}
%\end{table}

%\fixme{TEXT REFLECTS USUAL PRESENTATION OF LIMIT PLOTS - ALL RELEVANT
%  INFORMATION IN SHOWN IN REFERENCED FIGURES FOR T2CC - TO BE UPDATED.}
%Figures~\ref{fig:limits-t2cc-exp} and \ref{fig:limits-t2cc-obs} show
%the upper limit on the cross section at 95\% CL as a function of
%$m_{\sq}$ or $m_{\gl}$ and $m_{\rm LSP}$ for various simplified
%models. The point-to-point fluctuations are due to the finite number
%of pseudo-experiments used to determine the observed upper limit. The
%solid thick black line indicates the observed exclusion region
%assuming NLO+NLL~\cite{Beenakker:1996ch, susy-nlo-nll} SUSY cross
%section for squark pair production in the limit of very massive
%gluinos (or vice versa). The thin black lines represent the observed
%excluded region when varying the cross section by its theoretical
%uncertainty. The dashed purple lines indicate the median (thick line)
%$\pm 1 \sigma$ (thin lines) expected exclusion regions.
%
%%Figure~\ref{fig:t2cc-1d} shows the observed upper limit at 95\% CL on
%%the production cross section as a function of the top squark mass
%%($m_{\sTop}$) for the model \texttt{T2cc} when considering two
%%different \sTop-\chiz mass splittings of $\Delta m = 10\gev$ (left)
%%and $\Delta m = 80\gev$ (right). The observed upper limit (95\% CL) on
%%the production cross section is shown as a function of $m_{\sTop}$
%%(solid line), along with the expected upper limit and
%%$\mathbf{\pm2}\sigma$ {\bf experimental uncertainties} (long-dashed
%%line with shaded band), and the NLO+NLL top squark pair-production
%%cross section and theoretical uncertainties (dotted line with shaded
%%band).
%
%Figure~\ref{fig:t2cc-best-fit} shows the \scalht-binned observed data
%yields and expectations for the hadronic sample, as determined by a
%simultaneous fit to all data samples under the signal+background
%hypothesis. The observed event yields in data (black dots), the SM
%expectations (dark blue) and the sum of the SM backgrounds and signal
%expectation (pink) are shown. The signal expectations are for the best
%fit model \texttt{T2cc} with $m_{\sq} = 250\GeV$ and $m_{\text{LSP}} =
%230\GeV$. Three event categories are considered by the fit: (Top)
%\njetlow and $\nb = 0$, (Middle) \njethigh and $\nb = 0$, (Bottom)
%\njethigh and $\nb = 1$. The fit is performed (Left) for each
%individual event category or (Right) simultaneously across all three
%event categories.
%
%\begin{figure}[h!]
%  \begin{center}
%    \subfigure[$+1\sigma$ experimental, relative]{
%      \includegraphics[width=0.45\textwidth,page=3,trim=40 50 20 70,clip=true]{figures/limits/v0/exp/CLs_frequentist_T2cc_2012dev_0b_le3j_0b_ge4j_1b_ge4j_xsLimit_relative}
%    } \quad
%    \subfigure[$+1\sigma$ experimental, excluded points]{
%      \includegraphics[width=0.45\textwidth,page=3,trim=40 50 20 70,clip=true]{figures/limits/v0/exp/CLs_frequentist_T2cc_2012dev_0b_le3j_0b_ge4j_1b_ge4j_xsLimit_simpleExcl}
%    } \\
%    \subfigure[Nominal, relative]{
%      \includegraphics[width=0.45\textwidth,page=1,trim=40 50 20 70,clip=true]{figures/limits/v0/exp/CLs_frequentist_T2cc_2012dev_0b_le3j_0b_ge4j_1b_ge4j_xsLimit_relative}
%    } \quad 
%    \subfigure[Nominal, excluded points]{
%      \includegraphics[width=0.45\textwidth,page=1,trim=40 50 20 70,clip=true]{figures/limits/v0/exp/CLs_frequentist_T2cc_2012dev_0b_le3j_0b_ge4j_1b_ge4j_xsLimit_simpleExcl}
%    } \\
%    \subfigure[$-1\sigma$ experimental, relative]{
%      \includegraphics[width=0.45\textwidth,page=2,trim=40 50 20 70,clip=true]{figures/limits/v0/exp/CLs_frequentist_T2cc_2012dev_0b_le3j_0b_ge4j_1b_ge4j_xsLimit_relative}
%    } \quad 
%    \subfigure[$-1\sigma$ experimental, excluded points]{
%      \includegraphics[width=0.45\textwidth,page=2,trim=40 50 20 70,clip=true]{figures/limits/v0/exp/CLs_frequentist_T2cc_2012dev_0b_le3j_0b_ge4j_1b_ge4j_xsLimit_simpleExcl}
%    } \\
%    \caption{\label{fig:limits-t2cc-exp} \fixme{TEMPORARY PLACE
%        HOLDERS FOR THE FINAL LIMIT PLOT. HOWEVER, ALL RELEVANT
%        INFORMATION IS CONTAINED HERE.} Expected limits for the model 
%      \texttt{T2cc}. In the left column, the plots show the upper
%      limit on the production cross section relative to the
%      theoretical value. In the right column, the plots show the mass
%      points that are excluded (marked red). In all plots, the yellow
%      line should be ignored. The expected limits are shown in the
%      middle row, with limits corresponding to the $+1\sigma$ and
%      $+1\sigma$ variations in the experimental uncertainties shown
%      top and bottom, respectively. }
%  \end{center}
%\end{figure}
%
\begin{figure}[h!]
  \begin{center}
    \subfigure[\label{fig:upperLimits-t2cc}]{
      \includegraphics[width=0.70\textwidth,clip=true]{figures/limits/merged/T2cc/v2/CLs_frequentist_T2cc_2012pf_0b_le3j_0b_ge4j_1b_ge4j_xsLimit}
      }\\
    \subfigure[\label{fig:upperLimits-t2tt}]{
      \includegraphics[width=0.70\textwidth,clip=true,]{figures/limits/merged/T2tt/v6/CLs_frequentist_T2tt_2012pf_1b_ge4j_2b_ge4j_xsLimit}
      }
%    \subfigure[Observed, relative]{
%      \includegraphics[width=0.45\textwidth,page=4,trim=40 50 20 70,clip=true]{figures/limits/v0/obs/CLs_frequentist_T2cc_2012dev_0b_le3j_0b_ge4j_1b_ge4j_xsLimit_relative}
%    } \quad 
%    \subfigure[Observed, excluded points]{
%      \includegraphics[width=0.45\textwidth,page=4,trim=40 50 20 70,clip=true]{figures/limits/v0/obs/CLs_frequentist_T2cc_2012dev_0b_le3j_0b_ge4j_1b_ge4j_xsLimit_simpleExcl}
%    } \\
    \caption{\label{fig:upperLimits} Expected 
    and observed upper limits on the production cross section 
    for the for the models \texttt{T2cc} (top) and \texttt{T2tt} (bottom). }
  \end{center}
\end{figure}

\begin{figure*}[t!]
  \begin{center}
    \subfigure[\njethigh, $\nb = 1$, simultaneous fit]{
      \includegraphics[width=0.45\textwidth]{figures/fit/v22/wSignal/400_0/bestFit_2012pf_RQcdZero_fZinvAll_1b_ge4j-1hp_2b_ge4j-1h_signal_sel1b_ge4j}
    } 
    \subfigure[\njethigh, $\nb = 2$, simultaneous fit]{
      \includegraphics[width=0.45\textwidth]{figures/fit/v22/wSignal/400_0/bestFit_2012pf_RQcdZero_fZinvAll_1b_ge4j-1hp_2b_ge4j-1h_signal_sel2b_ge4j}
    } \\
    \caption{\label{fig:t2tt-best-fit-400}The comparison of
      the \scalht-binned observed data yields and expectations for the
      hadronic sample, as determined by a simultaneous fit to all data
      samples under the signal plus SM background hypothesis. The
      observed event yields in data (black dots), the SM expectations
      (dark blue solid line), and the signal expectations (pink solid
      line), as determined by the simultaneous fit, for the 
      signal model \texttt{T2tt} with $m_{\st} = 400\GeV$ and
      $m_{\text{LSP}} = 0\GeV$. Two event categories are
      considered: (a) \njethigh and $\nb = 1$, (b) \njethigh and
      $\nb = 2$.}
  \end{center}
\end{figure*}
\begin{figure*}[t!]
  \begin{center}
    \subfigure[profile likelihood ratio, simultaneous fit]{
      \includegraphics[width=0.45\textwidth]{figures/fit/v22/wSignal/400_0/intervalPlot_2012pf_RQcdZero_fZinvAll_1b_ge4j-1hp_2b_ge4j-1h_signal_95}
    } \\
    \caption{\label{fig:t2cc-int-400}The profile likelihood ratio 
      (defined in sec.~\ref{sec:cls}) as a function of the signal strength.
      The likelihood considers all data samples under the signal plus SM 
      background hypothesis for the signal model \texttt{T2tt} with 
      $m_{\st} = 400\GeV$ and $m_{\text{LSP}} = 0\GeV$.
      The minimum defines the signal strength estimate which maximizes the
      likelihood and the green vertical line on the right of the minimum 
      indicates the upper-limit at 95\% confidence level.}
  \end{center}
\end{figure*}

\begin{figure*}[t!]
  \begin{center}
     \subfigure[\njethigh, $\nb = 1$, simultaneous fit]{
      \includegraphics[width=0.45\textwidth,page=13]{figures/fit/v22/stackedSig/400_0/bestFit_2012pf_RQcdZero_fZinvAll_1b_ge4j-1p_2b_ge4j-1_sel1b_ge4j_smOnly.pdf}
    } 
    \subfigure[\njethigh, $\nb = 2$, simultaneous fit]{
      \includegraphics[width=0.45\textwidth,page=9]{figures/fit/v22/stackedSig/400_0/bestFit_2012pf_RQcdZero_fZinvAll_1b_ge4j-1p_2b_ge4j-1_sel2b_ge4j_smOnly.pdf}
    } \\
    \caption{\label{fig:t2tt-sig-400} The \scalht-binned 
      signal significance defined as the signal yield 
      divided by the $\sqrt{b+(.1b)^2}$ where $b$ is the
      SM expectation obtained by a fit to all 
      control data samples under the SM-only background 
      hypothesis for the two categories (a) \njethigh, $\nb = 1$ and (b) 
      \njethigh, $\nb = 2$ simultaneously. 
      The signal model is \texttt{T2tt} with 
      $m_{\st} = 400\GeV$ and $m_{\text{LSP}} = 0\GeV$.} 
  \end{center}
\end{figure*}

\begin{figure*}[t!]
  \begin{center}
    \subfigure[\njethigh, $\nb = 1$, simultaneous fit]{
      \includegraphics[width=0.45\textwidth]{figures/fit/v22/wSignal/575_0/bestFit_2012pf_RQcdZero_fZinvAll_1b_ge4j-1hp_2b_ge4j-1h_signal_sel1b_ge4j}
    } 
    \subfigure[\njethigh, $\nb = 2$, simultaneous fit]{
      \includegraphics[width=0.45\textwidth]{figures/fit/v22/wSignal/575_0/bestFit_2012pf_RQcdZero_fZinvAll_1b_ge4j-1hp_2b_ge4j-1h_signal_sel2b_ge4j}
    } \\
    \caption{\label{fig:t2tt-best-fit-575}The comparison of
      the \scalht-binned observed data yields and expectations for the
      hadronic sample, as determined by a simultaneous fit to all data
      samples under the signal plus SM background hypothesis. The
      observed event yields in data (black dots), the SM expectations
      (dark blue solid line), and the signal expectations (pink solid
      line), as determined by the simultaneous fit, for the
      signal model \texttt{T2tt} with $m_{\st} = 575\GeV$ and
      $m_{\text{LSP}} = 0\GeV$. Two event categories are
      considered: (a) \njethigh and $\nb = 1$, (b) \njethigh and
      $\nb = 2$.}
  \end{center}
\end{figure*}
\begin{figure*}[t!]
  \begin{center}
    \subfigure[profile likelihood ratio, simultaneous fit]{
      \includegraphics[width=0.45\textwidth]{figures/fit/v22/wSignal/575_0/intervalPlot_2012pf_RQcdZero_fZinvAll_1b_ge4j-1hp_2b_ge4j-1h_signal_95}
    } \\
    \caption{\label{fig:t2tt-int-575}The profile likelihood ratio 
      (defined in sec.~\ref{sec:cls}) as a function of the signal strength.
      The likelihood considers all data samples under the signal plus SM 
      background hypothesis for the signal model \texttt{T2tt} with 
      $m_{\st} = 575\GeV$ and $m_{\text{LSP}} = 0\GeV$.
      The minimum defines the signal strength estimate which maximizes the
      likelihood and the green vertical line on the right of the minimum 
      indicates the upper-limit at 95\% confidence level.}
  \end{center}
\end{figure*}

\begin{figure*}[t!]
  \begin{center}
     \subfigure[\njethigh, $\nb = 1$, simultaneous fit]{
      \includegraphics[width=0.45\textwidth,page=13]{figures/fit/v22/stackedSig/575_0/bestFit_2012pf_RQcdZero_fZinvAll_1b_ge4j-1p_2b_ge4j-1_sel1b_ge4j_smOnly.pdf}
    } 
    \subfigure[\njethigh, $\nb = 2$, simultaneous fit]{
      \includegraphics[width=0.45\textwidth,page=9]{figures/fit/v22/stackedSig/575_0/bestFit_2012pf_RQcdZero_fZinvAll_1b_ge4j-1p_2b_ge4j-1_sel2b_ge4j_smOnly.pdf}
    } \\
    \caption{\label{fig:t2cc-sig-575} The \scalht-binned 
      signal significance defined as the signal yield 
      divided by the $\sqrt{b+(.1b)^2}$ where $b$ is the
      SM expectation obtained by a fit to all 
      control data samples under the SM-only background 
      hypothesis for the two categories (a) \njethigh, $\nb = 1$ and (b) 
      \njethigh, $\nb = 2$ simultaneously. 
      The signal model is \texttt{T2tt} with 
      $m_{\st} = 575\GeV$ and $m_{\text{LSP}} = 0\GeV$.} 
  \end{center}
\end{figure*}

%%Figure~\ref{fig:limits-sms} shows the upper limit on the cross section
%%at 95\% CL as a function of $m_{\sq}$ or $m_{\gl}$ and $m_{\rm LSP}$
%%for various simplified models. The point-to-point fluctuations are due
%%to the finite number of pseudo-experiments used to determine the
%%observed upper limit. The solid thick black line indicates the
%%observed exclusion region assuming NLO+NLL~\cite{Beenakker:1996ch,
%%  susy-nlo-nll} SUSY cross section for squark pair production in the
%%limit of very massive gluinos (or vice versa). The thin black lines
%%represent the observed excluded region when varying the cross section
%%by its theoretical uncertainty. The dashed purple lines indicate the
%%median (thick line) $\pm 1 \sigma$ (thin lines) expected exclusion
%%regions.
%
%%The estimates on mass limits are determined conservatively from the
%%observed exclusion based on the theoretical production cross section
%%minus $1\sigma$ uncertainty.  The most stringent mass limits on
%%pair-produced sparticles are obtained at low LSP masses, while the
%%limits typically weaken for compressed spectra, \ie, points close to
%%the diagonal. In particular, for all of the considered simplified
%%models, there is an LSP mass beyond which no limit can be set. This is
%%illustrated in Figure~\ref{fig:t1}, where the most stringent limit on
%%the gluino mass of $950\GeV$ is obtained for low LSP masses. This
%%limit only weakens to $900\GeV$ when the LSP mass reaches
%%$425\GeV$. However, for LSP masses above $450\GeV$, no mass range can
%%be excluded for gluinos decaying to first- or second-generation
%%quarks. Table~\ref{tab:sms} summarises the mass limits obtained from
%%the considered simplified models.
%%
%%\begin{figure}[h!]
%%  \begin{center}
%%    \subfigure[\label{fig:t1}$\sGlu\sGlu\,\rightarrow\,\textrm{q}\bar{\textrm{q}}\chiz \textrm{q}\bar{\textrm{q}}\chiz$ (Model \texttt{T1})]{
%%      \includegraphics[width=0.45\textwidth]{figures/limits/v4/t1}
%%    } \quad
%%    \subfigure[\label{fig:t2}$\sQua\sQua\,\rightarrow\,\textrm{q}\chiz \bar{\textrm{q}}\chiz$ (Model \texttt{T2})]{ 
%%      \includegraphics[width=0.45\textwidth]{figures/limits/v4/t2}
%%    } \\
%%%    \subfigure[\label{fig:t2tt}$\sTop\sTop\,\rightarrow\,\textrm{t}\chiz \bar{\textrm{t}}\chiz$ (Model \texttt{T2tt})]{ 
%%%      \includegraphics[width=0.45\textwidth]{figures/limits/v1/t2tt}
%%%    } \quad 
%%    \subfigure[\label{fig:t2bb}$\sBot\sBot\,\rightarrow\,\textrm{b}\chiz \bar{\textrm{b}}\chiz$ (Model \texttt{T2bb})]{ 
%%      \includegraphics[width=0.45\textwidth]{figures/limits/v4/t2bb}
%%    } \\
%%    \subfigure[\label{fig:t1tttt}$\sGlu\sGlu\,\rightarrow\,\textrm{t}\bar{\textrm{t}}\chiz \textrm{t}\bar{\textrm{t}}\chiz$ (Model \texttt{T1tttt})]{
%%      \includegraphics[width=0.45\textwidth]{figures/limits/v4/t1tttt}
%%    } \quad 
%%    \subfigure[\label{fig:t1bbbb}$\sGlu\sGlu\,\rightarrow\,\textrm{b}\bar{\textrm{b}}\chiz \textrm{b}\bar{\textrm{b}}\chiz$ (Model \texttt{T1bbbb})]{
%%      \includegraphics[width=0.45\textwidth]{figures/limits/v4/t1bbbb}
%%    } \\
%%    \caption{\label{fig:limits-sms} Upper limit on cross section at
%%      95\% CL as a function of $m_{\sq}$ or $m_{\gl}$ and $m_{\rm
%%        LSP}$ for various simplified models. The solid thick black
%%      line indicates the observed exclusion region assuming NLO+NLL
%%      SUSY production cross section. The thin black lines represent
%%      the observed excluded region when varying the cross section by
%%      its theoretical uncertainty. The dashed purple lines indicate
%%      the median (thick line) $\pm 1 \sigma$ (thin lines) expected
%%      exclusion regions. 
%%      %The mass ranges considered for models \texttt{T2tt} and
%%      %\texttt{T1tttt} differ from the other models.  
%%    }
%%  \end{center}
%%\end{figure}
%%
%%\begin{figure*}[t!]
%%  \begin{center}
%%    \includegraphics[width=0.45\textwidth]{figures/limits/v4/T2tt_mlsp0_xmin300_smooth5_prelim.pdf} \,
%%    \includegraphics[width=0.45\textwidth]{figures/limits/v4/T2tt_mlsp50_xmin300_smooth5_prelim.pdf}  \\
%%    \includegraphics[width=0.45\textwidth]{figures/limits/v4/T2tt_mlsp100_xmin300_smooth5_prelim.pdf} \,
%%    \includegraphics[width=0.45\textwidth]{figures/limits/v4/T2tt_mlsp150_xmin350_smooth5_prelim.pdf} \\
%%    \caption{\label{fig:t2tt-1d} Excluded cross sections versus top
%%      squark mass $m_{\sTop}$ for the model \texttt{T2tt}, in which
%%      pair-produced top squarks each decay to a top quark and the LSP
%%      with a mass $m_{\rm LSP} = 0\gev$ (top left), $m_{\rm LSP} =
%%      50\gev$ (top right), $m_{\rm LSP} = 100\gev$ (bottom left),
%%      $m_{\rm LSP} = 150\gev$ (bottom right). The observed upper limit
%%      (95\% CL) on the production cross section is shown as a function
%%      of $m_{\sTop}$ (solid line), along with the expected upper limit
%%      and $\pm1\sigma$ experimental uncertainties (long-dashed line
%%      with shaded band), and the NLO+NLL top squark pair-production
%%      cross section and theoretical uncertainties (dotted line with
%%      shaded band).}
%%  \end{center}
%%\end{figure*}
%%
%%Figure~\ref{fig:t2tt-1d} shows the observed upper limit at 95\% CL on
%%the production cross section as a function of the top squark mass
%%($m_{\sTop}$) for the model \texttt{T2tt} when considering different
%%LSP masses in the range 0--150\GeV. No exclusion on possible top
%%squark masses is observed when considering the theoretical production
%%cross section minus $1\sigma$ uncertainty. However, the expected
%%exclusion covers the ranges 300--520\GeV, 320--520\GeV, and
%%420-480\GeV for $m_{\text{LSP}} = 0\GeV$, $m_{\text{LSP}} = 50\GeV$,
%%and $m_{\text{LSP}} = 100\GeV$ respectively. No exclusion is expected
%%for the LSP with a mass greater than 100\GeV.
%%%The expected reach for the T2tt model is summarised
%%%in Table~\ref{tab:sms-reach}. 
