\clearpage
\section{Event Selection\label{sec:eventSelection}}

As discused in~\ref{sec:susy}, all-hadronic SUSY signatures consist of events with
no isolated and detectable objects apart from energetic jets. To search
for an excess in such events, it is convenient to use a jet-based variable,
\scalht, to quantify the energy in an event.  \scalht is defined as the scalar
sum of the transverse energy of the jets in the event. The challanges due to large
backgrounds in a search for an excess in all-hadronic events becomes evident 
when compairing observed data events overlaid with simulated events from 
SM processes as a function of \scalht Figure [REF].
 
The dominant background are azimuthally-balanced multi-jet events, stemming from QCD
processes. This search reduces this background to negligible levels in the
selected region by employing the \alphat variable further discussd in section [REF].
Furthermore, by construction, \alphat also reduces backgrounds from severely mis-measured
jets. 

In absence of the multi-jet background, the remaining significant backgrounds are ...

\subsection{Data}
The data analyzed was recorded by CMS in 2012 between April 5th and Dec. 5th and
totals 19.47$\pm$ 2.6\%. The collected data is certified on a run-by-run 
basis, where initial automatic certification requires the LHC beams to be declared
stable and all CMS subdetectors ON. Further monitoring of the data was done realtime
by experts of each subdetector trough analysis of histograms updated and filled
each lumi section. Final cerfitication was done offline, and lumi sections passing
all criteria were listed in a Golden JSON file to be used by all analyses. 

\subsection{Event quality}

Each event is subjected to a series of commonly used filters in CMS to ensure good
quality data. Minimal requirements are that at least one primary vertex is identified 
and 25\% of the reconstructed tracks to be of good quality. Addionally, various filters 
prescribed by the MET group[REF] are applied. Events containing muons with inconsistent
energy are flagged by the muon pog [REF] and filtered out of the analysis. 

\clearpage
\section{Physics objects\label{sec:reconstruction}}

The definitions of the physics objects used in this analysis follow
the recommendations of the various Physics Object Groups (POGs).

\subsection{Jets}

Jets are reconstructed by combining information from multiple
sub-detectors using the Particle-Flow (PF) algorithm~\cite{PAS-PFT-09-001} 
and clustered by the anti-$k_{\rm T}$ algorithm~\cite{antikt} with
a size parameter of $0.5$. Three levels of jet energy corrections are 
applied; level 1 corrects for overlapping pp collisions 
(pile-up ~\cite{Cacciari2008119,1126-6708-2008-04-005}) in the jet, 
level 2 and 3 correct the jet energy response to be $\eta$ and $\pt$ independent.  
Further residual corrections are applied on data which correct for 
small remaining discripencies in the modelling of the response. The acceptance of
``fake'' jets is supressed by requring that jets pass jet identitifcation at the 
Loose working point~\cite{ref:jet-id}. This requires that the jet is comprised
of more than one particle and that those particles cannot all be neutral hadrons 
(i.e. neutrons) or all neutral particle deposits in the ECAL (i.e. photons).
Additionaly if the jet is reconstructed outside outside the tracker's 
instrumatation, the jet is required to have more than one charged 
constituent, of which not all of them deposit their energy in the ECAL. 
Only jets reconstructed within the calorimeters' barrel and endcap, 
i.e. $|\eta| <$ 3.0, and with transverse momentum $\pt >$ 50\gev  
are considered in the analysis. Jets originating from bottom quarks 
(b-jets) are identified through vertices that are displaced with respect to the primary 
interaction~\cite{CMS-PAS-BTV-12-001}. The algorithm used to tag b-jets 
is the Combined Secondary Vertex tagger, using the "Medium" working point, 
which is achieved by requiring a cut of $>$0.679 on the algorithm discriminator 
variable and results in a gluon/light-quark quark mis-tag rate of 1\% 
(where ``light'' means $u$, $d$ and $s$ quarks) and an efficiency in the 
range $60-70\%$ depending on the jet \pt. 

\subsection{Muons}

Muons are identified according to the Tight working point definition
($\sim$95\% efficiency) of the muon identification
algorithm~\cite{ref:muon-id}. The algorithm works to reject cosmic muons or 
muons from decays in flight for consideration in the analysis. 
A PF-based ``combined relative'' isolation~\cite{ref:muon-id} is determined 
within a cone size $\Delta R < 0.4$, and "$\Delta\beta$" corrections 
are applied to remove the effects of pileup. Table~\ref{tab:muon-id} summarizes the
identification and isolation requirements. 
%This object is used as a
%veto as part of the hadronic signal region definition, as described in
%Section~\ref{sec:vetoes}, and as part of the \mj control
%sample selection described in Section~\ref{sec:def-control-samples}.

\begin{table}[h!]
  \caption{Muon identification (Tight working point).\label{tab:muon-id}}
  \centering
  \footnotesize
  \begin{tabular}{ lc }
    \hline
    \hline
    Global Muon                            & True      \\
    PFMuon                                 & True      \\
    $\chi^{2}$ fit                         & $<10$     \\
    Muon chamber hits                      & $>0$      \\
    Muon station hits                      & $>1$      \\
    Transverse impact $d_{xy}$             & $<0.2\mm$ \\
    Longitudinal dist $d_{z}$              & $<0.5\mm$ \\
    Pixel hits                             & $>0$      \\
    Track layer hits                       & $>5$      \\
    PF Isolation ($\Delta\beta$ corrected) & $<0.12$   \\
    \hline
    \hline
  \end{tabular}
\end{table}

\subsection{Photons}
%
Selected photons must satisfy the Tight working point definition 
($\sim$70\% efficiency) of the simple cut-based photon identification
algorithm~\cite{ref:photon-id-egamma}. Pile-up corrected isolation is 
determined within a cone size $\Delta R < 0.3$ using the PF-based 
isolation algorithm~\cite{ref:photon-id-egamma}. 
Table~\ref{tab:photon-id-egamma} summarises the identification
and isolation requirements. 

\begin{table}[ht!]
  \caption{Photon identification (Tight working point).\label{tab:photon-id-egamma}}
  \centering
  \footnotesize
  \begin{tabular}{ ccc }
    \hline
    \hline
    Categories                    & Barrel                             & EndCap                             \\
    \hline
    Conversion safe electron veto & Yes                                & Yes                                \\
    Single Tower H/E              & 0.05                               & 0.05                               \\
    $\sigma_{i\eta i\eta}$        & 0.11                               & 0.31                               \\
    PF charged hadron isolation   & 0.70                               & 0.50                               \\
    PF neutral hadron isolation   & 0.4 + 0.04 $\times$ $\pt^{\gamma}$  & 1.5 + 0.04 $\times$ $\pt^{\gamma}$  \\
    PF photon isolation           & 0.5 + 0.005 $\times$ $\pt^{\gamma}$ & 1.0 + 0.005 $\times$ $\pt^{\gamma}$ \\
    \hline
    \hline
  \end{tabular}
\end{table}

\subsection{Electrons}

Electrons are identified according to the Veto working point
definition ($\sim$95\% efficiency) of the cut-based \verb!Egamma!
identification algorithm~\cite{ref:electron-id}. PF-based
isolation~\cite{ref:electron-isolation} is determined within a cone
size $\Delta R < 0.3$ and $\rho \times A_{\textrm{eff}}$ corrections
are applied to remove the effects of pileup. Table~\ref{tab:ele-id}
summarises the identification and isolation requirements. 
%This object
%is used as a veto as part of the hadronic signal region definition, as
%described in Section~\ref{sec:vetoes}.

\begin{table}[h!]
  \caption{Electron identification (Loose working point).\label{tab:ele-id}}
  \centering
  \footnotesize
  \begin{tabular}{ lcc }
    \hline
    \hline
    Categories                                               & Barrel    & EndCap    \\
    \hline
    $\Delta \eta_{In}$                                       & 0.007     & 0.009     \\
    $\Delta \phi_{In}$                                       & 0.15      & 0.10      \\
    $\sigma_{i\eta i\eta}$                                   & 0.01      & 0.03      \\
    H/E                                                      & 0.12      & 0.10      \\
    d0 (vtx)                                                 & 0.02      & 0.02      \\
    dZ (vtx)                                                 & 0.2       & 0.20      \\
    $\lvert(1/E_{\textrm{ECAL}} - 1/p_{\textrm{trk}})\rvert$ & 0.05      & 0.05      \\
    PF relative isolation                                    & 0.15      & 0.15      \\
    Vertex fit probability                                   & 10$^{-6}$ & 10$^{-6}$ \\
    Missing hits                                             & 1         & 1         \\
    \hline
    \hline
  \end{tabular}
\end{table}

%\subsection{Single isolated tracks}
%W boson can be identified if they
%Leptonic decay products of W bosons will leave one single isolated track (SIT) 
%in the tracker. A single isolated track (SIT) can be used to identify W bosons through
%their leptonic decays: $\textrm{W} \ra \mu \nu$, $\textrm{W} \ra e
%\nu$, and $\textrm{W} \ra \tau (\ra \ell) \nu$. Also, single prong
%decays of the tau lepton can be identified: $\textrm{W} \ra \tau (\ra
%h^{\pm} + n\pi^{0})\nu$. A single isolated track comprises a charged
%PF candidate that satisfies the requirements listed in
%Table~\ref{tab:sit-id}. The relative track isolation is determined
%from the vectorial sum of neighbouring charged PF candidates within a
%cone $\Delta R < 0.3$ and satisfying $\Delta z (\textrm{candidate,PV})
%< 0.05\cm$ around the candidate isolated track. This object can be
%used to efficiently suppress the ``lost lepton'' background from W and
%\ttbar, as described in Section~\ref{sec:vetoes}. This definition is
%based on the one used in SUS-13-011 (``single lepton stop search with
%transverse mass'')~\cite{singleleptonstop}.
%
%\begin{table}[h!]
%  \caption{Single isolated track identification.\label{tab:sit-id}}
%  \centering
%  \footnotesize
%  \begin{tabular}{ lc }
%    \hline
%    \hline
%    Track \Pt                      & $>10\gev$  \\
%    $\Delta z (\textrm{track,PV})$ & $<0.05\cm$ \\
%    Charge                         & $\neq 0$   \\
%    Relative track isolation       & $<0.1$     \\
%    \hline
%    \hline
%  \end{tabular}
%\end{table}
%
%\subsection{Missing transverse energy}
%
%Missing transverse energy, \met, is defined by the type-I corrected
%particle-flow (PF)-based MET algorithm~\cite{ref:MET-corrections}. The
%\met variable is only used in the following two cases: to define of
%the tranverse mass variable, \mt, which is in turn used as part of the
%selection criteria that define the \mj control sample, described in
%Section~\ref{sec:def-control-samples}; to define a cleaning filter
%applied after the \alphat requirement, as described in
%Section~\ref{sec:had-signal}.


\subsection{Triggers}

\subsubsection{Hadronic search region and control samples\label{sec:signal_triggers}} 

Only events passing one or more HLT triggers based on online quantities 
are recorded to be analyzed. In any analysis, it is generally not expected 
that all recorded events reconstructed offline pass the online 
trigger as detector conditions, energy corrections, and object-based quantities
differ offline. In this analysis, triggers at the HLT
based on quantities \scalht and \alphat (labeled as \verb!HTxxx_AlphaT0pyy!) 
are used with various thresholds to record candidate events for the hadronic search
region. In order to keep the computational time low, the online quantities
are constructed using calorimeter based jets (calo jets). Additionally, the use of
particle-flow jets in this analysis is expected to introduce inefficiencies.
Each \scalht bin is seeded by a single trigger chosen based on the
efficiency of the trigger in that \scalht bin. The \alphat thresholds of the
\verb!HTxxx_AlphaT0pyy! triggers were tuned according to the threshold
on the \scalht leg in order to fully suppress QCD multijet events~\cite{RA1Paper2012}
and simultaneously satisfying other criteria, such as maintaining
acceptable trigger rates.

The \verb!HTxxx_AlphaT0pyy! trigger efficiencies are measured with a
reference (\ie, unbiased) event sample recorded by an unprescaled,
loosely-isolated, eta-restricted single muon trigger,$\,$      $\verb!HLT_IsoMu24_eta2p1!$,
within the \verb!SingleMu! dataset. This sample contains events with  at 
least one isolated muon with $\pt > 25\gev$ and $|\eta| < 2.1$
(similar to the \mj control sample defined in 
Section~\ref{sec:def-control-samples}). Muons are required to be isolated: a cut of $\Delta
{\rm R} > 0.5$ is placed between all muons and jets in each event, and
only jets are considered in the calculation of \scalht, and
\alphat, \ie the muon is ignored.

\begin{table}[!h]
  \caption{List of signal triggers and their efficiencies (\%), as
    measured in the SingleMu data. The trigger efficiency is $\sim$100\% for all
    bins above $\scalht > 675\gev$.}  
  \label{tab:htalphat-triggers}
  \centering
  \footnotesize
  \begin{tabular}{ cccccc }
    \hline
    \hline
    Offline \scalht       & Offline \alphat & L1 seed (\verb!L1!)         & Trigger (\verb!HLT!)  & \multicolumn{2}{c}{Efficiency (\%)}          \\ [0.5ex]
    bin (\gev)         & threshold       & (highest thresholds)          &                         & $2 \leq \njet \leq 3$ & $\njet \geq 4$       \\ [0.5ex]
    \hline
    %$200 < \scalht < 275$ & 0.65            & \verb!DoubleJetC64!           & \verb!HT200_AlphaT0p57! & $81.8^{+0.4}_{-0.4}$  & $78.9^{+0.3}_{-0.4}$ \\
    %$275 < \scalht < 325$ & 0.60            & \verb!DoubleJetC64!           & \verb!HT200_AlphaT0p57! & $95.2^{+0.3}_{-0.4}$  & $90.0^{+1.2}_{-1.3}$ \\
    %$325 < \scalht < 375$ & 0.55            & \verb!DoubleJetC64 OR HTT175! & \verb!HT300_AlphaT0p53! & $97.9^{+0.3}_{-0.3}$  & $95.6^{+0.9}_{-1.0}$ \\
    $375 < \scalht < 475$ & 0.55            & \verb!DoubleJetC64 OR HTT175! & \verb!HT300_AlphaT0p53! & $94.2^{+0.5}_{-0.6}$  & $90.5^{+1.2}_{-1.3}$ \\
    $475 < \scalht < 575$ & 0.55            & \verb!DoubleJetC64 OR HTT175! & \verb!HT350_AlphaT0p52! & $96.2^{+0.8}_{-0.9}$  & $94.6^{+1.2}_{-1.4}$ \\
    $575 < \scalht < 675$ & 0.55            & \verb!DoubleJetC64 OR HTT175! & \verb!HT400_AlphaT0p51! & $95.4^{+1.4}_{-1.8}$  & $98.7^{+0.7}_{-1.12}$ \\
    $\scalht > 675$       & 0.55            & \verb!DoubleJetC64 OR HTT175! & \verb!HT400_AlphaT0p51! & $100^{+0.0}_{-2.0}$  & $100^{+0.0}_{-2.0}$ \\
    \hline
    \hline
  \end{tabular}
\end{table}


Table~\ref{tab:htalphat-triggers} summarizes the measured efficiencies
for the \verb!HTxxx_AlphaT0pyy! triggers in the relevant \scalht
bins. The trigger efficiencies are measured for both \njet
multiplicity bins. %The efficiencies are generally observed to be high,
%$\sim$100\%, except for the low \scalht region $200 < \scalht <
%325\gev$. The inefficiencies at low \scalht are mainly due to the L1
%seeds for which thresholds were raised to a relatively high level in
%order to maintain trigger rates in the high-pileup conditions towards
%the end of Run 1. 
The efficiencies are slightly lower in the higher
jet multiplicity category due to a larger number of jets summing to
the same \scalht, resulting in softer jets. 
Figure~\ref{fig:eff-alphat-le3j} shows the efficiency curves 
for the \verb!HTxxx_AlphaT0pyy! triggers in the three lowest \scalht bins, 
for the \njetlow and \njethigh categories.

\begin{figure}[!h]
  \begin{center}
    \subfigure[\njetlow, $375 < \scalht < 475 \gev$]{
      \includegraphics[width=0.4\textwidth,page=30]{figures/trigger/plotDump/v29/HT375_475_100_100_50_AlphaT_HT300xaT0p53_PF_le3j_RunAtFNAL.pdf}
    }
    \subfigure[\njethigh, $375 < \scalht < 475 \gev$]{
      \includegraphics[width=0.4\textwidth,page=30]{figures/trigger/plotDump/v29/HT375_475_100_100_50_AlphaT_HT300xaT0p53_PF_ge4j_RunAtFNAL.pdf}
    } \\
    \subfigure[\njetlow, $475 < \scalht < 525 \gev$]{
      \includegraphics[width=0.4\textwidth,page=30]{figures/trigger/plotDump/v29/HT475_575_100_100_50_AlphaT_HT350xaT0p52_PF_le3j_RunAtFNAL.pdf}
    } 
    \subfigure[\njethigh, $475 < \scalht < 525 \gev$]{
      \includegraphics[width=0.4\textwidth,page=30]{figures/trigger/plotDump/v29/HT475_575_100_100_50_AlphaT_HT350xaT0p52_PF_ge4j_RunAtFNAL.pdf}
    } \\
    \subfigure[\njetlow, $525 < \scalht < 675 \gev$]{
      \includegraphics[width=0.4\textwidth,page=27]{figures/trigger/plotDump/v29/HT575_675_100_100_50_AlphaT_HT400xaT0p51_PF_le3j_RunAtFNAL.pdf}
    }
    \subfigure[\njethigh, $525 < \scalht < 675 \gev$]{
      \includegraphics[width=0.4\textwidth,page=27]{figures/trigger/plotDump/v29/HT575_675_100_100_50_AlphaT_HT400xaT0p51_PF_ge4j_RunAtFNAL.pdf}
    } \\
    \caption{\label{fig:eff-alphat-le3j}
      Cumulative efficiency turn-on curves for the \scalht-\alphat 
      cross triggers (as summarized in Table~\ref{tab:htalphat-triggers}) 
      that record events for the three lowest \scalht bins for events 
      satisfying \njetlow (left) and \njethigh (right). 
    }
  \end{center}
\end{figure}
%
%\begin{figure}[!h]
%  \begin{center}
%    \subfigure[Differential, $375 < \scalht < 475 \gev$]{
%      \includegraphics[width=0.4\textwidth,page=20]{figures/trigger/plotDump/v29/HT375_475_100_100_50_AlphaT_HT300xaT0p53_PF_le3j_RunAtFNAL.pdf}
%    }
%    \subfigure[Cumulative, $375 < \scalht < 475 \gev$]{
%      \includegraphics[width=0.4\textwidth,page=30]{figures/trigger/plotDump/v29/HT375_475_100_100_50_AlphaT_HT300xaT0p53_PF_le3j_RunAtFNAL.pdf}
%    } \\
%    \subfigure[Differential, $475 < \scalht < 525 \gev$]{
%      \includegraphics[width=0.4\textwidth,page=20]{figures/trigger/plotDump/v29/HT475_575_100_100_50_AlphaT_HT350xaT0p52_PF_le3j_RunAtFNAL.pdf}
%    } 
%    \subfigure[Cumulative, $475 < \scalht < 525 \gev$]{
%      \includegraphics[width=0.4\textwidth,page=30]{figures/trigger/plotDump/v29/HT475_575_100_100_50_AlphaT_HT350xaT0p52_PF_le3j_RunAtFNAL.pdf}
%    } \\
%    \subfigure[Differential, $525 < \scalht < 675 \gev$]{
%      \includegraphics[width=0.4\textwidth,page=18]{figures/trigger/plotDump/v29/HT575_675_100_100_50_AlphaT_HT400xaT0p51_PF_le3j_RunAtFNAL.pdf}
%    }
%    \subfigure[Cumulative, $525 < \scalht < 675 \gev$]{
%      \includegraphics[width=0.4\textwidth,page=27]{figures/trigger/plotDump/v29/HT575_675_100_100_50_AlphaT_HT400xaT0p51_PF_le3j_RunAtFNAL.pdf}
%    } \\
%    \caption{\label{fig:eff-alphat-le3j}
%      (Left) Differential and (Right) cumulative efficiency turn-on 
%      curves for the \scalht-\alphat cross triggers (as summarised in 
%      Table~\ref{tab:htalphat-triggers}) that record events for the
%      three lowest \scalht bins  for events satisfying \njetlow. 
%    }
%  \end{center}
%\end{figure}

%begin{figure}[!h]
% \begin{center}
%   \subfigure[Differential, $375 < \scalht < 475 \gev$]{
%     \includegraphics[width=0.4\textwidth,page=20]{figures/trigger/plotDump/v29/HT375_475_100_100_50_AlphaT_HT300xaT0p53_PF_ge4j_RunAtFNAL.pdf}
%   }
%   \subfigure[Cumulative, $375 < \scalht < 475 \gev$]{
%     \includegraphics[width=0.4\textwidth,page=30]{figures/trigger/plotDump/v29/HT375_475_100_100_50_AlphaT_HT300xaT0p53_PF_ge4j_RunAtFNAL.pdf}
%   } \\
%   \subfigure[Differential, $475 < \scalht < 525 \gev$]{
%     \includegraphics[width=0.4\textwidth,page=20]{figures/trigger/plotDump/v29/HT475_575_100_100_50_AlphaT_HT350xaT0p52_PF_ge4j_RunAtFNAL.pdf}
%   } 
%   \subfigure[Cumulative, $475 < \scalht < 525 \gev$]{
%     \includegraphics[width=0.4\textwidth,page=30]{figures/trigger/plotDump/v29/HT475_575_100_100_50_AlphaT_HT350xaT0p52_PF_ge4j_RunAtFNAL.pdf}
%   } \\
%   \subfigure[Differential, $525 < \scalht < 675 \gev$]{
%     \includegraphics[width=0.4\textwidth,page=18]{figures/trigger/plotDump/v29/HT575_675_100_100_50_AlphaT_HT400xaT0p51_PF_ge4j_RunAtFNAL.pdf}
%   }
%   \subfigure[Cumulative, $525 < \scalht < 675 \gev$]{
%     \includegraphics[width=0.4\textwidth,page=27]{figures/trigger/plotDump/v29/HT575_675_100_100_50_AlphaT_HT400xaT0p51_PF_ge4j_RunAtFNAL.pdf}
%   } \\
%   \caption{\label{fig:eff-alphat-ge4j}
%     (Left) Differential and (Right) cumulative efficiency turn-on 
%     curves for the \scalht-\alphat cross triggers (as summarised in 
%     Table~\ref{tab:htalphat-triggers}) that record events for the
%     three lowest \scalht bins  for events satisfying \njethigh. 
%   }
% \end{center}
%end{figure}
\FloatBarrier



































































































%\clearpage
%\section{Triggers\label{sec:triggers}} 
%
%\subsection{Hadronic signal region\label{sec:signal_triggers}} 
%
%%Cross triggers at the HLT based on the quantities \scalht and \alphat
%%(labelled as \verb!HTxxx_AlphaT0pyy!) are used with various thresholds
%%to record candidate events for the hadronic signal region. Only a
%%single trigger is used to seed each \scalht bin of the signal region,
%%based on the \scalht threshold. The \alphat thresholds of the
%%\verb!HTxxx_AlphaT0pyy! triggers are tuned according to the threshold
%%on the \scalht leg in order to fully suppress QCD multijet events
%%(whilst simultaneously satisfying other criteria, such as maintaining
%%acceptable trigger rates).
%%
%%The \verb!HT200_Alphat0p57! trigger recorded candidate signal events
%%in the \verb!ParkedHTMHT! dataset, which was introduced at the
%%beginning of Run B in 2012. All other triggers were available in
%%Stream A from the start of Run 1 and seed the \verb!HTMHT! dataset.
%%
%%Table~\ref{tab:htalphat-triggers} summarises the thresholds used for
%%the \verb!HTxxx_AlphaT0pyy! triggers. Also listed are the Level-1
%%trigger seeds and thresholds. \verb!FastJet! corrections are applied
%%to the input jets for all HLT triggers. To ensure that the \scalht leg
%%of each \verb!HTxxx_AlphaT0pyy! trigger is efficient for the signal
%%region selection criteria, the lower bounds of the offline \scalht
%%bins are offset by 25\GeV with respect to the trigger \scalht
%%thresholds.
%%
%%The \verb!HTxxx_AlphaT0pyy! trigger efficiencies are measured with a
%%reference (\ie, unbiased) event sample recorded by an unprescaled,
%%loosely-isolated, eta-restricted single muon trigger,
%%\verb!HLT_IsoMu24_eta2p1!, within the \verb!SingleMu! dataset. A
%%sample of events containing at least one isolated muon with $\pt >
%%25\gev$ and $|\eta| < 2.1$ is used (similar to the \mj control sample
%%defined in Section~\ref{sec:def-control-samples}). A cut of $\Delta
%%{\rm R} > 0.5$ is placed between all muons and jets in each event, and
%%only jets are considered in the calculation of \scalht, \mht, and
%%\alphat, \ie the muon is ignored.
%%
%%Table~\ref{tab:htalphat-triggers} summarises the measured efficiencies
%%for the \verb!HTxxx_AlphaT0pyy! triggers in the relevant \scalht
%%bins. The trigger efficiencies are measured for both \njet
%%multiplicity bins. The efficiencies are generally observed to be high,
%%$\sim$100\%, except for the low \scalht region $200 < \scalht <
%%325\gev$. The inefficiencies at low \scalht are mainly due to the L1
%%seeds for which thresholds were raised to a relatively high level in
%%order to maintain trigger rates in the high-pileup conditions towards
%%the end of Run 1. The inefficiencies are slightly larger in the higher
%%jet multiplicity category due to a larger number of jets summing to
%%the same \scalht, resulting in softer
%%jets. Figures~\ref{fig:eff-alphat-le3j} and~\ref{fig:eff-alphat-ge4j}
%%show the efficiency curves for the \verb!HTxxx_AlphaT0pyy! triggers in
%%the three lowest \scalht bins, for the \njetlow and \njethigh
%%categories, respectively. The efficiencies are also determined at the
%%level of event categories and no dependence on \nb is observed, with
%%efficiencies agreeing within statistical uncertainties.
%%
%%%\fixme{Various cross-checks were performed, by comparing efficiencies
%%%  measured from different sub-samples, such as: exactly one isolated
%%%  muon versus at least two isolated muons (W-enriched versus
%%%  DY-enriched); exactly zero b-tagged jets versus at least one
%%%  b-tagged jet (W-enriched versus \ttbar-enriched); the nominal choice
%%%  of $\mht/\met < 1.25$ versus one in which this criterion is
%%%  inverted\footnote{This cross-check concerns the QCD multijet
%%%    background estimation method, that relies on a $\mht/met$ data
%%%    sideband, as described in Section~\ref{sec:qcd}} ($\mht/\met >
%%%  1.25$). For all cross-checks, the efficiencies were agreed within
%%%  statistical uncertainties for \scalht and \alphat values above the
%%%  trigger thresholds.}
%%
%%\begin{table}[!h]
%%  \caption{List of signal triggers and their efficiencies (\%), as
%%    measured in data. The trigger efficiency is $\sim$100\% for all
%%    bins above $\scalht > 475\gev$.}  
%%  \label{tab:htalphat-triggers}
%%  \centering
%%  \footnotesize
%%  \begin{tabular}{ cccccc }
%%    \hline
%%    \hline
%%    Offline \scalht       & Offline \alphat & L1 seed (\verb!L1_?!)         & Trigger (\verb!HLT_?!)  & \multicolumn{2}{c}{Efficiency (\%)}          \\ [0.5ex]
%%    region (\gev)         & threshold       & (highest thresholds)          &                         & $2 \leq \njet \leq 3$ & $\njet \geq 4$       \\ [0.5ex]
%%    \hline
%%    $200 < \scalht < 275$ & 0.65            & \verb!DoubleJetC64!           & \verb!HT200_AlphaT0p57! & $81.8^{+0.4}_{-0.4}$  & $78.9^{+0.3}_{-0.4}$ \\
%%    $275 < \scalht < 325$ & 0.60            & \verb!DoubleJetC64!           & \verb!HT200_AlphaT0p57! & $95.2^{+0.3}_{-0.4}$  & $90.0^{+1.2}_{-1.3}$ \\
%%    $325 < \scalht < 375$ & 0.55            & \verb!DoubleJetC64 OR HTT175! & \verb!HT300_AlphaT0p53! & $97.9^{+0.3}_{-0.3}$  & $95.6^{+0.9}_{-1.0}$ \\
%%    $375 < \scalht < 475$ & 0.55            & \verb!DoubleJetC64 OR HTT175! & \verb!HT350_AlphaT0p52! & $99.2^{+0.2}_{-0.2}$  & $98.7^{+0.5}_{-0.7}$ \\
%%    $\scalht > 475$       & 0.55            & \verb!DoubleJetC64 OR HTT175! & \verb!HT400_AlphaT0p51! & $99.8^{+0.1}_{-0.3}$  & $99.6^{+0.3}_{-0.7}$ \\
%%    \hline
%%    \hline
%%  \end{tabular}
%%\end{table}
%%
%%%\clearpage
%%%\begin{figure}[!h]
%%%  \begin{center}
%%%    \subfigure[$\njetlow,   275 < \scalht < 325 \gev$]{
%%%      \includegraphics[width=0.4\textwidth]{figures/cms}
%%%    }
%%%    \subfigure[$\njethigh,   275 < \scalht < 325 \gev$]{
%%%      \includegraphics[width=0.4\textwidth]{figures/cms}
%%%    } \\
%%%    \subfigure[$\njetlow,   325 < \scalht < 375 \gev$]{
%%%      \includegraphics[width=0.4\textwidth]{figures/cms}
%%%    } 
%%%    \subfigure[$\njethigh,   325 < \scalht < 375 \gev$]{
%%%      \includegraphics[width=0.4\textwidth]{figures/cms}
%%%    } \\
%%%    \subfigure[$\njetlow,   375 < \scalht < 475 \gev$]{
%%%      \includegraphics[width=0.4\textwidth]{figures/cms}
%%%    }
%%%    \subfigure[$\njethigh,   375 < \scalht < 475 \gev$]{
%%%      \includegraphics[width=0.4\textwidth]{figures/cms}
%%%    } \\
%%%    \caption{\label{fig:eff-alphat-diff}Efficiency turn-on
%%%      curve for the \scalht-\alphat cross trigger and \scalht bin
%%%      (paired as defined in Table~\ref{tab:htalphat-triggers}) for
%%%      events satisfying (left)
%%%      \njetlow and (right) \njethigh. }
%%%  \end{center}
%%%\end{figure}
%%
%\begin{figure}[!h]
%  \begin{center}
%    \subfigure[Differential, $375 < \scalht < 475 \gev$]{
%      \includegraphics[width=0.4\textwidth,page=20]{figures/trigger/plotDump/v28/HT375_475_100_100_50_AlphaT_HT250xaT0p55_PF_le3j_RunAtFNAL.pdf}
%    }
%    \subfigure[Cumulative, $375 < \scalht < 475 \gev$]{
%      \includegraphics[width=0.4\textwidth,page=30]{figures/trigger/plotDump/v28/HT375_475_100_100_50_AlphaT_HT250xaT0p55_PF_le3j_RunAtFNAL.pdf}
%    } \\
%    \subfigure[Differential, $475 < \scalht < 525 \gev$]{
%      \includegraphics[width=0.4\textwidth,page=20]{figures/trigger/plotDump/v28/HT475_575_100_100_50_AlphaT_HT300xaT0p53_PF_le3j_RunAtFNAL.pdf}
%    } 
%    \subfigure[Cumulative, $475 < \scalht < 525 \gev$]{
%      \includegraphics[width=0.4\textwidth,page=30]{figures/trigger/plotDump/v28/HT475_575_100_100_50_AlphaT_HT300xaT0p53_PF_le3j_RunAtFNAL.pdf}
%    } \\
%    \subfigure[Differential, $525 < \scalht < 675 \gev$]{
%      \includegraphics[width=0.4\textwidth,page=20]{figures/trigger/plotDump/v28/HT575_675_100_100_50_AlphaT_HT350xaT0p52_PF_le3j_RunAtFNAL.pdf}
%    }
%    \subfigure[Cumulative, $525 < \scalht < 675 \gev$]{
%      \includegraphics[width=0.4\textwidth,page=30]{figures/trigger/plotDump/v28/HT575_675_100_100_50_AlphaT_HT350xaT0p52_PF_le3j_RunAtFNAL.pdf}
%    } \\
%    \caption{\label{fig:eff-alphat-le3j}
%      (Left) Differential and (Right) cumulative efficiency turn-on 
%      curves for the \scalht-\alphat cross triggers (as summarised in 
%      Table~\ref{tab:htalphat-triggers}) that record events for the
%      three lowest \scalht bins  for events satisfying \njetlow. 
%    }
%  \end{center}
%\end{figure}
%%
%%\begin{figure}[!h]
%%  \begin{center}
%%    \subfigure[Differential, $200 < \scalht < 275 \gev$]{
%%      \includegraphics[width=0.4\textwidth,page=11]{figures/trigger/HT200_275_73_73_36_AlphaT_ge4j_RunAtFNAL}
%%    }
%%    \subfigure[Cumulative, $200 < \scalht < 275 \gev$]{
%%      \includegraphics[width=0.4\textwidth,page=18]{figures/trigger/HT200_275_73_73_36_AlphaT_ge4j_RunAtFNAL}
%%    } \\
%%    \subfigure[Differential, $275 < \scalht < 325 \gev$]{
%%      \includegraphics[width=0.4\textwidth,page=11]{figures/trigger/HT275_325_73_73_36_AlphaT_ge4j_RunAtFNAL}
%%    } 
%%    \subfigure[Cumulative, $275 < \scalht < 325 \gev$]{
%%      \includegraphics[width=0.4\textwidth,page=18]{figures/trigger/HT275_325_73_73_36_AlphaT_ge4j_RunAtFNAL}
%%    } \\
%%    \subfigure[Differential, $325 < \scalht < 375 \gev$]{
%%      \includegraphics[width=0.4\textwidth,page=11]{figures/trigger/HT325_375_86_86_43_AlphaT_ge4j_RunAtFNAL}
%%    }
%%    \subfigure[Cumulative, $325 < \scalht < 375 \gev$]{
%%      \includegraphics[width=0.4\textwidth,page=18]{figures/trigger/HT325_375_86_86_43_AlphaT_ge4j_RunAtFNAL}
%%    } \\
%%    \caption{\label{fig:eff-alphat-ge4j}
%%      (Left) Differential and (Right) cumulative efficiency turn-on 
%%      curves for the \scalht-\alphat cross triggers (as summarised in 
%%      Table~\ref{tab:htalphat-triggers}) that record events for the
%%      three lowest \scalht bins  for events satisfying \njethigh. 
%%    }
%%  \end{center}
%%\end{figure}
%%
%%%\begin{figure}[!h]
%%%  \begin{center}
%%%    \subfigure[$\njetlow,   275 < \scalht < 325 \gev$]{
%%%      \includegraphics[width=0.4\textwidth]{figures/cms}
%%%    }
%%%    \subfigure[$\njetlow,   275 < \scalht < 325 \gev$]{
%%%      \includegraphics[width=0.4\textwidth]{figures/cms}
%%%    } \\
%%%    \subfigure[$\njetlow,   325 < \scalht < 375 \gev$]{
%%%      \includegraphics[width=0.4\textwidth]{figures/cms}
%%%    } 
%%%    \subfigure[$\njetlow,   325 < \scalht < 375 \gev$]{
%%%      \includegraphics[width=0.4\textwidth]{figures/cms}
%%%    } \\
%%%    \subfigure[$\njetlow,   375 < \scalht < 475 \gev$]{
%%%      \includegraphics[width=0.4\textwidth]{figures/cms}
%%%    }
%%%    \subfigure[$\njetlow,   375 < \scalht < 475 \gev$]{
%%%      \includegraphics[width=0.4\textwidth]{figures/cms}
%%%    } \\
%%%    \caption{\label{fig:eff-alphat-diff-le3j}Left: distribution of the
%%%      \alphat variable in a given \scalht bin after applying the event
%%%      selection described in the text, for events collected with the
%%%      \texttt{IsoMu24\_eta2p1} trigger (red histogram) and the
%%%      \scalht-\alphat cross-trigger (black histogram). The
%%%      \scalht-\alphat cross-trigger used for each \scalht bin is
%%%      defined in Table~\ref{tab:htalphat-triggers}. Right: resulting
%%%      efficiency turn-on curve for the given \scalht-\alphat
%%%      cross-trigger and \scalht bin. All plots are constructed for the
%%%      \njetlow multiplicity bin.}
%%%  \end{center}
%%%\end{figure}
%%
%%%\begin{figure}[!h]
%%%  \begin{center}
%%%    \subfigure[$\njethigh,   275 < \scalht < 325 \gev$]{
%%%      \includegraphics[width=0.4\textwidth]{figures/cms}
%%%    }
%%%    \subfigure[$\njethigh,   275 < \scalht < 325 \gev$]{
%%%      \includegraphics[width=0.4\textwidth]{figures/cms}
%%%    } \\
%%%    \subfigure[$\njethigh,   325 < \scalht < 375 \gev$]{
%%%      \includegraphics[width=0.4\textwidth]{figures/cms}
%%%    } 
%%%    \subfigure[$\njethigh,   325 < \scalht < 375 \gev$]{
%%%      \includegraphics[width=0.4\textwidth]{figures/cms}
%%%    } \\
%%%    \subfigure[$\njethigh,   375 < \scalht < 475 \gev$]{
%%%      \includegraphics[width=0.4\textwidth]{figures/cms}
%%%    }
%%%    \subfigure[$\njethigh,   375 < \scalht < 475 \gev$]{
%%%      \includegraphics[width=0.4\textwidth]{figures/cms}
%%%    } \\
%%%    \caption{\label{fig:eff-alphat-diff-ge4j}Left: distribution of the
%%%      \alphat variable in a given \scalht bin after applying the event
%%%      selection described in the text, for events collected with the
%%%      \texttt{IsoMu24\_eta2p1} trigger (red histogram) and the
%%%      \scalht-\alphat cross-trigger (black histogram). The
%%%      \scalht-\alphat cross-trigger used for each \scalht bin is
%%%      defined in Table~\ref{tab:htalphat-triggers}. Right: resulting
%%%      efficiency turn-on curve for the given \scalht-\alphat
%%%      cross-trigger and \scalht bin. All plots are constructed for the
%%%      \njethigh multiplicity bin.}
%%%  \end{center}
%%%\end{figure}
%
%%\subsection{Hadronic control sample\label{sec:had_control_triggers}} 
%%
%%Prescaled \scalht triggers, labelled henceforth as \verb!HTxxx!, are
%%used with various thresholds to record events for the hadronic control
%%region.
%%Only a single trigger is used to seed each \scalht bin of the hadronic
%%control sample, based on the \scalht threshold, \ie the same approach
%%as used by the signal triggers. As a consequence, the \scalht
%%thresholds of 200, 250, 300, 350, and 400\gev for the \verb!HTxxx! and
%%\verb!HTxxx_AlphaT0pyy! triggers generally match for a given \scalht
%%bin in the signal region and hadronic control sample.
%%
%%Table~\ref{tab:ht-triggers} summarises the thresholds used for both
%%the \verb!HTxxx! triggers, respectively. Also listed are the Level-1
%%seeds, which are identical to the ones used for the signal triggers,
%%and the typical (HLT) prescales value used towards the end of Run 1
%%that control the rates to the level of $\sim$1~Hz per trigger. As in
%%the case of the signal triggers, the lower bounds of the \scalht bins
%%in the hadronic control sample are shifted by 25\gev with respect to
%%the \verb!HTxxx! trigger thresholds.
%%
%%As for the signal triggers, the \verb!HTxxx! trigger efficiencies are
%%measured using the same reference event sample recorded by the
%%\verb!HLT_IsoMu24_eta2p1! trigger. The same offline selection criteria
%%are applied to the sample of events containing at least one isolated
%%muon and jets. The muon(s) is(are) ignored in the calculation of
%%\scalht. 
%%
%%A second reference sample of events, recorded with the
%%\verb!HLT_Physics! minimum bias trigger, was used to cross-check the
%%efficiency measurements from the \mj sample. Both samples are expected
%%to provide unbiased measurements of the \verb!HTxxx! trigger
%%efficiencies and indeed the resulting measurements agree within
%%statistical uncertainties. Due to the limited size of both reference
%%samples, a weighted mean of the measured efficiencies is determined.
%%Table~\ref{tab:ht-triggers} summarises these efficiencies for the
%%\verb!HTxxx! triggers for the relevant bins in \scalht. Again, the
%%trigger efficiencies are measured for the two exclusive \njet
%%multiplicity bins.
%%%Similarly, Figs.~\ref{fig:eff-alphat-diff-le3j}
%%%and~\ref{fig:eff-ht-diff-ge4j} show the efficiency curves for the
%%%(prescaled) \verb!HTxxx! triggers in the three lowest \scalht bins,
%%%for the \njetlow and \njethigh categories, respectively. 
%%
%%For the \verb!HTxxx! triggers, a (non-deterministic) prescale is
%%applied at the HLT to maintain rate. The trigger logic only accepts
%%1$/P$ triggered events, where $P$ is the prescale value. This
%%(non-deterministic) logic introduces a random element that allows
%%measurements of the trigger efficiency to fluctuate above 100\% while
%%being statistically compatible with 100\%. Any meaurement will tend to
%%100\% eith increasing sample size, and cannot fluctuate above 100\%
%%for an unprescaled \verb!HTxxx! trigger. The measurements in
%%Table~\ref{tab:ht-triggers} are consistent with this behaviour: all
%%measurements are statistically compatible with 100\% and no
%%measurement fluctuates above 100\% for the region $\scalht >
%%775\gev$. The only exception is the measurement for the bin
%%(\njet,\scalht) = (2-3,200-275\gev) appears to be significantly below
%%100\%, which is also expected given the aforementioned inefficiencies
%%at Level-1. 
%%
%%While the efficiency measurements in Table~\ref{tab:ht-triggers} have
%%large statistical uncertainties, this limitation does not impact
%%significantly the multjet background estimation method, described in
%%Section~\ref{sec:qcd}, as the method relies on ratios of data yields
%%recorded by the same \verb!HTxxx!  trigger. Regardless, the
%%statistical uncertainties on the efficiencies are folded into the
%%method. 
%%
%%Finally, Table~\ref{tab:ht-triggers} also quotes ``\scalht leg signal
%%efficiencies'', which correspond to the \htalphat signal trigger
%%efficiency on the \alphat plateau (for \alphat threhold values of
%%0.70, 0.65, and 0.60 for the \scalht regions 200--275, 275--325, and
%%$>$325\gev, respectively). As discussed previously, any significant
%%inefficiency for the signal triggers is thought to arise from the slow
%%efficiency turn on for the \scalht leg due to the high thresholds used
%%for the L1 seeds; by comparison the efficiency turn on for the \alphat
%%leg is sharp. Hence, by measuring the signal trigger efficiency on the
%%\alphat efficiency plateau, it is thought that the inefficiency on the
%%\scalht leg can be isolated and compared with the efficiency
%%measurements for the \httrigger triggers. Indeed, the two sets of
%%measurements agree within statistical uncertainties and significant
%%inefficiencies are observed only for the lowest \scalht bins.
%%
%%\begin{table}[!h]
%%  \caption{List of \texttt{HTxxx} triggers and their efficiencies
%%    (\%), as measured in data. For the ``\scalht leg signal
%%    eff.'' columns, see text for further details.}
%%  \label{tab:ht-triggers}
%%  \centering
%%  \scriptsize
%%  \begin{tabular}{ ccccllll }
%%    \hline
%%    \hline
%%    Offline \scalht & L1 seed (\verb!L1_?!) & Trigger (\verb!HLT_?!) &    Typical & \multicolumn{2}{c}{Efficiency (\%)} &    \multicolumn{2}{c}{\scalht leg signal eff. (\%)} \\ [0.5ex]
%%   region (\gev) & (highest thresholds) &  & prescale & \multicolumn{1}{c}{$2 \leq \njet \leq 3$} & \multicolumn{1}{c}{$\njet \geq 4$} & \multicolumn{1}{c}{$2 \leq \njet \leq 3$} & \multicolumn{1}{c}{$\njet \geq 4$} \\ [0.5ex]
%%
%%    \hline                                                                                     
%%    $200 < \scalht < 275$  & \verb!DoubleJetC64!           & \verb!HT250! & 4800     & $\phantom{1}66.4 \pm 14.1$                & $154.3 \pm 154.3$                     & $\phantom{1}81.9 \pm \phantom{1}0.4$ & $\phantom{1}88.5 \pm \phantom{1}0.4$ \\
%%    $275 < \scalht < 325$  & \verb!DoubleJetC64 OR HTT175! & \verb!HT250! & 2400     & $\phantom{1}97.3 \pm 23.0$                & $\phantom{1}91.7 \pm \phantom{1}53.1$ & $\phantom{1}95.2 \pm \phantom{1}0.4$ & $\phantom{1}93.8 \pm \phantom{1}1.2$ \\
%%    $325 < \scalht < 375$  & \verb!DoubleJetC64 OR HTT175! & \verb!HT300! & 1200     & $\phantom{1}79.5 \pm 20.6$                & $198.1 \pm \phantom{1}81.2$           & $\phantom{1}98.0 \pm \phantom{1}0.3$ & $\phantom{1}95.9 \pm \phantom{1}1.2$ \\
%%    $375 < \scalht < 475$  & \verb!DoubleJetC64 OR HTT175! & \verb!HT350! & 600      & $108.7 \pm 18.7$                          & $\phantom{1}54.5 \pm \phantom{1}31.6$ & $\phantom{1}99.5 \pm \phantom{1}0.2$ & $\phantom{1}99.4 \pm \phantom{1}0.4$ \\
%%    $475 < \scalht < 575$  & \verb!DoubleJetC64 OR HTT175! & \verb!HT450! & 150      & $110.6 \pm 15.9$                          & $106.4 \pm \phantom{1}26.8$           & $100.0 \pm \phantom{1}0.3$ & $100.0 \pm \phantom{1}1.1$ \\
%%    $575 < \scalht < 675$  & \verb!DoubleJetC64 OR HTT175! & \verb!HT550! & 70       & $\phantom{1}96.1 \pm 14.7$                & $104.4 \pm \phantom{1}23.1$           & $100.0 \pm \phantom{1}1.0$ & $100.0 \pm \phantom{1}2.5$ \\
%%    $675 < \scalht < 775$  & \verb!DoubleJetC64 OR HTT175! & \verb!HT650! & 25       & $\phantom{1}94.3 \pm 15.4$                & $101.2 \pm \phantom{1}21.5$           & $100.0 \pm \phantom{1}3.4$ & $100.0 \pm \phantom{1}9.8$ \\
%%    $775 < \scalht < 875$  & \verb!DoubleJetC64 OR HTT175! & \verb!HT750! & 1        & $\phantom{1}96.9 \pm \phantom{1}6.1$      & $\phantom{1}94.4 \pm \phantom{11}8.3$ & $100.0 \pm \phantom{1}7.1$ & $100.0 \pm 53.5$ \\
%%    $875 < \scalht < 975$  & \verb!DoubleJetC64 OR HTT175! & \verb!HT750! & 1        & $100.0 \pm \phantom{1}8.4$                & $100.0 \pm \phantom{1}12.6$           & $100.0 \pm 19.8$ & $100.0 \pm 60.0$ \\
%%    $975 < \scalht < 1075$ & \verb!DoubleJetC64 OR HTT175! & \verb!HT750! & 1        & $100.0 \pm 11.2$                          & $100.0 \pm \phantom{1}15.3$           & $100.0 \pm 30.0$ & $100.0 \pm 70.0$ \\
%%    $\scalht > 1075$       & \verb!DoubleJetC64 OR HTT175! & \verb!HT750! & 1        & $100.0 \pm 15.0$                          & $100.0 \pm \phantom{1}22.9$           & $100.0 \pm 40.0$ & $100.0 \pm 80.0$ \\
%%    \hline
%%    \hline                   
%%  \end{tabular}              
%%\end{table}
%%
%%%\clearpage
%%%\begin{figure}[!h]
%%%  \begin{center}
%%%    \subfigure[$\njetlow,   275 < \scalht < 325 \gev$]{
%%%      \includegraphics[width=0.4\textwidth]{figures/cms}
%%%    }
%%%    \subfigure[$\njetlow,   275 < \scalht < 325 \gev$]{
%%%      \includegraphics[width=0.4\textwidth]{figures/cms}
%%%    } \\
%%%    \subfigure[$\njetlow,   325 < \scalht < 375 \gev$]{
%%%      \includegraphics[width=0.4\textwidth]{figures/cms}
%%%    } 
%%%    \subfigure[$\njetlow,   325 < \scalht < 375 \gev$]{
%%%      \includegraphics[width=0.4\textwidth]{figures/cms}
%%%    } \\
%%%    \subfigure[$\njetlow,   375 < \scalht < 475 \gev$]{
%%%      \includegraphics[width=0.4\textwidth]{figures/cms}
%%%    }
%%%    \subfigure[$\njetlow,   375 < \scalht < 475 \gev$]{
%%%      \includegraphics[width=0.4\textwidth]{figures/cms}
%%%    } \\
%%%    \caption{\label{fig:eff-ht-diff-le3j}Left: distribution of the
%%%      \scalht variable in a given \scalht bin after applying the event
%%%      selection described in the text, for events collected with the
%%%      \texttt{IsoMu24\_eta2p1} trigger (red histogram) and the \scalht
%%%      trigger (black histogram). The \scalht trigger used for each
%%%      \scalht bin is defined in Table~\ref{tab:ht-triggers}. Right:
%%%      resulting efficiency turn-on curve for the given \scalht trigger
%%%      and \scalht bin. Note that the \scalht triggers are heavily
%%%      prescaled and so large weights and fluctuations are
%%%      observed. All plots are constructed for the \njetlow
%%%      multiplicity bin.}
%%%  \end{center}
%%%\end{figure}
%%
%%%\begin{figure}[!h]
%%%  \begin{center}
%%%    \subfigure[$\njethigh,   275 < \scalht < 325 \gev$]{
%%%      \includegraphics[width=0.4\textwidth]{figures/cms}
%%%    }
%%%    \subfigure[$\njethigh,   275 < \scalht < 325 \gev$]{
%%%      \includegraphics[width=0.4\textwidth]{figures/cms}
%%%    } \\
%%%    \subfigure[$\njethigh,   325 < \scalht < 375 \gev$]{
%%%      \includegraphics[width=0.4\textwidth]{figures/cms}
%%%    } 
%%%    \subfigure[$\njethigh,   325 < \scalht < 375 \gev$]{
%%%      \includegraphics[width=0.4\textwidth]{figures/cms}
%%%    } \\
%%%    \subfigure[$\njethigh,   375 < \scalht < 475 \gev$]{
%%%      \includegraphics[width=0.4\textwidth]{figures/cms}
%%%    }
%%%    \subfigure[$\njethigh,   375 < \scalht < 475 \gev$]{
%%%      \includegraphics[width=0.4\textwidth]{figures/cms}
%%%    } \\
%%%    \caption{\label{fig:eff-ht-diff-ge4j}Left: distribution of the
%%%      \scalht variable in a given \scalht bin after applying the event
%%%      selection described in the text, for events collected with the
%%%      \texttt{IsoMu24\_eta2p1} trigger (red histogram) and the \scalht
%%%      trigger (black histogram). The \scalht trigger used for each
%%%      \scalht bin is defined in Table~\ref{tab:ht-triggers}. Right:
%%%      resulting efficiency turn-on curve for the given \scalht trigger
%%%      and \scalht bin. Note that the \scalht triggers are heavily
%%%      prescaled and so large weights and fluctuations are
%%%      observed. All plots are constructed for the \njethigh
%%%      multiplicity bin.}
%%%  \end{center}
%%%\end{figure}
%
%\subsection{Muon control samples\label{sec:muon_triggers}}
%
%%The procedure to determine the muon trigger efficiency follows the
%%recommendations of the muon POG~\cite{ref:muon-eff}. Events for the
%%\mj and \mmj control samples are recorded with a loosely-isolated,
%%$\eta$-restricted muon trigger \verb!HLT_IsoMu24_eta2p1!. The muon
%%trigger efficiency is determined for each of the \mj and \mmj control
%%samples according to the binning scheme in \njet and \scalht. Events
%%with muons satisfying the acceptance requirements $\pt > 25\gev$ and
%%$|\eta| < 2.1$ are considered.
%%
%%The \verb!HLT_IsoMu24_eta2p1! trigger efficiency is determined from
%%data in bins of muon $\pt$ and $|\eta|$ by the muon POG from a
%%tag-and-probe method~\cite{ref:muon-eff}. These efficiencies are
%%weighted by MC yields binned according to the muon $\pt$ and $|\eta|$
%%in order to determine a single weighted efficiency measurement per
%%(\njet,\scalht) bin. Simulation-to-data scale factors for muon
%%identification and isolation efficiencies are applied to MC
%%events. These scale factors are also determined by the muon POG in
%%bins of muon $\pt$ and $|\eta|$. Table~\ref{tab:muon-effs} summarises
%%the muon trigger efficiencies which are assumed to have a relative
%%systematic uncertainty of 1\%~\cite{ref:muon-eff}. Efficiencies are
%%higher for the \mmj sample due to the fact that both muons must
%%satisfy $\pt > 30\gev$ and so can be the source of the positive
%%trigger decision. A further consequence of these thresholds
%%requirements is that there is little dependence on \scalht. 
%%
%%\begin{table}[!h]
%%  \caption{Muon trigger efficiencies (\%). Statistical uncertainties
%%    are at the per-mille level, while a relative systematic
%%    uncertainty on all measurements is assumed to be 1\%.}  
%%  \label{tab:muon-effs}
%%  \centering
%%  \footnotesize
%%  \begin{tabular}{ cccccc }
%%    \hline
%%    \hline
%%    \scalht (GeV) \textbackslash \njet & \multicolumn{2}{c}{\mj} & \multicolumn{2}{c}{\mmj} \\ [0.5ex]
%%                                       & 2-3                     & $\geq$4 & 2-3 & $\geq$4  \\ [0.5ex]
%%    \hline
%%%    200--275  & 89.1 & 89.8 & 98.9 & 98.8 \\
%%%    275--325  & 89.3 & 89.8 & 98.9 & 98.8 \\
%%%    325--375  & 89.5 & 90.0 & 98.9 & 98.8 \\
%%%    375--475  & 89.7 & 90.3 & 98.9 & 98.9 \\
%%%    475--575  & 89.8 & 90.5 & 98.9 & 98.9 \\
%%%    575--675  & 90.0 & 90.6 & 98.9 & 98.9 \\
%%%    675--775  & 90.1 & 90.7 & 99.0 & 98.9 \\
%%%    775--875  & 90.2 & 90.8 & 98.9 & 98.9 \\
%%%    875--975  & 90.4 & 90.6 & 99.0 & 98.9 \\
%%%    975--1075 & 90.3 & 90.6 & 99.0 & 98.9 \\
%%%    $>$1075   & 90.0 & 91.2 & 99.0 & 99.1 \\
%%    150--200  & 87.2 & 88.1 & 98.4 & 98.4  \\
%%    200--275  & 87.5 & 88.1 & 98.5 & 98.4  \\
%%    275--325  & 87.8 & 88.2 & 98.5 & 98.4  \\
%%    325--375  & 87.9 & 88.4 & 98.6 & 98.6  \\
%%    375--475  & 88.1 & 88.6 & 98.6 & 98.5  \\
%%    475--575  & 88.2 & 88.8 & 98.6 & 98.6  \\
%%    575--675  & 88.4 & 88.9 & 98.6 & 98.6  \\
%%    675--775  & 88.5 & 89.0 & 98.7 & 98.6  \\
%%    775--875  & 88.6 & 89.1 & 98.6 & 98.6  \\
%%    875--975  & 88.8 & 89.0 & 98.7 & 98.6  \\
%%    975--1075 & 88.7 & 89.0 & 98.7 & 98.8  \\
%%    $>$1075   & 88.4 & 89.6 & 98.7 & 98.7  \\
%%    \hline
%%    \hline
%%  \end{tabular}
%%\end{table}
%%
%%\subsection{Photon control sample\label{sec:photon_triggers}}
%%
%%Events for the photon control sample are recorded with the
%%\verb!HLT_Photon150! trigger, which is $\sim100\%$ efficient for
%%$E_{\rm T}^{\rm photon} > 165\gev$ and $\scalht > 375\gev$, as shown
%%in Figure~\ref{fig:eff-photon}. The efficiency measurement is made
%%using the \verb!HLT_Photon90! trigger as a reference, and we assume
%%that the \verb!L1_SingleEG22! seed is fully efficient.
%%
%%\begin{figure}[!h]
%%  \begin{center}
%%    \subfigure[\njetlow]{
%%      \includegraphics[width=0.48\textwidth,page=3,trim=40 50 160 120,clip=true]{figures/trigger/g_barrel_375_caloJet_le3j.pdf}
%%    }
%%    \subfigure[\njethigh]{
%%      \includegraphics[width=0.48\textwidth,page=3,trim=40 50 160 120,clip=true]{figures/trigger/g_barrel_375_caloJet_ge4j.pdf}
%%    } \\
%%    \caption{\label{fig:eff-photon} Efficiency turn-on curves for
%%      the \texttt{HLT\_Photon150} trigger that records events that
%%      satisfy the \gj selection criteria, $E_{\rm T}^{\rm photon} >
%%      165\gev$, $\scalht > 375\gev$, and \njetlow (Left) and \njethigh
%%      (Right). }
%%  \end{center}
%%\end{figure}





\subsection{Object vetoes}

Standard Model processes with genuine \met from escaping neutrinos can
also produce leptons. Vetoing events with reconstructed leptons help supress
this background.  Events containing an isolated electron~\cite{PAS-EGM-10-004} 
with $\pt >%20\GeV$ and $|\eta| < 2.5$ or an isolated muon~\cite{PAS-MUO-10-002}
with $\pt > 10\GeV$ and $|\eta| < 2.5$ are vetoed. In order to keep this search
all-hadronic and not overlap with other analyses in CMS[REF], events with isolated
photons~\cite{PAS-EGM-10-006}  with $\pt > 25\GeV$ and $|\eta| < 2.5$ are also rejected.
Events which have a single isolated track~\ref{sec:sit} but no object 
associated with them are also rejected.



\subsection{Hadronic pre-selection}

Events are categorized into two jet multiplicites: $\njetlow$ and 
$\njethigh$.  Each event in these categories is further binned by the number
of jets identified as orginating from b quarks ($\bjets$). Events having 
three or more b-tagged jets are not considered in the analysis.      
Significant hadronic activity in the events is ensured by requiring
$\scalht > 375\GeV$. As described in~\ref{sec:trigger}, events are further
binned in 100~\gev bins starting from $\scalht = 375\GeV$ and ending with an open-ended
bin $\scalht > 1075~\gev$. The two highest-$\Et$ jets are subject to a 
higher threshold, also detailed in
Table~\ref{tab:jet-pt-thresholds}, and the highest-$\Et$ jet is
subjected to a tighter $\eta$ acceptance requirement ($|\eta| <
2.5$). The variables \scalht and \mht are computed from the number of
jets, \njet, that satisfy the \Et requirements listed in
Table~\ref{tab:jet-pt-thresholds}. 



\begin{table}[h!]
  \caption{Jet \Et thresholds per \scalht bin.\label{tab:jet-pt-thresholds}}
  \centering
  \footnotesize
  \begin{tabular}{ lcccc }
    \hline
    \hline
    \scalht bin    & $>$375  \\
    \hline
    Lead jet       & 100.0  \\
    Second jet     & 100.0  \\
    All other jets &  50.0  \\
    \hline
    \hline
  \end{tabular}
\end{table}

\clearpage
\subsection{Definition of \texorpdfstring{\alphat}{AlphaT}\label{sec:alphat}}

Multi-jet events stemming from QCD contribute overwhelmingly to
all-hadronic events recorded by the detector. Generally, these QCD events
have no significant inbalance in transverse energy and can 
be reduced with a requirement for high-\met. Yet, this approach has proven difficult, 
as the calculation of \met is sensitive to detector effects and conditions, 
and additionaly, in the high-\met phase space systematic uncertainties become difficult
to control. The variable, \alphat, introduced in 2008 by Randall and 
Tucker-Smith~\cite{Randall:2008rw} effectively supresses multi-jet 
events with no significant met without relying on the measurment
of \met. By construction it also introduces robustness against mismeasurements 
of transverse eneriges in multi-jet systems.  In the simplest case of a two-jet system,
\alphat is defined as
\begin{equation}
\label{eq:alphat}
\alphat\, =\, \frac{\Et^{{\rm j}_2}}{M_\text{T}} \, ,
\end{equation}

where $\Et^{\rm j_2}$ is the transverse energy of the less energetic
jet and $M_\text{T}$ is the transverse mass of the dijet system,
defined as

\begin{equation}
  \label{eq:mt}
  M_\text{T}\, = \,\sqrt{ \left( \sum_{i=1}^2 \Et^{{\rm j}_i}
    \right)^2 - \left( \sum_{i=1}^2 p_x^{{\rm j}_i} \right)^2 - \left(
      \sum_{i=1}^2 p_y^{{\rm j}_i} \right)^2} \, = \,\sqrt{\scalht^2 + \left(\mht\right)^2} \,  .
\end{equation}

where $\Et^{{\rm j}_i}$, $p_x^{{\rm j}_i}$, and $p_y^{{\rm j}_i}$ are,
respectively, the transverse energy and $x$ or $y$ components of the
transverse momentum of jet ${\rm j}_i$. In well measured QCD dijet events, 
transverse momentum conservation requires the $\pt$ of the two jets to be 
of equal magnitude and back-to-back in the plane transverse to the beam.
The value of \alphat for these type of events can be best seen
after transforming equation~\ref{eq:mt} into CMS coordinates:

\begin{equation}
  \label{eq:mt-polar}
  \alphat\, = \,\frac{\Et^{{\rm j}_2}}{\sqrt{2\Et^{{\rm j}_1}
   \Et^{{\rm j}_2} \left(1-cos\left(\Delta\phi\right)\right)}}\, .
\end{equation}

Well balanced back-to-back jets have \alphat values of 0.5 
For the case of an imbalance in the measured transverse energies 
of back-to-back jets \alphat is reduced to values smaller than 0.5, 
this gives the variable its robustness with respect to jet energy 
mismeasurements. Figure~\ref{fig:alphat_dist} shows the \alphat 
distribution in two jet multiplicity bins for events passing 
an HT trigger and basic pre-selection requirments. Multi-jet 
events from QCD drop to neglible levels above $\alphat>$ 0.55
while events with genuine \met continue to populate bins in excess
of  $\alphat>$ 0.55.

In the case where the two 
jets are not back-to-back in the transverse plane but rather the 
dijet system balances genuine \met, for example in leptonically 
decaying W boson recoiling off a syetem of jets, the values of 
\alphat can exceed 0.5.

\begin{figure}[h!t]
  \begin{center}
    \subfigure[\label{fig:alphat_le3j}]{
      \includegraphics[width=0.45\textwidth,]{figures/data-mc/AlphaT_le3j.png}
    } 
    \subfigure[\label{fig:alphat_ge4j}]{
      \includegraphics[width=0.45\textwidth,]{figures/data-mc/AlphaT_ge4j.png}
    } \\
    \caption{\alphat distribution from~\cite{RA1Paper2012} }
    \label{fig:alphat_dist}
  \end{center}
\end{figure}

An extension of \alphat calculation can be made for systems of 
more than two jets~\cite{cms-pas-sus-09001} by clustering the jets into a 
two pseudo-jets system. A pseudo-jet's $\Et$ is calculated
as the scalar sum of the contributing jets' transverse energy, while the total
transverse energy of the system, \scalht, is defined as the scalar sum of the 
pseudo-jet's transverse energy: $\scalht = \Et^{{\rm pj}_1}+\Et^{{\rm pj}_2}$.
All combinations of the n-jet system are tested, and the configuration which
balances the constructed pseudo-jets' \Et is chosen, ie the combinations which
minimized $\dht = \Et^{{\rm pj}_1}-\Et^{{\rm pj}_2}$. As balanced QCD events
are expected to have small \dht, this simple clustering criterion provides the best
separation between QCD events and events with genuine \met. 
Equation~(\ref{eq:alphat}) can therefore be generalised as:

\begin{equation}
  \label{eq:alphat2}
  \alphat\, = \,\frac{1}{2} \times \frac{\scalht -
    \dht}{\sqrt{\scalht^2 - \mht^2}} \, = \,\frac{1}{2} \times 
  \frac{1 - (\dht/\scalht)}{\sqrt{1 - (\mht/\scalht)^2}} \, . 
\end{equation}

In the limit that  $\dht \rightarrow 0$, the ratio of missing energy and
visible energy can be expressed as a function of \alphat:

\begin{equation}
  \label{eq:alphat3}
  \frac{\mht}{\scalht} \, = \, \sqrt{ 1 - \frac{1}{4 \cdot \alphat^2} }
\end{equation}

This analysis uses an \alphat threshold of 0.55, which results in events
having more than 40\% percent as much missing transverse energy as visible transverse
energy. For an event with $\scalht=$~375~\gev, this amounts to nearly 160 GeV in \mht.  
%For reference, under the assumption of $\dht = 0$, the values of
%$\alphat = 0.55$, 0.60, and 0.65 map onto values of the ratio
%$\mht/\scalht = 0.42$, 0.55, and 0.64.
%


\subsection{The hadronic signal region\label{sec:had-signal}}

Following the hadronic pre-selection, the multijet background from QCD
is still several orders of magnitude larger than the typical signal
expected from SUSY. The multijet background can be rejected with very
high efficiency by requiring $\alphat > 0.55$ (plus the application of
two dedicated cleaning filters, described below in
Sec.~\ref{sec:had-signal}).

These criteria are sufficient to suppress the QCD multijet
contribution to the sub-percent level with respect to the non-multijet
backgrounds for the region $\scalht > 325\gev$. For the regions $200 <
\scalht < 275\gev$ and $275 < \scalht < 325\gev$, higher thresholds of
0.65 and 0.60 are used, as detailed in
Table~\ref{tab:alphat-thresholds}. The method used to determine the
multijet prediction as a function of the \alphat threshold is
described in Section~\ref{sec:qcd}. The higher \alphat thresholds
provide added protection against the effect of jets below threshold
contributing significiantly to \mht, ensuring that the relative
multijet contribution is maintained at the sub-percent level, even for
the new region at very low \scalht (where the jets are softest and
resolutions are poor) and the higher pileup conditions experienced
during Run D. The choice of \scalht and \alphat thresholds are also
driven by trigger constraints.

\begin{table}[h!]
  \caption{\alphat and (effective) \mht/\scalht and \mht thresholds per \scalht bin.\label{tab:alphat-thresholds}}
  \centering
  \footnotesize
  \begin{tabular}{ lcccc }
    \hline
    \hline
    \scalht bin  & 200--275   & 275--325   & 325--375   & $>$375       \\
    \hline
    \alphat      & 0.65       & 0.60       & 0.55       & 0.55         \\
    \mht/\scalht & $\sim$0.64 & $\sim$0.55 & $\sim$0.42 & $\sim$0.42   \\
    \mht         & $\sim$130  & $\sim$150  & $\sim$135  & $\gtrsim$155 \\
    \hline
    \hline
  \end{tabular}
\end{table}

Finally, some additional cleaning filters are added following
the \alphat requirement to protect against pathological effects such
as reconstruction failures or severe energy losses due to detector
inefficiencies. 

To protect against multiple jets failing the $\Et$ threshold, the
jet-based estimate of the missing transverse energy, \mht, is compared
to the Particle Flow estimate of missing transverse energy, $\pfmet$,
and events with $R_{\rm miss}=\mht/\pfmet > 1.25$ are rejected.

To protect against severe energy losses, events with significant jet
mismeasurements caused by masked regions in the ECAL (which amount to
about 1\% of the ECAL channel count), or by missing instrumentation in
the barrel-endcap gap, are removed with the following procedure. The
jet-based estimate of the missing transverse energy, \mht, is used to
identify jets most likely to have given rise to the \mht as those
whose momentum is closest in $\phi$ to the total $\vec{\mht}$ which
results after removing them from the event.  The azimuthal distance
between this jet and the recomputed \mht is referred to as
$\Delta\phi^*$ in what follows. Events with $\Delta\phi^* < 0.5$ are
rejected if the distance in the ($\eta,\phi$) plane between the
selected jet and the closest masked ECAL region, $\Delta R_{\rm
  ECAL}$, is smaller than 0.3. Similarly, events are rejected if the
jet points within 0.3 in $\eta$ of the ECAL barrel-endcap gap at
$|\eta| = 1.5$. These final selections complete the definition of the
acceptance of the hadronic signal sample.



%
%Once all the signal selection requirements have been imposed, the
%contribution from QCD multijet events is expected to be negligible, as
%demonstrated in Section~\ref{sec:qcd}.
%
%The remaining backgrounds in the hadronic signal region stem from SM
%processes with genuine \met in the final state.  In the case of events
%where no b-quark jets are identified, the largest backgrounds with
%genuine \met arise from the production of W and Z bosons in
%association with jets. The weak decay \znunu\ is the only significant
%contribution from Z + jets events. For W + jets events, the two
%dominant sources are leptonic W decays in which the lepton is not
%reconstructed or fails the isolation or acceptance requirements, and
%the weak decay $\wtaunu$ where the $\tau$ decays hadronically and is
%identified as a jet. Contributions from SM processes such as
%single-top, Drell-Yan, and diboson production are also expected. For
%events with one or more reconstructed b-quark jets, \ttbar production
%followed by semi-leptonic weak decays becomes the most important
%single background source. For events with only one reconstructed
%b-quark jet, the contribution of both W + jets and Z + jets
%backgrounds are of a similar size to the \ttbar background.  For
%events with two reconstructed b-quark jets, \ttbar production
%dominates, while events with three or more reconstructed b-quark jets
%originate almost exclusively from \ttbar events, in which at least one
%jet is misidentified as originating from a bottom quark.
%
%In order to estimate the contributions from each of these backgrounds,
%three data control samples are used, which are binned identically to
%the signal region. Two independent estimates of the irreducible
%background of \znunu\ + jets events are determined from the data
%control samples comprising $Z\rightarrow\mu\mu$ + jets and $\gamma$ +
%jets events, which have similar kinematic properties but different
%acceptances. The $Z\rightarrow\mu\mu$ + jets events have similar
%kinematic properties when the two muons are ignored, but a smaller
%branching fraction, while the $\gamma$ + jets events have similar
%kinematic properties when the photon is
%ignored~\cite{PAS-SUS-08-002,Bern:2011pa}, but a larger production
%cross section. A $\mu$ + jets data sample provides an estimate for all
%other SM backgrounds, which is dominated by \ttbar and W production
%leading to W + jets final states.
%
%As described previously, the event selection criteria for the control
%samples are defined to ensure that any potential contamination from
%multijet events is negligible. Further, the control sample selection
%criteria also suppress contributions from a wide variety of SUSY
%models, including those considered in this analysis. Any potential
%signal contamination in the data control samples is accounted for in
%the fitting procedure described in Section~\ref{sec:results}.
%
%\subsection{Overview of the method\label{sec:background-method}}
%
%The method used to estimate the aforementioned SM background
%contributions in the hadronic signal region relies on the use of a
%{\it transfer factor} (TF) determined from MC samples to transform the
%observed yield in a given \scalht, jet (\njet) and b-tag (\nb)
%multiplicity bin of a control sample, $\nobs^{\rm
%  control}(\scalht,\njet,\nb)$, into a predicted yield for the
%corresponding bin of the hadronic signal region, $\npre^{\rm
%  signal}(\scalht,\njet,\nb)$. The choice of \njet and \nb event
%categorisation and \scalht binning in the control samples is identical
%to that for the signal region, as defined in Table~\ref{tab:ht-bins}
%in Section~\ref{sec:selection}. 
%
%Each transfer factor is simply a ratio of the yields obtained from MC
%simulation for the same bin of the signal region and a given control
%sample:
%
%\begin{equation}
%  \label{equ:tf-ratio}
%  {\rm TF} = \frac{N_{\rm MC}^{\rm signal}(\scalht,\njet,\nb)}{N_{\rm
%      MC}^{\rm control}(\scalht,\njet,\nb)} 
%\end{equation}
%
%In this way, ``na\"ive'' predictions for the total SM background can
%be made by considering separately the sum of the predictions from the
%\mj and \gj samples or the \mj and \mmj samples:
%
%\begin{equation}
%  \label{equ:pred-method}
%  \npre^{\rm signal}(\scalht,\njet,\nb) = \frac{N_{\rm MC}^{\rm
%      signal}(\scalht,\njet,\nb)}{N_{\rm MC}^{\rm
%      control}(\scalht,\njet,\nb)} \times \nobs^{\rm
%    control}(\scalht,\njet,\nb)   
%\end{equation}
%
%When constructing the transfer factors, the MC expectations for the
%following SM processes are considered: W + jets ($N_{\rm W}$), \ttbar
%+ jets ($N_{\ttbar}$), \znunu\ + jets ($N_{\znunu}$), DY + jets
%($N_{\mathrm DY}$), \gj ($N_\gamma$), single top + jets
%production via the s, t, and tW-channels ($N_{\rm top}$), and WW +
%jets, WZ + jets, and ZZ + jets ($N_{\rm di-boson}$). Details on the MC
%samples used are given in Sec.~\ref{sec:mc-samples}. All MC samples
%are normalised to the intergrated luminosity of the appopriate data
%sample.
%
%While ``na\"ive'' predictions can be made using
%Equ.~\ref{equ:pred-method}, the fitting prcedure that provides the
%final result is defined formally by the likelihood model described in
%Sec.~\ref{sec:statistics}. In summary, the observation in each bin
%(defined in terms of the variables \njet, \nb, and \scalht) of the
%signal sample is modelled as Poisson-distributed about the sum of a SM
%expectation (and a potential signal contribution). The components of
%this SM expectation are related to the expected yields in the control
%samples via transfer factors derived from simulation, as described
%in Sec.~\ref{sec:background-method}. The observations in each bin
%(again defined by \njet, \nb, and \scalht) of the control samples are
%similarly modelled as Poisson-distributed about the expectated yields
%for each control sample. In this way, for a given bin, the observed
%yields in the signal and control samples are all connected via the
%transfer factors derived from simulation.
%
%The sum of expected yields from all MC samples, obtained for the
%relevant control sample selection, enter the denominator of each
%transfer factor:
%
%\begin{equation}
%  \label{equ:ratio-denom}
%  N_{\rm MC}^{\rm control}(\scalht,\njet,\nb) = N_{\rm W} + N_{\ttbar} + N_{\znunu} +
%N_{\rm DY} + N_{\gamma} + N_{\rm top} + N_{\rm di-boson}
%\end{equation}
%
%For the b jet multiplicity bins satisfying $n_b \leq 1$, the \mj
%control sample is used to predict primarily the W + jets and \ttbar +
%jets backgrounds, but all remaining residual SM backgrounds except
%\znunu\ + jets events are also considered. (The \znunu\ + jets
%background is instead accounted for through the \mmj and \gj samples.)
%The sum of expected yields from all MC samples except \znunu\ + jets,
%obtained for the hadronic signal region selection, enter the numerator
%of each transfer factor:
%
%\begin{equation}
%  \label{equ:ratio-numer-mj}
%  N_{\rm MC}^{\rm signal}(\scalht,\njet,\nb \leq 1) = N_{\rm W} +
%  N_{\ttbar} + N_{\rm DY} + N_{\rm top} + N_{\rm di-boson}
%\end{equation}
%
%For the same b jet multiplicity bins ($0 \leq \nb \leq 1$), the \mmj
%and \gj control samples are used to predict the \znunu\ + jets process
%only, and the expected yields in the bins of the signal region as
%obtained from the \znunu\ sample enter the numerator of each transfer
%factor:
%
%\begin{equation}
%  \label{equ:ratio-numer-mmj}
%  N_{\rm MC}^{\rm signal}(\scalht,\njet,0 \leq \nb \leq 1) = N_{\znunu}
%\end{equation}
%
%For the b jet multiplicity bins satisfying $n_b \geq 2$, the \mj
%control sample is again used to predict primarily the W + jets and
%\ttbar + jets backgrounds, but also to predict all other remaining
%residual SM backgrounds {\it including} \znunu\ + jets events. The sum
%of expected yields from all MC samples {\it including} \znunu\ + jets,
%obtained for the hadronic signal region selection, enter the numerator
%of each transfer factor:
%
%\begin{equation}
%  \label{equ:ratio-numer-mj}
%  N_{\rm MC}^{\rm signal}(\scalht,\njet,\nb \geq 2) = N_{\rm W} +
%  N_{\ttbar} + N_{\rm DY} + N_{\rm top} + N_{\rm di-boson} + N_{\znunu}
%\end{equation}
%
%In this case, the \mmj and \gj control samples are not used, as the
%yields in these two data control samples are expected to be negligible
%due to the requirement of at least two b jets per event. The method of
%using a W + jets sample to predict the \znunu\ + jets background has
%been used previously~\cite{RA1Paper, RA1Paper2011}, and this approach
%is addressed by a set of closure tests described in
%Sec.~\ref{sec:bkgd-syst}, in which a \mj sample (rich in W + jets and
%\ttbar events) is used to make predictions of yields in a \mmj sample
%(rich in Z$\rightarrow\mu\mu$ + jets events). The data control samples
%used to predict the SM backgrounds for each event category are
%summarised in Table~\ref{tab:fit-plots}.
%
%\begin{table}[ht!]
%  \caption{Summary of control samples used to predict the SM
%    background for each event category. }
%  \label{tab:fit-plots}
%  \centering
%  \begin{tabular}{ lll }
%    \hline
%    \hline
%    \njet   & \nb     & Control samples \\ [1.0ex]
%    \hline
%    2--3    & 0       & \mj, \mmj, \gj  \\
%    2--3    & 1       & \mj, \mmj, \gj  \\
%    2--3    & 2       & \mj             \\
%    $\geq$4 & 0       & \mj, \mmj, \gj  \\
%    $\geq$4 & 1       & \mj, \mmj, \gj  \\
%    $\geq$4 & 2       & \mj             \\
%    $\geq$4 & 3       & \mj             \\
%    $\geq$4 & $\geq4$ & \mj             \\
%    \hline
%    \hline
%  \end{tabular}
%\end{table}
%
%The selection criteria for the three control samples closely resemble
%those for the signal region, differing mainly through the use of a
%muon, di-muon, or photon {\it tag} (that is ignored in the calculation
%of jet-based kinematic variables such as \scalht, \mht, \alphat, \etc)
%and some minimal additional kinematic requirements (\eg invariant or
%transerve mass windows) to obtain W, Z, and \ttbar-enriched event
%samples. The same selection criteria are designed to suppress signal
%contamination in the control samples so that unbiased data-driven
%estimates for the SM backgrounds in the signal region can be
%made. Hence, we refer to these samples as {\it control} samples
%although in the final simultaneous fit, any potential signal
%contamination is properly taken into account.
%
%The control sample definitions and binning scheme are chosen so that
%the reliance on simulation to extrapolate correctly from a control
%region to the signal region is minimised. Many systematic effects are
%expected to cancel largely in the transfer factor. However, a
%systematic uncertainty is assigned to each transfer factor to account
%for theoretical uncertainties and effects such as the mismodelling of
%kinematics (\eg acceptances) and instrumental effects (\eg
%reconstruction efficiencies), as described in
%Sec.~\ref{sec:bkgd-syst}.
%
%%Kinematic cuts are applied to enrich as much as possible the \wj,
%%\ttbar, and \znunu components in the muon and di-muon control
%%samples. The definition of the samples are geared towards efficiency
%%rather than purity (even so, the purities are at the level $>$90\%)
%%and any contamination from "backgrounds" (\eg \ttbar in the case of
%%the \mmj sample) are simply incorporated into the transfer
%%factors. Alternatively, a zero b-jet requirement can be applied to the
%%\mmj sample to obtain a higher purity of Z$\rightarrow\mu\mu$ + jets
%%events (\ie, with reduced contamination from \ttbar), which can then
%%be used to give an expectation for the \znunu + jets background for
%%all b-jet categories in the signal region.
%
%\subsection{Definition of the control samples\label{sec:def-control-samples}}
%
%\subsubsection{The \texorpdfstring{\mj}{muon plus jets} control sample}
%
%Events from the \wj and \ttbar processes are found in the hadronic
%signal sample due to unidentified leptons (either out of acceptance or
%not reconstructed) and hadronic tau decays originating from
%high-p$_{T}$ W bosons. An estimate of these background processes is
%obtained through the use of a \mj sample. The selection criteria for
%this sample are chosen to identify W bosons decaying to a muon and a
%neutrino in the phase-space of the signal. The muon is not considered
%in the calculation of event-level variables such as \scalht, \mht and
%\alphat. All cuts on such jet-based quantities are consistent with
%those applied in the hadronic search region and the same \njet, \nb,
%and \scalht binning is used. The only exception is that no \alphat
%requirement is made, as motivated by the discussion in
%Sec.~\ref{sec:larger}. In order to select events containing W bosons,
%exactly one tight isolated muon within an acceptance of \PT $>$ 30
%\gev and $|\eta| <$ 2.1 is required (due to the trigger), and the
%transverse mass of the W candidate must satisfy $30 < \mt(\mu,\pfmet)
%< 125\gev$ (to suppress QCD multijet and potential signal
%events). Events are vetoed if $\Delta R(\mu,\textrm{jet}_i) < 0.5$
%running over all jets $i$. The single isolated track veto, described
%in Sections~\ref{sec:reconstruction} and~\ref{sec:vetoes}, is also
%applied, which considers all single isolated tracks in the event
%except that associated with the identified, isolated muon. Finally,
%the cleaning cut $\mht/\pfmet$ is also applied, as done in the signal
%region, where the \pfmet is adjusted to account for the transverse
%momentum of the identified, isolated muon.
%
%% Events are vetoed if either of the following conditions are met:
%% $\Delta R(\mu,\textrm{jet}_i) < 0.5$, running over all jets $i$; or
%% a second muon candidate exists that is either loose, non-isolated or
%% outside acceptance and the two muons have an invariant mass that
%% satisfies $m_{Z} - 25 < M_{\mu_1\mu_2} < m_{Z} + 25$ (to suppress
%% $Z\rightarrow\mu\mu$).
%
%\subsubsection{The \texorpdfstring{\mmj}{di-muon plus jets} control sample}
%
%The \znunu\ + jets process forms an irreducible background and can be
%estimated using the \zmumu + jets process, which has similar kinematic
%properties but a different acceptance and a smaller branching ratio. A
%background estimate is obtained through the use of a \mmj sample. Most
%of the selection criteria are identical to those for the \mj sample,
%but the few that differ are tuned to identify Z bosons decaying to two
%muons in the kinematic phase space of the signal region. The muons are
%not considered in the calculation of event-level variables such as
%\scalht, \mht and \alphat. All cuts on such jet-based quantities are
%consistent with those applied in the hadronic search region and the
%same \njet, \nb, and \scalht binning is used. The only exception is
%that no \alphat requirement is made, as motivated by the discussion in
%Sec.~\ref{sec:larger}. In order to select an event sample containing Z
%bosons, exactly two tight isolated muons within an acceptance of $\Pt
%> 30\gev$ and $|\eta| < 2.1$ are required (due to the trigger). The
%invariant mass of the two muons must satisfy $m_{Z} - 25 <
%M_{\mu_1\mu_2} < m_{Z} + 25$. Events are vetoed if $\Delta
%R(\mu_{i},\textrm{jet}_j) < 0.5$ is satisfied, running over all muons
%$i$ and all jets $j$. The single isolated track veto, described in
%Sections~\ref{sec:reconstruction} and~\ref{sec:vetoes}, is also
%applied, considering all single isolated tracks in the event except
%those associated with the two identified, isolated muons. Finally, the
%cleaning cut $\mht/\pfmet$ is also applied, as done in the signal
%region, where the \pfmet is adjusted to account for the transverse
%momenta of the two identified, isolated muons. The \mmj sample can be
%used to make predictions in all the \scalht bins, providing coverage
%at low \scalht where the \gj sample cannot.
%
%\subsubsection{The \texorpdfstring{\gj}{photon plus jets} control sample}
%
%The \znunu\ + jets process can also be estimated using the \gj
%process, which has a larger cross section and kinematic properties
%similar to those of \znunu\ events when the photon is
%ignored~\cite{PAS-SUS-08-002,Bern:2011pa}. The \gj sample is defined
%by requiring exactly one photon satisfying tight isolation criteria
%and within an acceptance of $\pt > 165\gev$ and $|\eta| <
%1.45$. Furthermore, events are vetoed if $\Delta
%R(\gamma,\textrm{jet}_j) < 1.0$ is satisfied, running over all jets
%$j$. As for the muon-based samples, the photon is not considered in
%the calculation of event-level variables such as \scalht, \mht and
%\alphat. All cuts on jet-based quantities are consistent with those
%applied in the hadronic search region, and the same \HT binning is
%used. Given that the photon is ignored, the \gj sample can only be
%used for the region $\scalht > 375\gev$ due to the photon acceptance
%of $\pt > 165\gev$ (enforced by the trigger) and the requirement
%$\alphat > 0.55$.
%
%\subsection{Increasing the acceptance of the muon control samples\label{sec:larger}}
%
%As described in Sec.~\ref{sec:def-control-samples} above, the
%selection criteria of the three control samples are defined such that
%the background composition and event kinematics of the three control
%samples mirror as closely as possible those for the signal
%region. This is done in order to minimise the reliance on the
%simulation to model correctly the backgrounds and event kinematics in
%the control and signal samples.
%
%However, in the case of the \mj and \mmj samples, no requirement is
%made on \alphat in the selection criteria of the samples. This is made
%possible by the remaining kinematic selection criteria, which are
%sufficiently selective to ensure that the muon samples remain rich in
%events from the \wj, \ttbar and \zmumu processes with negligible
%contamination from QCD multijet events. These selection criteria
%include, for example, requiring exactly one or two tight isolated
%muon(s), and imposing acceptance windows on the invariant mass of the
%di-muon sytem or the transverse mass of the muon-\pfmet system, as
%described above. 
%%The absence of QCD multijet events is demonstrated by the control
%%distributions shown in Sec.~\ref{sec:est-control-samples} below. 
%Thus, the acceptance of the two muon control samples can be
%significantly increased, which simultaneously improves their
%predictive power and further reduces the effect of any potential
%signal contamination.  In the case of the \gj sample (used only for
%the region $\scalht > 375\gev$), the requirement $\alphat > 0.55$ is
%still necessary to suppress contamination from QCD multijet events,
%even after the substantial photon \pt cut in the offline selection.
%
%The extrapolation in the variable \alphat is tested through a
%dedicated set of closure tests, described in Sec.~\ref{sec:bkgd-syst},
%which demonstrate that the different \alphat requirements for the \mj
%and \mmj control samples and signal region have no significant
%systematic bias on the prediction. That the \alphat variable
%introduces no acceptance bias for processes with genuine \met is due
%to the accurate modelling by the CMS simulation of such processes,
%namely W + jets, \ttbar, and \znunu\ + jets. Background estimates for
%these processes are provided by the \mj and \mmj samples, as
%identified at the beginning of this Section. Importantly, the same
%cannot be said for QCD multijet events, as in this case the only
%events that survive the \alphat cut are pathological cases in which a
%jet is severely mismeasured or even lost due to detector
%inefficiencies. Such effects certainly do change the event kinematics
%and, in these cases, an \alphat cut will selectively choose events
%with particular kinematic features and topologies. For these
%pathological cases, one cannot rely on MC to model correctly the
%behaviour and therefore the \alphat acceptance. This is a crucial
%distinction to be made between QCD multijet events and processes with
%significant genuine \met. The assumption is that processes with
%genuine \met are selected by the \alphat variable based on the
%escaping invisible particle(s) rather than any pathological effects.
%
%\subsection{Distributions from the data control samples\label{sec:est-control-samples}}
%
%Distributions of key variables for the \mj, \mmj, and \gj samples are
%shown in below in Figs.~\ref{fig:mu-distr}, \ref{fig:mumu-distr},
%and~\ref{fig:phot-distr}, respectively. The first two figures show
%the (leading) muon \pt and isolation distributions, along with the
%leading jet \pt, \scalht, \mht and \alphat distributions. For the \gj
%sample, the photon \pt distribution is%and isolation distributions are
%shown in place of the corresponding muon distribution.%s.
%No requirement is made on the number of b-jets per event. In general,
%the agreement between data and simulation is good, giving confidence
%that the samples are well understood. The MC distributions highlight
%the composition of each sample. The contribution from QCD multijet
%events is expected to be negligible.
%
%Figure~\ref{fig:jet-bjet} shows the jet and b-jet multiplicity
%distributions for the \mj, \mmj, and \gj control samples, as defined
%in Sec.~\ref{sec:def-control-samples} and following the requirement
%$\scalht > 375\gev$. An accurate modelling of the jet multiplicity in
%data is achieved for all samples. The b-jet distributions demonstrate
%the changing background composition as a function of the number of
%b-jets. For the \mj and \mmj samples, the requirement of zero b-jets
%results in sub-samples that are rich in W and Z bosons, respectively,
%with little contamination from \ttbar. The \ttbar background becomes
%dominant in the \mj sample when exactly one b-jet is required. The
%requirement of up to two b-tags per event significantly suppresses all
%processes except for \ttbar production. Requiring at least three
%b-tags also suppresses \ttbar production. In the case of the \mmj
%sample, some contamination from \ttbar is observed for $\nb =
%1$. Requiring more than one b-jet significantly suppresses the yields
%in both the \mmj and \gj samples.
%
%\clearpage
%\begin{figure}[!h]
%  \centering
%  \subfigure[Muon \pt.]{
%    \includegraphics[width=0.4\textwidth]{figures/data-mc/v1/muon/Stack_muPt_Muon_all_OneMuon_-1To-2b_log}
%  } 
%  \subfigure[Muon isolation.]{
%    \includegraphics[width=0.4\textwidth]{figures/data-mc/v1/muon/Stack_muonIso_Muon_all_OneMuon_-1To-2b_log}
%  } \\
%  \subfigure[Transverse mass.]{
%    \includegraphics[width=0.4\textwidth]{figures/data-mc/v1/muon/Stack_PFMTmu_Muon_all_OneMuon_-1To-2b_log}
%  } 
%  \subfigure[\scalht.]{
%    \includegraphics[width=0.4\textwidth]{figures/data-mc/v1/muon/Stack_HT_Muon_all_OneMuon_-1To-2b_log}
%  } \\
%  \subfigure[\mht.]{
%    \includegraphics[width=0.4\textwidth]{figures/data-mc/v1/muon/Stack_MHT_Muon_all_OneMuon_-1To-2b_log}
%  } 
%  \subfigure[\alphat.]{
%    \includegraphics[width=0.4\textwidth]{figures/data-mc/v1/muon/Stack_AlphaT_Muon_all_OneMuon_-1To-2b_log}
%  } 
%  \caption{Data--MC comparisons of key variables for the \mj control
%    sample, for the region $\scalht > 275\GeV$. Bands represent the
%    uncertainties due to the limited size of MC samples. No
%    requirement on the number of b-jets per event is made.}
%  \label{fig:mu-distr}
%\end{figure}
%
%\begin{figure}[!h]
%  \centering
%  \subfigure[Leading muon \pt.]{
%    \includegraphics[width=0.4\textwidth]{figures/data-mc/v1/mumu/Stack_muPt_Muon_all_DiMuon_-1To-2b_log}
%  } 
%  \subfigure[Muon isolation.]{
%    \includegraphics[width=0.4\textwidth]{figures/data-mc/v1/mumu/Stack_muonIso_Muon_all_DiMuon_-1To-2b_log}
%  } \\
%  \subfigure[Di-muon invariant mass.]{
%    \includegraphics[width=0.4\textwidth]{figures/data-mc/v1/mumu/Stack_Zmass_Muon_all_DiMuon_-1To-2b_log}
%  } 
%  \subfigure[\scalht.]{
%    \includegraphics[width=0.4\textwidth]{figures/data-mc/v1/mumu/Stack_HT_Muon_all_DiMuon_-1To-2b_log}
%  } \\
%  \subfigure[\mht.]{
%    \includegraphics[width=0.4\textwidth]{figures/data-mc/v1/mumu/Stack_MHT_Muon_all_DiMuon_-1To-2b_log}
%  } 
%  \subfigure[\alphat.]{
%    \includegraphics[width=0.4\textwidth]{figures/data-mc/v1/mumu/Stack_AlphaT_Muon_all_DiMuon_-1To-2b_log}
%  } 
%  \caption{Data--MC comparisons of key variables for the \mmj control
%    sample, for the region $\scalht > 275\GeV$. Bands represent the
%    uncertainties due to the limited size of MC samples. No
%    requirement on the number of b-jets per event is made.}
%  \label{fig:mumu-distr}
%\end{figure}
%
%\begin{figure}[!h]
%  \centering
%  \subfigure[Photon \pt.]{
%    \includegraphics[width=0.4\textwidth]{figures/data-mc/v1/photon/Stacked_PhotonPt_all_Photon_375_upwards}
%  } 
%  \subfigure[\scalht.]{
%    \includegraphics[width=0.4\textwidth]{figures/data-mc/v1/photon/Stacked_HT_after_alphaT_55_all_Photon_375_upwards}
%  } \\
%  \subfigure[\mht.]{
%    \includegraphics[width=0.4\textwidth]{figures/data-mc/v1/photon/Stacked_MHT_after_alphaT_55_all_Photon_375_upwards}
%  } 
%  \subfigure[\alphat.]{
%    \includegraphics[width=0.4\textwidth]{figures/data-mc/v1/photon/Stacked_AlphaT_all_Photon_375_upwards}
%  } 
%  \caption{Data--MC comparisons of key variables for the \gj control
%    sample, for the region $\scalht > 375\GeV$. Bands represent the
%    uncertainties due to the limited size of MC samples. No
%    requirement on the number of b-jets per event is made.}
%  \label{fig:phot-distr}
%\end{figure}
%
%\begin{figure}[!h]
%  \centering
%  \subfigure[\njet for the \mj sample.]{
%    \includegraphics[width=0.4\textwidth]{figures/data-mc/v1/muon/Stack_ncommjet_Muon_all_OneMuon_-1To-2b_log}
%  } 
%  \subfigure[\njet for the \mmj sample.]{
%    \includegraphics[width=0.4\textwidth]{figures/data-mc/v1/mumu/Stack_ncommjet_Muon_all_DiMuon_-1To-2b_log}
%  } \\
%  \subfigure[\njet for the \gj sample.]{
%    \includegraphics[width=0.4\textwidth]{figures/data-mc/v1/photon/Stacked_JetMultiplicityAfterAlphaT_55_all_Photon_375_upwards}
%  } 
%  \subfigure[\nb for the \mj sample.]{
%    \includegraphics[width=0.4\textwidth]{figures/data-mc/v1/muon/Stack_nbjet_Muon_all_OneMuon_-1To-2b_log}
%  } \\
%  \subfigure[\nb for the \mmj sample.]{
%    \includegraphics[width=0.4\textwidth]{figures/data-mc/v1/mumu/Stack_nbjet_Muon_all_DiMuon_-1To-2b_log}
%  } 
%  \subfigure[\nb for the \gj sample.]{
%    \includegraphics[width=0.4\textwidth]{figures/data-mc/v1/photon/Stacked_Btag_Post_AlphaT_5_55_all_Photon_375_upwards}
%  } 
%  \caption{Data--MC comparison of the number of reconstructed jets
%    (top) and b-jets (bottom) per event in the (left) \mj sample,
%    (middle) \mmj sample, and (right) \gj control sample. Bands
%    represent the uncertainties due to the limited size of MC
%    samples.}\label{fig:jet-bjet}
%\end{figure}

%\subsection{Transfer factors and ``na\"ive'' predictions\label{sec:est-control-samples}}
%
%Appendix~\ref{app:tf} contains tables summarising the observed and
%expected yields from data and simulation, respectively, in the bins of
%the three control samples. Also listed are the expectations from
%simulation for the various background contributions in the signal
%region, along with the corresponding transfer factors. The yields
%are binned in \scalht, jet multiplicity and number of b-jets per
%event. The errors associated with the transfer factors reflect the
%uncertainty due to the finite size of the MC samples used to determine
%the factors. Any trigger inefficiency is also factored into the
%transfer factors (\ie, the trigger is effectively emulated and
%yields from the MC samples are corrected to account for any
%inefficiency). Also, all MC expectations are corrected to account for
%any discrepancies between data and MC for the efficiency and mistag
%rate of the b-tagging algorithm used, as described further in
%Sec.~\ref{sec:btag-eff-correction}. However, no systematic
%uncertainties on the transfer factors are quoted in the tables.
%
%The same tables also list ``na\"ive'' predicted yields obtained from
%each control sample for individual SM backgrounds in the signal region
%(\eg, W + jets and \ttbar from the \mj sample, and \znunu + jets from
%the \mmj and \gj samples). These predictions are given for
%illustrative purposes only. For the analysis result, the predictions
%for the total SM background are determined by a fit to the yields in
%the signal region and all three control samples, as described in
%Sec.~\ref{sec:statistics}. In addition to observed yields, the fit
%takes as input the transfer factors with their associated
%statistical and systematic uncertainties. 
%
%Illustrative predictions for the total SM background can be made for
%each bin in the signal region, by combining the individual
%predictions. One such combination can be made by using the individual
%predictions from the \mj and \mmj samples (or, alternatively, the
%predictions from the \mj and \gj samples), the result of which can be
%compared with the observed yields in the bins of the signal
%region. Predictions in this way are made for bins of the three
%exclusive b-tag categories requiring exactly zero and one b-tags per
%event. When requiring at least two b-tagged jets per event, only the
%\mj sample has sufficiently large yields to predict accurately the
%total SM background. The errors on the total SM predictions reflect
%statistical uncertainties only. It is again noted that these
%``na\"ive'' predictions are for illustration only, with the final SM
%expectations for all signal region bins given by the simultaneous fit
%to yields in all data samples. Table~\ref{tab:tf-summary} summarises
%the contents of the various sections in the Appendix that list tables
%containing observed yields, MC expectations, transfer factors, and
%``na\"ive'' predictions for individual and total SM backgrounds.
%
%\begin{table}[h!]
%  \caption{Each section in Appendix~\ref{app:tf} contains tables that
%    list: observed yields, MC expectations, transfer factors, and
%    ``na\"ive'' predictions for individual SM backgrounds, for each of
%    the \mj, \mmj, and \gj control samples; and total SM predictions
%    when combining the individual predictions from the \mj and \mmj
%    samples and the \mj and \gj samples, separately.}
%  \label{tab:tf-summary}
%  \centering
%  \footnotesize
%  \begin{tabular}{ llll }
%    \hline
%    Section            & \njet bin & \nb bin & Control samples used \\ [0.5ex]
%    \hline
%    \ref{app:23j0b}    & 2--3      & 0       & \mj, \mmj, \gj       \\
%    \ref{app:23j1b}    & 2--3      & 1       & \mj, \mmj, \gj       \\
%    \ref{app:23j2b1mu} & 2--3      & 2       & \mj                  \\
%    \ref{app:4j0b}     & $\geq$4   & 0       & \mj, \mmj, \gj       \\
%    \ref{app:4j1b}     & $\geq$4   & 1       & \mj, \mmj, \gj       \\
%    \ref{app:4j2b1mu}  & $\geq$4   & 2       & \mj                  \\
%    \ref{app:4j3b1mu}  & $\geq$4   & 3       & \mj                  \\
%    \ref{app:4j4b1mu}  & $\geq$4   & $\geq$4 & \mj                  \\
%    \hline
%  \end{tabular}
%\end{table}


%\begin{table}[h!]
%  \caption{}
%  \label{tab:}
%  \centering
%  \footnotesize
%  \begin{tabular}{ lll }
%    \hline
%    \hline
%    \njet bin & \nb bin & Control samples used \\ [0.5ex]
%    \hline
%    2--3      & 0       & \mj, \mmj, \gj       \\
%    2--3      & 1       & \mj, \mmj, \gj       \\
%    2--3      & 2       & \mj                  \\
%    $\geq$4   & 0       & \mj, \mmj, \gj       \\
%    $\geq$4   & 1       & \mj, \mmj, \gj       \\
%    $\geq$4   & 2       & \mj                  \\
%    $\geq$4   & 3       & \mj                  \\
%    $\geq$4   & $\geq$4 & \mj                  \\
%    \hline
%    \hline
%  \end{tabular}
%\end{table}

