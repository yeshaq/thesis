%%%%%%%%%%%%%%%%%%%%%%%%%%%%%%%%
\clearpage
\section{Likelihood model}
\label{sec:statistics}

Consider a given category of event as defined by \njet and \nb.

%%%%%%%%%%%%%%%%%%%%%%%%%%%%%%%%
\subsection{Hadronic sample}
\label{sec:hadronicLikelihood}

Let $N$ be the number of bins of \HT, which need not have equal width.
Let $n^i$ represent the number of events observed satisfying all
selection requirements in each \HT bin $i$.  Then the likelihood of
the observations is written this way:

\begin{equation}
L_{hadronic}=\prod_i \mathrm{Pois}(n^i |\, b^i + s^i)
\label{eq:hadronicLikelihood}
\end{equation}

where $b^i$ represents the expected Standard Model background in bin
$i$, $s^i$ represents the expected number of signal events in bin $i$,
and $\mathrm{Pois}$ represents the Poisson distribution.  It is
assumed that:

\begin{equation}
  b^i\equiv \ewk^i
  \label{eq:ewkTotal}
\end{equation}

where $\ewk^i$ is the expected yield of electroweak events in bin $i$.
By construction, the contribution from QCD events is expected to be negligible, 
Sec.~\ref{sec:qcd}


%%%%%%%%%%%%%%%%%%%%%%%%%%%%%%%%
%\subsection{\texorpdfstring{\HT}{HT} evolution models}
%\label{sec:htEvolutionModel}
%
%The hypothesis that for a process $p$ the $\alt$ ratio falls
%exponentially in \HT can be written this way:
%
%\begin{equation}
%R_{\alt}(\HT) = A e^{-k \HT}
%\end{equation}
%
%where $A$ and $k$ are parameters whose values will be determined.  Let
%$m_i$ represent the number of events observed with $\alt \le 0.55$ in
%each \HT bin $i$, and let \meanHt{i} represent the mean \HT of such
%events.  The expected background from the process is written thus:
%
%\begin{equation}
%b_{p}^{i} = \int_{x_i}^{x_{i+1}}\! \frac{\mathrm{d}N}{\mathrm{d}\HT}
%R_{\alt}\, \mathrm{d}\HT \quad , 
%\end{equation}
%
%where $\frac{\mathrm{d}N}{\mathrm{d}\HT}$ is the distribution of \HT
%for events with $\alt \le 0.55$, $x_i$ is the lower edge of the
%bin, and $x_{i+1}$ is the upper edge of the bin ($\infty$ for the
%final bin).  It is assumed that
%
%\begin{equation}
%\frac{\mathrm{d}N}{\mathrm{d}\HT}(x) = \sum_i
%m^{i}\delta(x-\meanHt{i}) \quad , 
%\end{equation}
%
%\ie within a bin the whole distribution occurs at the mean value of
%\HT in that bin. Then:
%
%\begin{equation}
%b_{p}^{i} = \int_{x_i}^{x_{i+1}}\! m^{i}\delta(x-\meanHt{i}) Ae^{-kx}\, \mathrm{d}x = m^{i} Ae^{-k \meanHt{i}} \quad .
%\label{eq:biDiracExp}
%\end{equation}
%

%%%%%%%%%%%%%%%%%%%%%%%%%%%%%%%%
\subsection{Electroweak control samples\label{sec:ewk}}

Let \fZinv{i} represent the expected yield from \znunu in bin $i$
divided by the expected electroweak background $\ewk^{i}$.  It is
a floating parameter limited between zero and one.

Let:

\begin{equation}
  \zInv{i} \equiv \fZinv{i} \times \ewk^i 
  \label{eq:ZinvEwk}
\end{equation}

\begin{equation}
  ttW^{i} \equiv (1-\fZinv{i})\times \ewk^i
  \label{eq:ttWEwk}
\end{equation}

The variable $\zInv{i}$ thus represents the expected number of \znunu
events in \HT bin $i$ of the hadronically selected sample, and the
variable $ttW^i$ represents the expected number of events from SM
$W$-boson production (including top quark decays) in \HT bin $i$ of
the hadronically selected sample.

In each bin $i$ of \HT, there are three measurements: $n_{ph}^i$,
$n_{\mu}^i$, and $n_{\mu\mu}^i$, representing the event counts in the
photon, single-muon, and double-muon control samples.  Each of these
measurements has a corresponding yield in simulated data: $MC_{ph}^i$,
$MC_{\mu}^i$, and $MC_{\mu\mu}^i$.  The simulation also gives expected
amounts of $\zInv{}$ and $t\bar{t}+W$ in the hadronically-selected
sample: $MC_{\zInv{}}^i$ and $MC_{t\bar{t}+W}^i$.  After defining

\begin{equation}
r_{ph}^i = \frac{MC_{ph}^i}{MC_{\zInv{}}^i};\, r_{\mu\mu}^i =
\frac{MC_{\mu\mu}^i}{MC_{\zInv{}}^i};\, r_{\mu}^i =
\frac{MC_{\mu}^i}{MC_{t\bar{t}+W}^i}\quad ,
\end{equation}

these likelihood functions are used:

\begin{equation}
\label{eq:photonLikelihood}
L_{ph}= \prod_i \mathrm{Pois}(n_{ph}^i |\, \rho_{phZ}^j \cdot
r_{ph}^{i} \cdot \zInv{i})
\end{equation}

\begin{equation}
\label{eq:mumuLikelihood}
L_{\mu\mu}=\prod_i \mathrm{Pois}(n_{\mu\mu}^i |\, \rho_{\mu\mu Z}^j
\cdot r_{\mu\mu}^{i} \cdot \zInv{i})
\end{equation}

\begin{equation}
\label{eq:muonLikelihood}
L_{\mu}=\prod_i \mathrm{Pois}(n_{\mu}^i |\, \rho_{\mu W}^j \cdot
r_{\mu}^{i} \cdot ttW^{i} + s_{\mu}^i)\quad .
\end{equation}

Equation~\ref{eq:photonLikelihood} can be used to estimate the maximum
likelihood value for $\zInv{i}$ (the expectation for the \znunu\ +
jets background in the hadronic signal region) given the observations
$n_{ph}^i$ in the photon control sample and the ratios $r_{ph}^i$. A
similar construction is used when estimating $\zInv{i}$ from the
di-muon control sample (Equ.~\ref{eq:mumuLikelihood}) and $ttW^{i}$
from the single muon control sample
(Equ.~\ref{eq:muonLikelihood}). The measurements in each of the
control samples and the hadronic signal region, along with the ratios
$r_{ph}^{i}$, $r_{\mu\mu}^{i}$, and $r_{\mu}^{i}$, are all considered
simultaneously through the relationships defined in
Equs.~\ref{eq:ewkTotal}, \ref{eq:ZinvEwk}, and
\ref{eq:ttWEwk}. The ratios $r_{ph}^{i}$, $r_{\mu\mu}^{i}$, and
$r_{\mu}^{i}$ are simply the inverse of the translation factor (1/TF)
defined in Equ.~\ref{equ:pred-method}
(Sec.~\ref{sec:background-method}). More specifically, $MC_{ph}^i$,
$MC_{\mu}^i$, and $MC_{\mu\mu}^i$ are the yields obtained from MC
after applying the selection criteria for the photon, single muon and
di-muon samples, as defined by Equ.~\ref{equ:ratio-denom}
(Sec.~\ref{sec:background-method}). The variables $MC_{t\bar{t}+W}^i$ and
$MC_{\zInv{}}^i$ are defined by Equs.~\ref{equ:ratio-numer-mj} and
\ref{equ:ratio-numer-mmj} (Sec.~\ref{sec:background-method}), respectively.

The parameters $\rho_{phZ}^j$, $\rho_{\mu\mu Z}^j$, and $\rho_{\mu
  W}^j$ represent ``correction factors'' that accommodate the
systematic uncertainties associated with the control-sample-based
background constraints.  The quantities $\sigma_{phZ}^j$,
$\sigma_{\mu\mu Z}^j$, and $\sigma_{\mu W}^j$ represent the relative
systematic uncertainties for the control sample constraints, taken
into account with the following terms:

\begin{equation}
\label{eq:ewkSyst}
L_{\rm EWK\, syst.}=\prod_j \mathrm{Logn}( 1.0 |\,\rho_{\mu W}^j,
\sigma_{\mu W}^j)\times \mathrm{Logn}( 1.0 |\,\rho_{\mu\mu Z}^j,
\sigma_{\mu\mu Z}^j)\times \mathrm{Logn}( 1.0 |\,\rho_{phZ}^j,
\sigma_{phZ}^j) \quad ,
\end{equation}

where Logn is the log-normal
distribution~\cite{cousins-log-normal}:

\begin{equation}
\label{eq:log-normal}
\mathrm{Logn}(x |\,\mu,\sigma_{\mathrm{rel.}}) =
\frac{1}{x\sqrt{2\pi}\ln{k}}\exp{\left(-\frac{\ln^2{\left(\frac{x}{\mu}\right)}}{2\ln^2{k}}\right)};\quad
k=1+\sigma_{\mathrm rel.}\quad. 
\end{equation}

Seven (resp. one) parameters per control sample are used to span up to
eleven (resp. four) \HT bins, as shown in Table~\ref{tab:systMap}.

\begin{table}\centering
\caption{The systematic parameters used in \HT bins.  Left: categories
  with up to eleven bins; right: category with four bins.}
\label{tab:systMap}
\footnotesize
\begin{tabular}{lccccccccccc}
\hline
\hline
\HT bin ($i$)         & 0 & 1 & 2 & 3 & 4 & 5 & 6 & 7 & 8 & 9 & 10 \\
\hline
syst. parameter ($j$) & 0 & 1 & 2 & 3 & 3 & 4 & 4 & 5 & 5 & 6 & 6 \\
\hline
\hline
\end{tabular} \ \ 
\begin{tabular}{lccccc}
\hline
\hline
\HT bin ($i$)         & 0 & 1 & 2 & 3\\
\hline
syst. parameter ($j$) & 0 & 0 & 0 & 0\\
\hline
\hline
\end{tabular}
\end{table}

\newcommand{\rpi}{\ensuremath{r_{\mu}^{\prime\ i}}\xspace}

Alternatively, the single muon sample can be used to constrain the
total EWK background thus:

\begin{equation}
\rpi \equiv \frac{MC_{\mu}^i}{MC_{t\bar{t}+W+\zInv{}}^i}\quad ;
\end{equation}
\begin{equation}
\label{eq:muonLikelihoodTotalEwk}
L_{\mu}=\prod_i \mathrm{Pois}(n_{\mu}^i |\, \rho_{\mu W}^j \cdot
\rpi \cdot \ewk^{i} + s_{\mu}^i)\quad .
\end{equation}

The photon and di-muon likelihoods are dropped, as are the parameters
\fZinv{}.

%%%%%%%%%%%%%%%%%%%%%%%%%%%%%%%%
\subsection{Contributions from signal}
\label{sec:signalContrib}

Let $x$ represent the cross section for a particular signal model, and
let $l$ represent the recorded luminosity.  Let $\epsilon^{i}_{had}$
(resp.  $\epsilon^{i}_{\mu}$) be the analysis efficiency as simulated
for the model in \HT bin $i$ of the hadronic (resp. single muon
control) sample.  Let $\delta$ represent the relative uncertainty on
the signal yield, assumed to be fully correlated among the bins, and
let $\rho_{sig}$ represent the ``correction factor'' to the signal
yield which accommodates this uncertainty.  Let $f$ represent an
unknown multiplicative factor on the signal cross section, for which
an allowed interval shall be determined.

Then the expected hadronic signal yield $s^i$ from
Equation~\ref{eq:hadronicLikelihood} is written as $s^i \equiv
f\rho_{sig} xl\epsilon_{had}^i$, and the ``signal contamination'' in
the muon control sample $s_{\mu}^i$ from
Equation~\ref{eq:muonLikelihood} is treated analogously: $s_{\mu}^i
\equiv f\rho_{sig} xl\epsilon_{\mu}^i$.  The systematic uncertainty on
the signal efficiency is included via an additional term in the
likelihood:

\begin{equation}
L_{sig}=\mathrm{Logn}(1.0 |\,\rho_{sig}, \delta) \quad .
\end{equation}

%%%%%%%%%%%%%%%%%%%%%%%%%%%%%%%%
\subsection{Total likelihood}
\label{sec:totalLikelihood}

The likelihood function for a given selection $k$ is the product of
the terms described in the previous sections:

\begin{equation}
L^k = L_{hadronic}^k \times L_{\mu}^k \times L_{ph}^k \times
L_{\mu\mu}^k\times L_{\rm EWK\, syst.}^k \quad .
\end{equation}

In a category with 11 \HT bins and three control samples (single muon,
di-muon, and photon), there are 40 nuisance parameters:
$\{\ewk^{i}\}_{i=0}^{10}$, $\{\fZinv{i}\}_{i=0}^{10}$, $\rho_{phZ}^j$
(with $j=3,4,5,6$), $\rho_{\mu\mu Z}^j$, $\rho_{\mu W}^j$, (with
$j=0,1,2,3,4,5,6$).  In a category with eight \HT bins and one control
sample (single muon), there are 18 nuisance parameters (drop
$\{\fZinv{i}\}_{i=0}^{10}$, $\rho_{phZ}^j$, $\rho_{\mu\mu Z}^j$).  In
a category with three \HT bins and one control sample (single muon),
there are five nuisance parameters: $\ewk^{0,1,2,3}$, $\rho_{\mu
  W}^0$.  When considering signal, there is also the parameter
$\rho_{sig}$; when multiple categories are fit simultaneously, the
total likelihood is

%In a category with eleven \HT bins and three control samples (single
%muon, di-muon, and photon), there are 40 nuisance parameters:
%$\{\ewk^{i}\}_{i=0}^{10}$, $\{\fZinv{i}\}_{i=0}^{10}$, $\rho_{phZ}^j$
%(with $j=3,4,5,6$), $\rho_{\mu\mu Z}^j$, $\rho_{\mu W}^j$, (with
%$j=0,1,2,3,4,5,6$).  In a category with nine \HT bins and one control
%sample (single muon), there are 15 nuisance parameters (drop
%$\{\fZinv{i}\}_{i=0}^{10}$, $\rho_{phZ}^j$, $\rho_{\mu\mu Z}^j$).  In
%a category with three \HT bins and one control sample (single muon),
%there are five nuisance parameters: $\ewk^{0,1,2,3}$, $\rho_{\mu
%  W}^0$.  When considering signal, there is also the parameter
%$\rho_{sig}$; when multiple categories are fit simultaneously, the
%total likelihood is

\begin{equation}
L = L_{sig}\times \prod_k L_{hadronic}^k
\times L_{\mu}^k \times L_{ph}^k \times L_{\mu\mu}^k \times L_{\rm EWK\, syst.}^k \quad .
\end{equation}

