\clearpage
\section{Physics objects\label{sec:reconstruction}}

The definitions of the physics objects used in this analysis follow
the recommendations of the various Physics Object Groups (POGs).

\subsection{Jets}

%Jets are reconstructed from the energy deposits in the calorimeter
%towers, clustered by the anti-$k_{\rm T}$ algorithm~\cite{antikt} with
%a size parameter of $0.5$. The raw jet energies measured by the
%calorimeter systems are corrected to remove the effects of additional,
%overlapping pp collisions (pile-up)~\cite{Cacciari2008119,
%  1126-6708-2008-04-005}, and to establish a uniform relative response
%in $\eta$ and a calibrated absolute response in transverse momentum
%\pt~\cite{Chatrchyan:2011ds}. Jets are identified using a Loose
%working point~\cite{ref:jet-id} as summarised in
%Table~\ref{tab:calojetid} and are subjected to \verb!L1FastJet!,
%\verb!L2!, \verb!L3!, and \verb!L2L3Residual!
%corrections~\cite{ref:jet-jes}.
%
%\begin{table}[!h]
%  \caption{Criteria for the ``loose'' jet ID working point.\label{tab:calojetid}}
%  \footnotesize
%  \begin{center}
%    \begin{tabular}{ll}
%      \hline
%      \hline
%      Requirement                & Definition                                                       \\
%      \hline
%      f$_{HPD} < 0.98$           & Fractional contribution from the ``hottest'' Hybrid Photo Diode. \\
%      f$_{EM} > 0.01$            & Minimum electromagnetic fractional component.                    \\
%      N$^{90}_{\rm hits} \geq$ 2 & Number of channels containing at least 90\% of total energy      \\
%      \hline
%      \hline
%    \end{tabular}
%  \end{center}
%\end{table}


\subsection{b-tagged jets\label{sec:b-tagging}}

%Jets originating from bottom quarks (``b-tagged jets'' or ``b-jets'')
%are identified through vertices that are displaced with respect to the
%primary interaction~\cite{CMS-PAS-BTV-12-001}. The algorithm used to tag
%b-jets is the Combined Secondary Vertex tagger, using the "Medium"
%working point, which is achieved by requiring a cut of $>$0.679 on the
%algorithm discriminator variable and results in a gluon/light-quark
%quark mis-tag rate of 1\% (where ``light'' means $u$, $d$ and $s$
%quarks) and an efficiency in the range $60-70\%$ depending on the jet
%\pt. This tagger is supported by the b-POG~\cite{CMS-PAS-BTV-12-001} and
%its performance is well understood: the efficiency with which jets
%from heavy quarks are identified and the purity of the selection have
%been studied extensively with data and
%simulation~\cite{CMS-PAS-BTV-12-001}.
%%The algorithm performance has been also independently verified using a
%%\ttbar MC sample and the measured efficiency and purity is in
%%agreement with those provided by the b-POG~\cite{CMS-PAS-BTV-12-001}. The
%%mis-tag rate of hadronically-decaying tau leptons has also been
%%studied using MC. The effect on the final hadronic signal selection is
%%found to be negligible. An additional b-jet-enriched QCD multijet
%%sample was also tested and no events where found to survive the signal
%%selection critera.

\subsection{Muons}

%Muons are identified according to the Tight working point definition
%($\sim$95\% efficiency) of the muon identification
%algorithm~\cite{ref:muon-id}. A PF-based ``combined relative''
%isolation~\cite{ref:muon-id} is determined within a cone size $\Delta
%R < 0.4$, and "$\Delta\beta$" corrections are applied to remove the
%effects of pileup. Table~\ref{tab:muon-id} summarizes the
%identification and isolation requirements. This object is used as a
%veto as part of the hadronic signal region definition, as described in
%Section~\ref{sec:vetoes}, and as part of the \mj and \mmj control
%sample selections described in Section~\ref{sec:def-control-samples}.
%
%\begin{table}[h!]
%  \caption{Muon identification (Tight working point).\label{tab:muon-id}}
%  \centering
%  \footnotesize
%  \begin{tabular}{ lc }
%    \hline
%    \hline
%    Global Muon                            & True      \\
%    PFMuon                                 & True      \\
%    $\chi^{2}$ fit                         & $<10$     \\
%    Muon chamber hits                      & $>0$      \\
%    Muon station hits                      & $>1$      \\
%    Transverse impact $d_{xy}$             & $<0.2\mm$ \\
%    Longitudinal dist $d_{z}$              & $<0.5\mm$ \\
%    Pixel hits                             & $>0$      \\
%    Track layer hits                       & $>5$      \\
%    PF Isolation ($\Delta\beta$ corrected) & $<0.12$   \\
%    \hline
%    \hline
%  \end{tabular}
%\end{table}

\subsection{Photons}
%
%Photons are identified according to the Tight working point definition 
%($\sim$70\% efficiency) of the simple cut-based photon identification
%algorithm~\cite{ref:photon-id-egamma}. PF-based
%isolation~\cite{ref:photon-id-egamma} is determined within a cone size
%$\Delta R < 0.3$ and $\rho \times A_{\textrm{eff}}$ corrections are
%applied to remove the effects of
%pileup. Table~\ref{tab:photon-id-egamma} summarises the identification
%and isolation requirements. This object is used as a veto as part of
%the hadronic signal region definition, as described in
%Section~\ref{sec:vetoes}, and as part of the \gj control sample
%selection described in Section~\ref{sec:def-control-samples}.
%
%
%\begin{table}[ht!]
%  \caption{Photon identification (Tight working point).\label{tab:photon-id-egamma}}
%  \centering
%  \footnotesize
%  \begin{tabular}{ ccc }
%    \hline
%    \hline
%    Categories                    & Barrel                             & EndCap                             \\
%    \hline
%    Conversion safe electron veto & Yes                                & Yes                                \\
%    Single Tower H/E              & 0.05                               & 0.05                               \\
%    $\sigma_{i\eta i\eta}$        & 0.11                               & 0.31                               \\
%    PF charged hadron isolation   & 0.70                               & 0.50                               \\
%    PF neutral hadron isolation   & 0.4 + 0.04 $\times$ $\pt^{\gamma}$  & 1.5 + 0.04 $\times$ $\pt^{\gamma}$  \\
%    PF photon isolation           & 0.5 + 0.005 $\times$ $\pt^{\gamma}$ & 1.0 + 0.005 $\times$ $\pt^{\gamma}$ \\
%    \hline
%    \hline
%  \end{tabular}
%\end{table}

\subsection{Electrons}

%Electrons are identified according to the Loose working point
%definition ($\sim$90\% efficiency) of the cut-based \verb!Egamma!
%identification algorithm~\cite{ref:electron-id}. PF-based
%isolation~\cite{ref:electron-isolation} is determined within a cone
%size $\Delta R < 0.3$ and $\rho \times A_{\textrm{eff}}$ corrections
%are applied to remove the effects of pileup. Table~\ref{tab:ele-id}
%summarises the identification and isolation requirements. This object
%is used as a veto as part of the hadronic signal region definition, as
%described in Section~\ref{sec:vetoes}.
%
%\begin{table}[h!]
%  \caption{Electron identification (Loose working point).\label{tab:ele-id}}
%  \centering
%  \footnotesize
%  \begin{tabular}{ lcc }
%    \hline
%    \hline
%    Categories                                               & Barrel    & EndCap    \\
%    \hline
%    $\Delta \eta_{In}$                                       & 0.007     & 0.009     \\
%    $\Delta \phi_{In}$                                       & 0.15      & 0.10      \\
%    $\sigma_{i\eta i\eta}$                                   & 0.01      & 0.03      \\
%    H/E                                                      & 0.12      & 0.10      \\
%    d0 (vtx)                                                 & 0.02      & 0.02      \\
%    dZ (vtx)                                                 & 0.2       & 0.20      \\
%    $\lvert(1/E_{\textrm{ECAL}} - 1/p_{\textrm{trk}})\rvert$ & 0.05      & 0.05      \\
%    PF relative isolation                                    & 0.15      & 0.15      \\
%    Vertex fit probability                                   & 10$^{-6}$ & 10$^{-6}$ \\
%    Missing hits                                             & 1         & 1         \\
%    \hline
%    \hline
%  \end{tabular}
%\end{table}

\subsection{Single isolated tracks}

%A single isolated track (SIT) can be used to identify W bosons through
%their leptonic decays: $\textrm{W} \ra \mu \nu$, $\textrm{W} \ra e
%\nu$, and $\textrm{W} \ra \tau (\ra \ell) \nu$. Also, single prong
%decays of the tau lepton can be identified: $\textrm{W} \ra \tau (\ra
%h^{\pm} + n\pi^{0})\nu$. A single isolated track comprises a charged
%PF candidate that satisfies the requirements listed in
%Table~\ref{tab:sit-id}. The relative track isolation is determined
%from the vectorial sum of neighbouring charged PF candidates within a
%cone $\Delta R < 0.3$ and satisfying $\Delta z (\textrm{candidate,PV})
%< 0.05\cm$ around the candidate isolated track. This object can be
%used to efficiently suppress the ``lost lepton'' background from W and
%\ttbar, as described in Section~\ref{sec:vetoes}. This definition is
%based on the one used in SUS-13-011 (``single lepton stop search with
%transverse mass'')~\cite{singleleptonstop}.
%
%\begin{table}[h!]
%  \caption{Single isolated track identification.\label{tab:sit-id}}
%  \centering
%  \footnotesize
%  \begin{tabular}{ lc }
%    \hline
%    \hline
%    Track \Pt                      & $>10\gev$  \\
%    $\Delta z (\textrm{track,PV})$ & $<0.05\cm$ \\
%    Charge                         & $\neq 0$   \\
%    Relative track isolation       & $<0.1$     \\
%    \hline
%    \hline
%  \end{tabular}
%\end{table}
%
%\subsection{Missing transverse energy}
%
%Missing transverse energy, \met, is defined by the type-I corrected
%particle-flow (PF)-based MET algorithm~\cite{ref:MET-corrections}. The
%\met variable is only used in the following two cases: to define of
%the tranverse mass variable, \mt, which is in turn used as part of the
%selection criteria that define the \mj control sample, described in
%Section~\ref{sec:def-control-samples}; to define a cleaning filter
%applied after the \alphat requirement, as described in
%Section~\ref{sec:had-signal}.
