\clearpage
\subsection{Physics objects\label{sec:reconstruction}}

The definitions of the physics objects used in this analysis follow
the recommendations of the various Physics Object Groups (POGs).

\subsubsection{Jets\label{recJet}}

Jets are reconstructed by combining information from multiple
sub-detectors using the Particle-Flow (PF) algorithm~\cite{PAS-PFT-09-001} 
and clustered by the anti-$k_{\rm T}$ algorithm~\cite{antikt} with
a size parameter of $0.5$. Three levels of jet energy corrections are 
applied; level 1 corrects for overlapping pp collisions 
(pile-up ~\cite{Cacciari2008119,1126-6708-2008-04-005}) in the jet, 
level 2 and 3 correct the jet energy response to be $\eta$ and $\pt$ independent.  
Further residual corrections are applied on data which correct for 
small remaining discripencies in the modelling of the response. The acceptance of
``fake''  and poorly reconstructed jets is supressed by selecting jets that pass 
jet identitifcation at the Loose working point~\cite{ref:jet-id}. This requires 
that the jet is comprised of more than one particle and that those particles 
cannot all be neutral hadrons or all neutral particle deposits in the ECAL. 
Additionaly if the jet is reconstructed beyond the tracker's instrumatation, 
i.e. $|\eta| <$ 2.4,  the jet is required to have more than one charged constituent, 
of which not all of them deposit their energy in the ECAL. 
Only jets reconstructed within the barrel and endcap of the calorimeters,
i.e. $|\eta| <$ 3.0, and with transverse momentum $\pt >$ 50\gev  
are considered in the analysis. Jets originating from bottom quarks 
(b-jets) are identified through vertices that are displaced with respect to the primary 
interaction~\cite{CMS-PAS-BTV-12-001}. The algorithm used to tag b-jets 
is the Combined Secondary Vertex tagger, using the "Medium" working point, 
which is achieved by requiring a cut of $>$0.679 on the algorithm discriminator 
variable and results in a gluon/light-quark quark mis-tag rate of 1\% 
(where ``light'' means $u$, $d$ and $s$ quarks) and an efficiency in the 
range $60-70\%$ depending on the jet \pt. 

\subsubsection{Muons\label{sec:recMuon}}

Muons are identified according to the Tight working point definition
($\sim$95\% efficiency) of the muon identification
algorithm~\cite{ref:muon-id}. The algorithm works to reject cosmic muons or 
muons from decays in flight for consideration in the analysis. 
A PF-based ``combined relative'' isolation~\cite{ref:muon-id} is determined 
within a cone size $\Delta R < 0.4$, and "$\Delta\beta$" corrections 
are applied to remove the effects of pileup. Table~\ref{tab:muon-id} summarizes the
identification and isolation requirements. 

\begin{table}[h!]
  \caption{Muon identification (Tight working point).\label{tab:muon-id}}
  \centering
  \footnotesize
  \begin{tabular}{ lc }
    \hline
    \hline
    Global Muon                            & True      \\
    PFMuon                                 & True      \\
    $\chi^{2}$ fit                         & $<10$     \\
    Muon chamber hits                      & $>0$      \\
    Muon station hits                      & $>1$      \\
    Transverse impact $d_{xy}$             & $<0.2\mm$ \\
    Longitudinal dist $d_{z}$              & $<0.5\mm$ \\
    Pixel hits                             & $>0$      \\
    Track layer hits                       & $>5$      \\
    PF Isolation ($\Delta\beta$ corrected) & $<0.12$   \\
    \hline
    \hline
  \end{tabular}
\end{table}

\subsubsection{Photons\label{sec:recPhot}}
%
Selected photons must satisfy the Tight working point definition 
($\sim$70\% efficiency) of the simple cut-based photon identification
algorithm~\cite{ref:photon-id-egamma}. Pile-up corrected isolation is 
determined within a cone size $\Delta R < 0.3$ using the PF-based 
isolation algorithm~\cite{ref:photon-id-egamma}. 
Table~\ref{tab:photon-id-egamma} summarises the identification
and isolation requirements. 

\begin{table}[ht!]
  \caption{Photon identification (Tight working point).\label{tab:photon-id-egamma}}
  \centering
  \footnotesize
  \begin{tabular}{ ccc }
    \hline
    \hline
    Categories                    & Barrel                             & EndCap                             \\
    \hline
    Conversion safe electron veto & Yes                                & Yes                                \\
    Single Tower H/E              & 0.05                               & 0.05                               \\
    $\sigma_{i\eta i\eta}$        & 0.11                               & 0.31                               \\
    PF charged hadron isolation   & 0.70                               & 0.50                               \\
    PF neutral hadron isolation   & 0.4 + 0.04 $\times$ $\pt^{\gamma}$  & 1.5 + 0.04 $\times$ $\pt^{\gamma}$  \\
    PF photon isolation           & 0.5 + 0.005 $\times$ $\pt^{\gamma}$ & 1.0 + 0.005 $\times$ $\pt^{\gamma}$ \\
    \hline
    \hline
  \end{tabular}
\end{table}
\FloatBarrier
\subsubsection{Electrons\label{sec:recElectron}}

Electrons are identified according to the Veto working point
definition ($\sim$95\% efficiency) of the cut-based \verb!Egamma!
identification algorithm~\cite{ref:electron-id}. PF-based
isolation~\cite{ref:electron-isolation} is determined within a cone
size $\Delta R < 0.3$ and $\rho \times A_{\textrm{eff}}$ corrections
are applied to remove the effects of pileup. Table~\ref{tab:ele-id}
summarises the identification and isolation requirements. 

\begin{table}[h!]
  \caption{Electron identification (Loose working point).\label{tab:ele-id}}
  \centering
  \footnotesize
  \begin{tabular}{ lcc }
    \hline
    \hline
    Categories                                               & Barrel    & EndCap    \\
    \hline
    $\Delta \eta_{In}$                                       & 0.007     & 0.009     \\
    $\Delta \phi_{In}$                                       & 0.15      & 0.10      \\
    $\sigma_{i\eta i\eta}$                                   & 0.01      & 0.03      \\
    H/E                                                      & 0.12      & 0.10      \\
    d0 (vtx)                                                 & 0.02      & 0.02      \\
    dZ (vtx)                                                 & 0.2       & 0.20      \\
    $\lvert(1/E_{\textrm{ECAL}} - 1/p_{\textrm{trk}})\rvert$ & 0.05      & 0.05      \\
    PF relative isolation                                    & 0.15      & 0.15      \\
    Vertex fit probability                                   & 10$^{-6}$ & 10$^{-6}$ \\
    Missing hits                                             & 1         & 1         \\
    \hline
    \hline
  \end{tabular}
\end{table}


\subsubsection{Missing transverse energy\label{sec:recMET}}

Missing transverse energy, \met, is defined as the scalar sum of the transverse momenta 
of all reconstructed objects. The analysis uses the particle-flow MET 
algorithm~\cite{ref:MET-corrections} with jet energy corrections applied. 
The \met variable is only used in the following two cases: to define of the tranverse mass 
variable, \mt, which is in turn used as part of the selection criteria that define the 
\mj control sample, described in section~\ref{sec:def-control-samples}; to define a 
cleaning filter applied after the \alphat requirement, as described in 
section~\ref{sec:finalSelections}.

\subsubsection{Single isolated tracks\label{sec:recSIT}}

A single isolated track (SIT)~\cite{singleleptonstop} can be used to 
identify W bosons through their leptonic decays. A single isolated track 
is identifed if a charged PF object has at least a $\pt$ of 10\gev, 
is near the primary vertex ($\Delta z (\textrm{track,PV}) <0.05\cm$) 
and has a relative track isolation of $<0.1$. This object can be
used to help suppress the ``lost lepton'' background from W and
\ttbar, as described in section~\ref{sec:vetoes}. 


