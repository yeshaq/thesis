\clearpage
\section{Systematic uncertainties on simplified models\label{app:signal}}

\subsection{T2cc\label{app:t2cc}}

%\begin{figure}[h!]
%  \begin{center}
%    \subfigure[$m_{\sTop} = 250\gev, m_{\rm LSP} = 170\gev$]{
%      \includegraphics[width=0.6\textwidth, trim=0 0 0 30, clip=true]{figures/sms/t2cc/v25/T2cc_sig_inj_250_170}
%    } \\
%%    \subfigure[$m_{\sTop} = 250\gev, m_{\rm LSP} = 230\gev$]{
%%      \includegraphics[width=0.6\textwidth, trim=0 0 0 30, clip=true]{figures/sms/t2cc/v25/T2cc_sig_inj_250_230}
%%    } \\
%    \subfigure[$m_{\sTop} = 250\gev, m_{\rm LSP} = 240\gev$]{
%      \includegraphics[width=0.6\textwidth, trim=0 0 0 30, clip=true]{figures/sms/t2cc/v25/T2cc_sig_inj_250_240}
%    } \\
%    \caption{The expected signal significance (in terms of the number
%      of standard deviations) per signal region bin for the
%      \texttt{T2cc} model with $m_{\sTop} = 250\gev$ and $m_{\rm LSP}
%      = 170\gev$ (Top) and $m_{\rm LSP} = 170\gev$ (Bottom).}
%    % the best fit point $m_{\rm LSP} = 170\gev$ (Middle) 
%    \label{fig:sms-t2cc-sig}
%  \end{center}
%\end{figure}

\begin{figure}[h!]
  \begin{center}
    \subfigure[\label{fig:sms-pdf-t2cc-0b_le3j}\njetlow, $\nb = 0$]{
      \includegraphics[width=0.43\textwidth,page=2]{figures/sms/t2cc/v1/t2cc_unc}
    }
%    \subfigure[\label{fig:sms-pdf-t2cc-1b_le3j}\njetlow, $\nb = 1$]{
%      \includegraphics[width=0.43\textwidth,page=9]{figures/sms/t2cc/v1/t2cc_unc}
%    }\\
    \subfigure[\label{fig:sms-pdf-t2cc-0b_ge4j}\njethigh, $\nb = 0$]{
      \includegraphics[width=0.43\textwidth,page=23]{figures/sms/t2cc/v1/t2cc_unc}
    }
    \subfigure[\label{fig:sms-pdf-t2cc-1b_ge4j}\njethigh, $\nb = 1$]{
      \includegraphics[width=0.43\textwidth,page=30]{figures/sms/t2cc/v1/t2cc_unc}
    }\\
    \caption{\label{fig:sms-pdf-t2cc}Ratio of efficiency times
      acceptance for the central value of the envelope calculation relative 
      to the nominal PDF set used to produce the \texttt{T2cc} sample. The categories
      used to interpret \texttt{T2cc} are shown.}
  \end{center}
\end{figure}

\begin{figure}[h!]
  \begin{center}
    \subfigure[\njetlow, $\nb = 0$.]{
      \includegraphics[width=0.35\textwidth, page=7]{figures/sms/t2cc/v1/t2cc_unc}
    }
    \subfigure[\njetlow, $\nb = 0$.]{
      \includegraphics[width=0.35\textwidth, page=6]{figures/sms/t2cc/v1/t2cc_unc}
    }\\
%   \subfigure[\njetlow, $\nb = 1$.]{
%      \includegraphics[width=0.35\textwidth, page=14]{figures/sms/t2cc/v1/t2cc_unc}
%    }
%    \subfigure[\njetlow, $\nb = 1$.]{
%      \includegraphics[width=0.35\textwidth, page=13]{figures/sms/t2cc/v1/t2cc_unc}
%    }\\
    \subfigure[\njethigh, $\nb = 0$.]{
      \includegraphics[width=0.35\textwidth, page=28]{figures/sms/t2cc/v1/t2cc_unc}
    }
    \subfigure[\njethigh, $\nb = 0$.]{
      \includegraphics[width=0.35\textwidth, page=27]{figures/sms/t2cc/v1/t2cc_unc}
    }\\
    \subfigure[\njethigh, $\nb = 1$.]{
      \includegraphics[width=0.35\textwidth, page=35]{figures/sms/t2cc/v1/t2cc_unc}
    }  
    \subfigure[\njethigh, $\nb = 1$.]{
      \includegraphics[width=0.35\textwidth, page=34]{figures/sms/t2cc/v1/t2cc_unc}
    }\\
    \caption{\label{fig:sms-jes-t2cc}The fractional change in
      signal efficiency due to systematically (Left) decreasing and
      (Right) increasing all jet energies by their JES uncertainties. For
      each mass point, the largest value between the two variations is assigned
      as the JES systematic uncertainty. The categories used to interpret \texttt{T2cc} are shown.}
  \end{center}
\end{figure}

\begin{figure}[h!]
  \begin{center}
    \subfigure[\njetlow, $\nb = 0$.]{
      \includegraphics[width=0.35\textwidth, page=1]{figures/sms/t2cc/v1/t2cc_unc}
    }
    \subfigure[\njetlow, $\nb = 0$.]{
      \includegraphics[width=0.35\textwidth, page=4]{figures/sms/t2cc/v1/t2cc_unc}
    }\\
%    \subfigure[\njetlow, $\nb = 1$.]{
%      \includegraphics[width=0.35\textwidth, page=8]{figures/sms/t2cc/v1/t2cc_unc}
%    }
%    \subfigure[\njetlow, $\nb = 1$.]{
%      \includegraphics[width=0.35\textwidth, page=11]{figures/sms/t2cc/v1/t2cc_unc}
%    }\\
    \subfigure[\njethigh, $\nb = 0$.]{
      \includegraphics[width=0.35\textwidth, page=22]{figures/sms/t2cc/v1/t2cc_unc}
    }
    \subfigure[\njethigh, $\nb = 0$.]{
      \includegraphics[width=0.35\textwidth, page=25]{figures/sms/t2cc/v1/t2cc_unc}
    }\\
    \subfigure[\njethigh, $\nb = 1$.]{
      \includegraphics[width=0.35\textwidth, page=29]{figures/sms/t2cc/v1/t2cc_unc}
    }  
    \subfigure[\njethigh, $\nb = 1$.]{
      \includegraphics[width=0.35\textwidth, page=32]{figures/sms/t2cc/v1/t2cc_unc}
    }\\
    \caption{\label{fig:sms-isr-t2cc}The fractional change in signal
      efficiency due to systematically (Left) decreasing and (Right)
      increasing event weights according to ISR uncertainties. For
      each mass point, the largest value between the two variations is assigned
      as the ISR systematic uncertainty. The categories used to interpret \texttt{T2cc} are shown.}
  \end{center}
\end{figure}

\begin{figure}[h!]
  \begin{center}
    \subfigure[\njetlow, $\nb = 0$.]{
      \includegraphics[width=0.35\textwidth, page=3]{figures/sms/t2cc/v1/t2cc_unc}
    }
    \subfigure[\njetlow, $\nb = 0$.]{
      \includegraphics[width=0.35\textwidth, page=5]{figures/sms/t2cc/v1/t2cc_unc}
    }\\
%    \subfigure[\njetlow, $\nb = 1$.]{
%      \includegraphics[width=0.35\textwidth, page=10]{figures/sms/t2cc/v1/t2cc_unc}
%    }
%    \subfigure[\njetlow, $\nb = 1$.]{
%      \includegraphics[width=0.35\textwidth, page=12]{figures/sms/t2cc/v1/t2cc_unc}
%    }\\
    \subfigure[\njethigh, $\nb = 0$.]{
      \includegraphics[width=0.35\textwidth, page=24]{figures/sms/t2cc/v1/t2cc_unc}
    }
    \subfigure[\njethigh, $\nb = 0$.]{
      \includegraphics[width=0.35\textwidth, page=26]{figures/sms/t2cc/v1/t2cc_unc}
    }\\
    \subfigure[\njethigh, $\nb = 1$.]{
      \includegraphics[width=0.35\textwidth, page=31]{figures/sms/t2cc/v1/t2cc_unc}
    }  
    \subfigure[\njethigh, $\nb = 1$.]{
      \includegraphics[width=0.35\textwidth, page=33]{figures/sms/t2cc/v1/t2cc_unc}
    }\\
    \caption{\label{fig:sms-btag-t2cc}The fractional change in signal
      efficiency due to systematically (Left) decreasing and (Right)
      increasing all b-tag efficiencies according to the scale factor
      uncertainties. For each mass point, the largest value between the 
      two variations is assigned as the b-tag systematic uncertainty. 
      The categories used to interpret \texttt{T2cc} are shown.}
  \end{center}
\end{figure}

%\begin{figure}[h!]
%  \begin{center}
%    \subfigure[\njetlow, $\nb = 0$.]{
%     \includegraphics[width=0.48\textwidth,page=1]{figures/sms/t2cc/v1/t2cc_pfJet_totalUnc.pdf}
%    }                                                                  
%    \subfigure[\njetlow, $\nb = 1$.]{                                  
%     \includegraphics[width=0.48\textwidth,page=2]{figures/sms/t2cc/v1/t2cc_pfJet_totalUnc.pdf}
%    }\\                                                                
%    \subfigure[\njethigh, $\nb = 0$.]{                                 
%      \includegraphics[width=0.48\textwidth,page=5]{figures/sms/t2cc/v1/t2cc_pfJet_totalUnc.pdf}
%    }                                                                  
%    \subfigure[\njethigh, $\nb = 1$.]{                                 
%      \includegraphics[width=0.48\textwidth,page=6]{figures/sms/t2cc/v1/t2cc_pfJet_totalUnc.pdf}
%    }\\
%    \caption{\label{fig:sms-total-t2cc}The total systematic
%      uncertainty in the signal efficiency times acceptance for all
%      relevant event categories for the \texttt{T2cc} intepretation.}
%  \end{center}
%\end{figure}

%\FloatBarrier
%%%%%%%%%%%%%%%%%%%%%%%%%%%%%%%%%%%%%%%%%%%%%%%%%%%%%%%%%%%%%%%%%%%%%%%%%%%%%%%%
%%%%%%%%%%%%%%%%%%%%%%%%%%%%%%%%%%%%%%%%%%%%%%%%%%%%%%%%%%%%%%%%%%%%%%%%%%%%%%%%
%%%%%%%%%%%%%%%%%%%%%%%%%%%%%%%%%%%%%%%%%%%%%%%%%%%%%%%%%%%%%%%%%%%%%%%%%%%%%%%%

\clearpage
\subsection{T2tt\label{app:t2tt}}

%
\begin{figure}[h!]
  \begin{center}
    \subfigure[\label{fig:sms-pdf-t2tt-1b_ge4j}\njethigh, $\nb = 1$]{
      \includegraphics[width=0.43\textwidth,page=2]{figures/sms/t2tt/v1/t2tt_unc}
    }
    \subfigure[\label{fig:sms-pdf-t2tt-2b_ge4j}\njethigh, $\nb = 2$]{
      \includegraphics[width=0.43\textwidth,page=2]{figures/sms/t2tt/v1/t2tt_unc}
    }\\
    \caption{\label{fig:sms-pdf-t2tt}Ratio of efficiency times
      acceptance for the central value of the envelope calculation relative 
      to the nominal PDF set used to produce the \texttt{T2tt} sample. The categories
      used to interpret \texttt{T2tt} are shown.}
  \end{center}
\end{figure}

\begin{figure}[h!]
  \begin{center}
    \subfigure[\njethigh, $\nb = 1$.]{
      \includegraphics[width=0.35\textwidth, page=14]{figures/sms/t2tt/v1/t2tt_unc}
    }
    \subfigure[\njethigh, $\nb = 1$.]{
      \includegraphics[width=0.35\textwidth, page=13]{figures/sms/t2tt/v1/t2tt_unc}
    }\\
    \subfigure[\njethigh, $\nb = 2$.]{
      \includegraphics[width=0.35\textwidth, page=21]{figures/sms/t2tt/v1/t2tt_unc}
    }
    \subfigure[\njethigh, $\nb = 2$.]{
      \includegraphics[width=0.35\textwidth, page=20]{figures/sms/t2tt/v1/t2tt_unc}
      }\\     
      \caption{\label{fig:sms-jes-t2tt}The fractional change in
      signal efficiency due to systematically (Left) decreasing and
      (Right) increasing all jet energies by their JES uncertainties. For
      each mass point, the largest value between the two variations is assigned
      as the JES systematic uncertainty. The categories used to interpret \texttt{T2tt} are shown.}
  \end{center}
\end{figure}

\begin{figure}[h!]
  \begin{center}
    \subfigure[\njethigh, $\nb = 1$.]{
      \includegraphics[width=0.35\textwidth, page=8]{figures/sms/t2tt/v1/t2tt_unc}
    }
    \subfigure[\njethigh, $\nb = 1$.]{
      \includegraphics[width=0.35\textwidth, page=11]{figures/sms/t2tt/v1/t2tt_unc}
    }\\
    \subfigure[\njethigh, $\nb = 2$.]{
      \includegraphics[width=0.35\textwidth, page=15]{figures/sms/t2tt/v1/t2tt_unc}
    }
    \subfigure[\njethigh, $\nb = 2$.]{
      \includegraphics[width=0.35\textwidth, page=18]{figures/sms/t2tt/v1/t2tt_unc}
    }\\
    \caption{\label{fig:sms-isr-t2tt}The fractional change in signal
      efficiency due to systematically (Left) decreasing and (Right)
      increasing event weights according to ISR uncertainties. For
      each mass point, the largest value between the two variations is assigned
      as the ISR systematic uncertainty. The categories used to interpret \texttt{T2tt} are shown.}
  \end{center}
\end{figure}

\begin{figure}[h!]
  \begin{center}
    \subfigure[\njethigh, $\nb = 1$.]{
      \includegraphics[width=0.35\textwidth, page=10]{figures/sms/t2tt/v1/t2tt_unc}
    }
    \subfigure[\njethigh, $\nb = 1$.]{
      \includegraphics[width=0.35\textwidth, page=12]{figures/sms/t2tt/v1/t2tt_unc}
    }\\
    \subfigure[\njethigh, $\nb = 2$.]{
      \includegraphics[width=0.35\textwidth, page=17]{figures/sms/t2tt/v1/t2tt_unc}
    }
    \subfigure[\njethigh, $\nb = 2$.]{
      \includegraphics[width=0.35\textwidth, page=19]{figures/sms/t2tt/v1/t2tt_unc}
    }\\
    \caption{\label{fig:sms-btag-t2tt}The fractional change in signal
      efficiency due to systematically (Left) decreasing and (Right)
      increasing all b-tag efficiencies according to the scale factor
      uncertainties. For each mass point, the largest value between the 
      two variations is assigned as the b-tag systematic uncertainty. 
      The categories used to interpret \texttt{T2tt} are shown.}
  \end{center}
\end{figure}

