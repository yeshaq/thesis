\clearpage
\section{Introduction and motivation\label{sec:intro}}

In this note we present an update of the search for a missing energy
signature in dijet and multijet events using the kinematic variable
\alphat, as first introduced in Refs.~\cite{Randall:2008rw,
  cms-pas-sus-08005, cms-pas-sus-09001} and described in
Sec.~\ref{sec:alphat}. 

This analysis follows closely that described in
Ref.~\cite{RA1Paper2012} and the corresponding Analysis
Note~\cite{RA1Paper2012ANHCP}. All changes to the analysis concern the
goal of improving sensitivity to compressed-spectrum models with small
mass splittings ($\Delta m$) between the parent (\ie gluino or squark)
and daughter (\ie LSP) SUSY particles. The most significant changes
are listed in (the previous) Section~\ref{sec:changes}.

These results are based on a data sample of pp collisions collected in
2012 at a centre-of-mass energy of 8 TeV, which corresponds to an
integrated luminosity of 18.5$\pm$0.5\fbinv. The data sample comprises
events recorded with the same signal triggers as used in
Ref.~\cite{RA1Paper2012} plus a new ``parked'' trigger that allows the
signal region phase space to further enlarged with respect to the
previous analysis. This ``parked'' trigger was not available for the
run period \verb!Run2012A!, which is the reason why the intergrated
luminosity is marginally lower than that typically quoted by other
analyses (\ie $\sim$19--20\fbinv).

A search for an excess of events in data over the Standard Model
expectation is performed in multijet final-states with significant
\met. The dominant background is multijet production, a manifestation
of quantum chromodynamics (QCD), which is suppressed by very tight
requirements on the \alphat variable to a negligible level. To
estimate the remaining significant backgrounds, we make use of four
data control samples: a \mj sample to determine the background from
\wj, \ttbar and single top events; a \gj sample to determine the
irreducible background from \znunu\ + jets events; a \mmj sample that
is also used to determine the \znunu\ + jets background; and finally a
multijet-enriched hadronic control sample to determine any residual
contribution from multijet production.

%\subsection{Focus of this analysis\label{sec:parked}}
%
%\fixme
%
%Parked data and compressed-spectrum models.
%
%This search focuses on event topologies in which new heavy particles
%are pair-produced, each of which then decays to a weakly interacting
%massive particle (WIMP) that remains undetected, thus leading to a
%missing energy signature. In the case of SUSY, the candidate heavy
%particles are squarks and gluinos and the WIMP candidate is the
%lightest (and stable) neutralino $\chiznew_1$.
%
%Thus, this search requires at least two high-\pt jets and significant
%\met in the final state. 
%
%the search has been adapted to
% improve the sensitivity to final-state signatures rich in heavy
% quarks; 
%
%The results presented below are interpreted
%in the context of SUSY, although they are also applicable to other New
%Physics scenarios that are characterised by a missing transverse
%energy signature, such as Extra Dimensions and Little Higgs models. 
%
%The results are interpreted using simplified model spectra
%(SMS)~\cite{Alwall:2008ag,Alwall:2008va,sms}. A simplified model is
%defined by an effective Lagrangian describing the interactions of a
%small number of new particles, which can be equally well described by
%a small number of observables, such as masses and cross-sections.
%Simplified models are therefore particularly useful for evaluating
%phase-space coverage of both individual searches and experiment-wide
%search programs, as well as providing an excellent starting point for
%characterizing positive signals of new physics.
%
%The results presented below are interpreted in the context of SUSY,
%although they are also applicable to other New Physics scenarios that
%are characterised by a missing transverse energy signature, such as
%Extra Dimensions and Little Higgs models.
