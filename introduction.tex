\clearpage
\section{Introduction\label{sec:intro}}

The past few years have been an exciting time in experimental
particle physics. The Large Hadron Collider, first turned on in 2008,
proved to be a resounding success by providing proton-proton collisions
at a record-breaking energy of 8~\TeV. The detectors surrounding the
collider have been able to make some of the most precise measurements
of the properties of particles ever measured, and discovered new resonances. 
One of the primary goals of the experiments was fulfilled when the Higgs 
boson was discovered in 2012. With the discovery, our current knowledge of particle physics was solidified.  
But glaring and important questions remain: What is the nature of the
dark matter that makes up over 96\% of the matter in the universe?
How does gravity behave at the quantum scale? What mechanism gives the
fundamental forces such varying strengths? Theories have been proposed
to answer such questions and this dissertation describes an experimental
search conducted and interpreted in the framework of such a theory.
The analysis conducted is a natural extension of two previous analyses I have directly
contributed to~\cite{RA1Paper2011FULL,RA1Paper2012ANHCP}, but it utilizes more 
complex objects than previously done.
